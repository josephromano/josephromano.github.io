\documentclass[11pt]{NSF}
\sloppy

\usepackage{latexsym}
\usepackage{graphicx}
\usepackage{draftcopy}
\usepackage{longtable}
\usepackage{hyperref}

% some definitions
 
% begin equation, itemize, etc.

\def\be{\begin{equation}}
\def\ee{\end{equation}}
\def\bi{\begin{itemize}}
\def\ei{\end{itemize}}
\def\ben{\begin{enumerate}}
\def\een{\end{enumerate}}
\def\i{\item{}}

%%%%%%%%%%%%%%%%%%%%%%%%%%%%%%%%%%%%%%%%%%%%%%%%%%%%%%%%%%%%%%
            
\begin{document}

\section{6. Spectrum Analysis of Instruments and Voice}

PURPOSE AND BACKGROUND

We continue with our discussion of harmonics. 
All musical instruments produce tones that
are unique and have a characteristic {\em timbre} or quality of sound. The
frequency spectra tell the harmonic content of a tone and how it can
be synthesized. Electronic keyboards make use of this to reproduce
sounds. We analyze the sound from a variety of string instruments, 
wind instruments, and the human voice.

\subsection{String Instruments}

All musical instruments use a driving force to set an oscillator into
motion. Stringed instruments use a bow or plucking for exciting the
vibrations. 
The strings of a violin are tuned to the notes G3, D4, A4, and E5;
for the guitar, they are E2, A2, D3, G3, B3 and E4. You see that the
note G3 is common to both instruments. 
Figure~\ref{f:1} shows an example of a frequency spectrum from the 
open G3 strings of a guitar (top) and violin (bottom). \\

Play G3 on the violin by plucking the lowest string and Observe the different frequency
spectra in the FFT (Fast Fourier Transform) mode of the Electroacoustics Toolbox.
Then play the same G3 on the guitar and observe its frequency spectrum. 

\underline{Question 1}: 
What are some simularties and differences between the tambre and the frequency spectrum from the violin and guitar? 
\\
\\
\\
\\
\\

We may also be interested in comparing different frequency spectrums created by the same instrument.
Play G3, the lowest note on the violin, by bowing and by plucking. 

\underline{Question 2}: 
What are some simularties and differences between the frequency spectrum from the plucked and the bowed string?
\\
\\
\\
\\
\\

Placing a finger down on the fingerboard reduces the effective length of the string and increases
the pitch. Now repeat the above experiment plucking the violin string at a higher pitch. 
\\
\\
\\
\\

\underline{Question 3}: 
How did the frequency spectrums change between the lower and higher pitch.
\\
\\
\\
\\
\\
%
\begin{figure}[hbtp]
\begin{center}
\includegraphics[width=.6\textwidth]{fig6_1}
\caption{Frequency spectra of the plucked open G3 strings of a guitar
(top) and violin (bottom). The fundamental frequency
(pitch) and frequencies of the harmonics are the same. But the timbre
(quality of sound) is very different because of the different relative
amplitudes of the harmonics.}
\label{f:1}
\end{center}
\end{figure}


\subsection{Wind Instruments} 
Wind instruments use air as the vibrating medium. Brass instruments
have closed pipes, with the closed end near the mouth. Woodwinds such
as the clarinet, oboe and bassoon also are closed pipes, with a reed
at the closed end. The lowest harmonics of these instruments are
primarily the odd harmonics, as is expected for closed pipes. This
applies especially to the clarinet due to its straight cylindrical
bore. For other wind instruments, all harmonics are present without a
special dominance of odd or even harmonics. Flutes, piccolos,
recorders and, more exotically, an ocarina, are open pipes with even
and odd harmonics.  \\

Use a slide whistle and obtain the frequency spectrum of its lowest note.
Figure~\ref{f:2} shows an example of the simple, almost purely sinusoidal, frequency
spectrum from a slide whistle. Try to over-blow the lowest note and note the next
harmonic f3 of the closed pipe. Blowing harder may produce f5 and even f7.

\underline{Question 4}: 
What is the frequency range of the slide whistle by moving its piston?
How does the frequency spectrum of the slide whistle compare to that of the violin? 
\\
\\
\\
\\
\\
\\

Play and record the sound spectra from a flexible corrugated plastic tube (“whirly”) by swirling
it around in a circle. \\

\underline{Question 5}: Determine the harmonic numbers N that are active in the spectrum of the corrugated plastic
tube. Note that the fundamental most likely does not show. What are the musical intervals
between the harmonics that are present (e.g. octave, fifth, fourth, third)?
\\
\\
\\
\\
\\

%
\begin{figure}[hbtp]
\begin{center}
\includegraphics[width=.6\textwidth]{fig6_2}
\caption{Frequency spectrum of a slide whistle with 
$f =880~\textrm{Hz}$.}
\label{f:2}
\end{center}
\end{figure}


\subsection{Voice}

The human vocal tract is an intricate system for producing sound. The
voice of each person is unique. The sound is produced, and its quality
determined, by the vocal tract consisting of throat, nasal cavity, and
mouth. Each of these components acts as a resonator with
characteristic resonant frequencies. The different vowel sounds come
from different regions of the vocal tract. This allows for a large
variety of sounds, but some general characteristics exist. \\

\underline{Question 6}: 
The human throat has a typical overall length of 17 cm. Consider
it as a simple pipe, with one end closed at the vocal folds and the
other open at the mouth. What is the fundamental resonance frequency?
What are the frequencies of the next three harmonics?
\\
\\
\\
\\
\\

Have a male and female student sing the vowels “oo” or “ah” into the microphone. Observe the
resulting frequency spectrum with the FFT tool. Figure 3 shows such a spectrum for a male voice
singing “ah” with a pitch of 220 Hz. \\

The frequency regions where several neighboring harmonics have high
amplitudes are called {\em vocal formants}. Most people have similar
formants because of the similar size and shape of their vocal tracts. 
The individual resonators of the tract produce the different formants.
They can be adjusted by a change in size and shape of the throat,
nasal cavity, and mouth. How this is done distinguishes a great singer
from a bad one. Vocal formants are what we listen to in order to
recognize persons. Adjusting the cavities of the vocal tract changes
the formant regions. Adjusting the tension in the vocal cords changes
the pitch and associated harmonics. \\

Figure~\ref{f:4} shows the frequency spectra of a male voice (top) 
and female voice (bottom) making the vowel sound ``ee".
Note the dominant formant regions.

Have two or more students with noticeably different voices sing the vowel sounds “oo”,
“ah”, and “ee” into the microphone. Acquire the frequency spectra. Compare the formant
regions of the students. In Table 1, record the first and second formant region for the two
students. An example for the vowel sound “ee” from a male and female student is shown in
Figure 4

\begin{table}[hbtp]
\begin{center}
Vowel sounds and corresponding vocal formant regions\\
\begin{tabular}{| c | c | c | c | c | }
\hline
&\multicolumn{2}{c}{Student 1} \vrule
&\multicolumn{2}{c}{Student 2} \vrule\\
\hline
Sound & \phantom{ }1st Formant region \phantom{ } & 2nd Formant region & \phantom{ }1st Formant region\phantom{ }\  & 2nd Formant region\\
\hline
oo &  &  &  &  \\
\hline
ah  &  &  &  &  \\
\hline
ee  &  &  &  &  \\
\hline
\end{tabular}
\caption{Table}
\label{t:1}
\end{center}
\end{table}

\underline{Question 7}:  
Write a summary on the findings from these formant region measurements. 
How did formant regions compare between different vowel sounds? 
Compare the differences in sound and formants between the two students vowels.
\\
\\
\\
\\
\\
\\

\underline{Question 8}: 
Telephones transmit frequencies only in the approximate range
300-3000~Hz. Why do you think this frequency range is 
sufficient for most purposes? \\
\\
\\
\\

The human vocal tract produces various types of sound. Continuant sounds are consonant sounds
such as “m” and “n” that have a soft continuous tone. Sibilant sounds are consonants such as “s”
and “z” that can be continuous and sound rather harsh. Plosive sounds are short and explosive
like “p” and “t”. Observe some of these sounds and their frequency spectra.

\underline{Question 9}: 
Which vocal sounds sound more “musical”? Hint: Which sounds have a discrete frequency
spectrum as compared to a more continuous spectrum with many closely spaced frequencies
characteristic of noise? 
\\
\\
\\
\\
\\
%
\begin{figure}[hbtp]
\begin{center}
\includegraphics[width=.6\textwidth]{fig6_4}
\caption{Vowel sound ``ee” from a male voice (top) and 
female voice (bottom). 
The female voice has purer harmonics. 
Note the formant regions.}
\label{f:4}
\end{center}
\end{figure}


\end{document}

%%%%%%%%%%%%%%%%%%%
\begin{table}[hbtp]
\begin{center}
\begin{tabular}{ |p{0.8in} | p{0.8in} | p{0.9in}|}
\hline
Sound & 1st formant region (Hz) & 2nd formant region (Hz) \\
\hline
oo & & \\  
\hline
ah & & \\  
\hline
ee & & \\  
\hline
\end{tabular}
\caption{}
\label{t:}
\end{center}
\end{table}

\begin{figure}[hbtp]
\begin{center}
\includegraphics[width=.6\textwidth]{fig6_3}
\caption{}
\label{f:}
\end{center}
\end{figure}

\begin{figure}[hbtp]
\begin{center}
\includegraphics[width=.6\textwidth]{fig6_5}
\caption{}
\label{f:}
\end{center}
\end{figure}



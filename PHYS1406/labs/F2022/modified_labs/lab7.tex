\documentclass[11pt]{NSF}
\sloppy

\usepackage{latexsym}
\usepackage{graphicx}
\usepackage{draftcopy}
\usepackage{longtable}
\usepackage{hyperref}

% some definitions
 
% begin equation, itemize, etc.

\def\be{\begin{equation}}
\def\ee{\end{equation}}
\def\bi{\begin{itemize}}
\def\ei{\end{itemize}}
\def\ben{\begin{enumerate}}
\def\een{\end{enumerate}}
\def\i{\item{}}

%%%%%%%%%%%%%%%%%%%%%%%%%%%%%%%%%%%%%%%%%%%%%%%%%%%%%%%%%%%%%%
            
\begin{document}

\section{7. Sound Intensity, Hearing, Just Noticeable Difference (JND)} 

PURPOSE AND BACKGROUND

We can hear a wide range of sound intensities and frequencies. The
intensity between the threshold of hearing and the threshold of pain
varies by a factor of $10^{12}$, i.e., by 12 orders of magnitude or 120
decibel. The corresponding range in the amplitude of {\em air pressure 
fluctuations} is a factor of $10^6$. 
In view of this extreme range in sound intensity level, 
numbers are most conveniently expressed in power-of-ten
notation and with a decibel or dB-scale.  

Here we study sound intensity levels (SIL) and the frequency response of
the human ear. We also discuss ``just noticeable differences” (JND) in
intensity and frequency that the ear can discern.  

The ear is sensitive to a range in frequencies from about 20~Hz to 20~kHz.  
This audible range thus covers a factor of 1000 or $10^3$ in frequency, 
which is not nearly as large as the intensity range of $10^{12}$.  
In order to cover these large ranges, the ear
response is compressed or logarithmic with respect to both frequency
and sound intensity.

\subsection{Theory and Experiment}

The amplitude of a sound wave corresponds to air pressure fluctuations
(compressions and rarefactions of the air) in a longitudinal wave.

The {\em threshold of hearing} is a sound intensity at the ear of 
%
\be
I_0=1\times 10^{-12}~\textrm{W/m}^2
\quad{\rm at}\quad
f = 1000~\textrm{Hz}
\ee
%
This is the reference intensity for sound intensity measurements.
 
The {\em sound intensity level} (SIL) is defined by comparing
any intensity $I$ to the threshold of hearing $I_0$ 
according to 
%
\be
\rm{SIL} = 10\,\log({I}/{I_0})~{\rm dB}
\label{e:SIL}
\ee
%
where the logarithm is taken to the base 10.
The inverse equation is
%
\be
I=I_0\,10^{({\rm SIL}/10~{\rm dB})}
\label{e:inverse}
\ee
%
SIL is measured in {\em decibels} or dB.

For example, let the sound intensity in a room be 
$I = 1\times 10^{-6}~{\rm W/m}^2$. 
The SIL is then 
\be
\rm{SIL} 
= 10\,\log\left(\frac{1\times 10^{-6}}{1\times 10^{-12}}\right)~{\rm dB}
=10\, \log 10^6~{\rm dB} = 10\times 6~{\rm dB} = 60~{\rm dB}
\ee

The SIL also can be used to express a change in intensity from one 
value to another, without referring to the threshold of hearing $I_0$. 
We are then dealing with a {\em change} in SIL, denoted by 
$\Delta\,{\rm SIL}$, and not the SIL itself. 

For instance, if the intensity $I$ doubles to $2I$, we have
%
\be
\Delta\,{\rm SIL} = 10\,\log (2I/I)~{\rm dB}
= 10\,\log 2~{\rm dB} 
= 10\times 0.3~{\rm dB} 
= 3~{\rm dB}
\ee
Therefore, a doubling in intensity corresponds to an increase of 3 dB in the SIL.


\underline{Question 1} Use a sound level meter and find the SIL of the background noise in the room. There always is
ambient noise from air conditioners, computer fans etc. The sound intensity level in a typical
environment generally is much higher than the threshold of hearing. What is the measured SIL of
the background noise in our laboratory? 
\[ SIL=\rule{1.5cm}{0.15mm}dB\]

\underline{Question 2} What is the sound intensity I of this background noise, expressed in units of $\frac{W}{m^2}$? Hint: Use equation 3.
\[I=\rule{1.5cm}{0.15mm}\frac{W}{m^2}\]

\underline{Question 3} Use the FEaT Sound Level Meter software and record the sound intensity level of one student
clapping
\[ SIL_1=\rule{1.5cm}{0.15mm}dB\]

\underline{Question 4} Calculate the theoretical increase in sound intensity level, if the intensity $I_10$ for ten students
clapping is ten times the intensity $I_1$ for one:
\[ SIL_10 - SIL_1=\rule{1.5cm}{0.15mm}dB\]

\underline{Question 5} Have 10 students clap. Measure the actual value and record it here.
\[ SIL_10=\rule{1.5cm}{0.15mm}dB\]

\underline{Question 6} At a frequency $f =1000~{\rm Hz}$, an intensity of $I = 1~{\rm W/m}^2$ 
becomes quite painful to the ear.
What is the sound intensity level in dB of a 1000~Hz sinusoidal 
tone at the threshold of pain?
\[ SIL=\rule{1.5cm}{0.15mm}dB\]




\subsection{Frequency Response of the Ear}

The ear can hear sound over a wide range of frequencies 
from about 20~Hz to 20~kHz. 
However, the perceived loudness varies quite dramatically
with frequency. The so-called {\em Fletcher-Munson curves} in
Figure~\ref{f:1} show
lines of equal perceived loudness. 
%
\begin{figure}[hbtp]
\begin{center}
\includegraphics[width=.9\textwidth]{fig7_1}
\caption{Fletcher-Munson curves of equal loudness. 
(From ``Physics of Sound” by R.A.~Berg and D.G.~Stork.)}
\label{f:1}
\end{center}
\end{figure}
%
The curve at the bottom marked 
``0~phons” represents the threshold of hearing, and the line marked 
``120~phons” represents the threshold of pain. 
Each curve has a ``phon” designation and
indicates equal perceived loudness as a function of frequency. 
The ``decibel” and ``phon” scales agree by convention at a frequency
of 1000~Hz (see Figure~\ref{f:1}). For example, if a loudspeaker produces a
1000~Hz tone with ${\rm SIL} = 60~{\rm dB}$ at your location, 
you perceive this sound intensity as a loudness  of 60~phon. 
If on the other hand the speaker produces
a tone at 100~Hz with the same ${\rm SIL} = 60~{\rm dB}$, 
you hear this as less loud than the 1000~Hz tone. 
In order for the two frequencies to sound
equally loud, the speaker must produce the 100 Hz tone at about 
${\rm SIL} = 70~{\rm dB}$ instead. 
Verify this on the curve labeled ``60 phons”.

You can also see from Figure~\ref{f:1} that the human ear is most sensitive to
sound around 4000 Hz, where the Fletcher-Munson curves dip lowest.
Therefore, if you follow a Fletcher-Munson curve from 4000 Hz to lower
frequencies, the sound intensity must be raised to be perceived as
equally loud. The same applies to higher frequencies above 4000 Hz.

Open three Signal Generator tools in the FEaT software. Set them to frequencies of 100, 1000,
4000 Hz. Set the Master Volume of all three tools to 20\% maximum. Use the volume knob on
the stereo receiver to adjust the f =1000 Hz tone to 60 dB on a calibrated Sound Level Meter.
7-4
Adjust the Master Volume of the other two tools for a perceived loudness equal to that of the
1000 Hz tone. 

\underline{Question 7} What is the measured SIL of the 100Hz tone?

\underline{Question 8} What is the measured SIL of the 4000 Hz tone?

\underline{Question 9} From the Fletcher-Munson curve labeled “60 phons” in Figure 1, read the SIL at 100 Hz and
4000 Hz. How close are your measurements to the dB-values on the 60-phon curve? 

\subsection{Loudness in Sones}

The decibel values that we have discussed above are based on 
{\em objective} measurements of the sound intensity. 
There also exists a {\em subjective} ``sone" scale that tells 
what sounds ``twice as loud” to many persons. 
Such a ``twice as loud curve” is shown as a straight line in 
Figure~\ref{f:3}. 
On the sone scale, 1~sone corresponds to a loudness level of 
40~phon for a pure sine wave with $f = 1000~{\rm Hz}$. 
(Recall that for the special case of a pure tone at a frequency 
of 1000~Hz, the number of phon is the same 
as the number of dB.)
%
\begin{figure}[hbtp]
\begin{center}
\includegraphics[width=.6\textwidth]{fig7_3}
\caption{Sone scale, with ``twice as loud” meaning a doubling in the
sone number. The reference is 1~sone at a loudness level of 40~phon.
The phon scale is the same as the dB scale for a pure tone at 1000~Hz.}
\label{f:3}
\end{center}
\end{figure}

Figure~\ref{f:3} shows that, in order for sound to be perceived as twice as
loud, the sound intensity level must be higher by 10 phon (or 10~dB at
1000~Hz).
For example, for an increase in loudness from 1 sone to 2 sone, the 
sound intensity increases by 10~phon from 40~phon to 50~phon. 
Generally, for every increase in sound intensity by 10~phon, the sone 
number doubles. 
Example: For a doubling in loudness from 4 to 8 sone, the sound intensity 
increases from 60 to 70~phon.

\underline{Question 10} Start with a 1000 Hz sine tone at SIL = 60 dB and increase the intensity without looking at
the sound level meter until you perceive the sound as twice as loud. By how many dB did the
SIL increase? What would you expect the increase in dB to be according to the theory above?

\underline{Question 11} According to Figure~\ref{f:3}, what is the increase in phon for a doubling in loudness from 10 to 20
sone? \\

\underline{Question 12} How many times louder does a 90 phon tone sound than a 60 phon tone? 


Application:
The sone scale is used for specifying the loudness of fans and
appliances. For instance, quiet bathroom fans have a rating of 1 to 
2~sones; louder ones have a rating of 3 to 4~sones or more.

\subsection{Just Noticeable Difference in Intensity}

The just noticeable difference (JND) in intensity is the smallest
change in SIL that the ear can discern. Usually a 25\% or 1~dB change in
intensity is detected. This depends somewhat on sound intensity and
frequency as can be seen in Figure 2. As the intensity or frequency
decreases, the ear becomes less sensitive to changes in intensity.

\underline{Question 13} Express a 25\% change in intensity $I$ as a change in dB. 
Hint: Calculate $\Delta\,{\rm SIL}$ for $I_2 = 1.25 I_1$.

Use an external function generator (without the computer) that produces sine waves and square
waves. Play the sound through a loudspeaker. Use a portable sound level meter to read the sound
intensity level in the room. Play a sine wave. Adjust the SIL on the signal generator so that it
reads 80 dB. Increase the intensity slowly until you hear a change in intensity.

\underline{Question 14} What is your measured JND from the sound level meter readings for a sine wave? 

Use two signal generators at 1000 Hz and switch quickly between them. Keep switching between
the generators while you change the SIL on one of them.

\underline{Question 15} What is your JND when changing the intensity quickly? Compare with a slow change.

\underline{Question 16} What is the value for the JND in Figure~\ref{f:2} on the 1000 Hz curve at 80 dB? 
Compare your values for this from questions 14
and 15 with the value from Figure~\ref{f:2}

\underline{Question 17} Compare the JND of a square wave at f = 1000 Hz with that of a sine wave. Alternate
quickly between the two types of waves. For which do you get a smaller JND, i.e. for which can
you hear smaller differences in SIL? Can you give a reason for this? 

Give a reason for your answer. (Hint: Consider the harmonics in the square wave.)

%
\begin{figure}[hbtp]
\begin{center}
\includegraphics[width=.7\textwidth]{fig7_2}
\caption{Just noticeable difference curves in intensity 
for 70~Hz, 200~Hz, and 1000~Hz sinusoidal tones. 
(From ``Physics of Sound” by R.A.~Berg and D.G.~Stork.)}
\label{f:2}
\end{center}
\end{figure}

\subsection{Just Noticeable Difference in Frequency}

In addition to being able to discern changes in sound intensity, 
we have an even better ability to notice changes in frequency. 
Figure~\ref{f:jnd_pitch} shows the JND in frequency, comparing 
it to the size of the critical bands on the cochlea.
%
\begin{figure}[hbtp]
\begin{center}
\includegraphics[width=.4\textwidth]{freqJNDa.jpg}
\caption{Just noticeable difference in frequency,
comparing it to the size of the critical bands on the cochlea.
(From ``Science of Sound," by Rossing, Moore, and Wheeler.)}
\label{f:jnd_pitch}
\end{center}
\end{figure}
%

To experimentally determine the just noticeable difference in
frequency, we play two pure tones one right after the other,
starting with the same frequency. 
We then increase one frequency slightly and keep playing both tones 
in succession.
The JND in frequency is when you can first discern a difference in 
the frequency (i.e., pitch) of the two tones.
One can express the JND in frequency as the either the 
difference between the two frequencies or as a percentage relating
the frequency difference to the starting frequency.

\underline{Question 18} Play two pure tones sequentially. Start with the same frequency. Increase one frequency
slightly and keep playing both tones one after another. When do you hear the just noticeable
difference in frequency? Do this at frequencies of 200 Hz and 800 Hz.

\underline{Question 19} how do your measured JND at these frequencies compare to Figure~\ref{f:jnd_pitch}?


\end{document}


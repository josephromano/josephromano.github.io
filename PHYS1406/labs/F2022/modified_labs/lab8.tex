\documentclass[11pt]{NSF}
\sloppy

\usepackage{latexsym}
\usepackage{graphicx}
\usepackage{draftcopy}
\usepackage{longtable}
\usepackage{hyperref}

% some definitions
 
% begin equation, itemize, etc.

\def\be{\begin{equation}}
\def\ee{\end{equation}}
\def\bi{\begin{itemize}}
\def\ei{\end{itemize}}
\def\ben{\begin{enumerate}}
\def\een{\end{enumerate}}
\def\i{\item{}}

%%%%%%%%%%%%%%%%%%%%%%%%%%%%%%%%%%%%%%%%%%%%%%%%%%%%%%%%%%%%%%
\begin{document}
     
\section{8. Room Acoustics}

PURPOSE AND BACKGROUND

An enclosed space has characteristic resonance frequencies for
standing waves between walls and other surfaces. We already have
discussed the much simpler case of resonating air columns in pipes.
The acoustics of a room depends on the volume of the room, the surface
area of the walls, the sound absorption properties of the materials,
and persons and furniture in the room. When you sing in a shower, you
will notice that certain frequencies are enhanced. This is due to
resonances in the box-shaped volume of the shower. Resonances may also
play an undesirable role in concert halls. These resonances may cause
feedback noise if electronic sound amplification is used. In such
cases the resonances can be removed from the frequency spectrum with
electronic equalizers.

A study of room acoustics includes the frequency analysis and time
analysis of sound. The {\em reverberation time} is the time it takes 
for the sound intensity to decay by 60~dB or a factor of $10^6$.

Large concert halls and churches have reverberation times of up to a
few seconds. Our lecture room has a reverberation time of about 0.5~s. 
Sound travels with speed
$v = 346~{\rm m/s}$ at $25{}^\circ{\rm C}$. 
A large
concert hall has large distances for sound to travel and consequently
a long decay time. The materials from which sound is reflected also
affect the reverberation time. Sound absorbing materials such as
cloth, egg crating, and acoustical boards greatly absorb sound and have a
short reverberation time. Highly reflective materials such as concrete
walls and tile floors reflect sound with little absorption and result
in long reverberation times. The sound absorption of a material
depends on frequency and hence so does the reverberation time. One can
``tune” the reverberation time of a room for best acoustics by the
choice of materials and their placement.

\subsection{Room Resonances}

For the simplest case of a box-like room, with all surfaces constructed 
of the same material, the resonant frequencies are given by the formula
%
\be
f_{N_xN_yN_z} = \frac{v}{2}\sqrt{
\left(\frac{N_x}{L_x}\right)^2+
\left(\frac{N_y}{L_y}\right)^2+
\left(\frac{N_z}{L_z}\right)^2}
\label{e:1}
\ee
%
where $N_x$, $N_y$, $N_z$ are integer harmonic numbers; $L_x$, $L_y$,
$L_z$ are the dimensions of the room; and $v$ is the speed of sound in
air. For example, the lowest resonance frequency (fundamental) for the
$x$-direction is $f_{100} = v/2L_x$, with $N_x=1$ and $N_y=N_z=0$.
This is the same as the fundamental frequency of a vibrating
string, where the wavelength was twice the length of the string. Now
the wavelength is twice the $x$-dimension, $L_x$, of the box.  The $y$
and $z$-directions have their resonance frequencies as well,
calculated in the same way. Many higher resonance frequencies exist
for the standing waves in each direction, and also when waves get
reflected at an angle with respect to the walls of the box. These
frequencies are obtained from equation~(\ref{e:1}) with more than one of
the harmonic numbers $N_x$, $N_y$, $N_z$ equal to 1 or larger.

Our lecture room has a complicated geometry. It is not a ``box” 
and therefore it has a much more complex resonance spectrum than that from
equation~(\ref{e:1}). The room contains furniture, equipment, and people that
change the resonances. Nevertheless, we shall assume in a grand
simplification that the room is box-like and calculate the lowest
resonances.  

Measure the length x, width y, and average height z of the music laboratory. \\ \\
\[L_x = \rule{1.5cm}{0.15mm}m  \rule{1.5cm}{0.0mm}  L_y = \rule{1.5cm}{0.15mm}m \rule{1.5cm}{0.0mm}  L_z = \rule{1.5cm}{0.15mm}m\]

\underline{Question 1}  Calculate the three fundamental resonance frequencies of the room
from the formula
%
\be
f= \frac{v}{2L}
\ee
%
where $L$ is any of the lengths $L_x$, $L_y$, $L_z$ and $v = 346~{\rm
m/s}$ at an assumed room temperature of $25{}^\circ{\rm C}$.
(Don't forget to first convert the lengths from feet to meters using
$1~{\rm m}= 3.28~{\rm ft}$.)
\[f_{100} = \rule{1.5cm}{0.15mm}Hz  \rule{1.5cm}{0.0mm}  f_{010} = \rule{1.5cm}{0.15mm}Hz \rule{1.5cm}{0.0mm}  f_{001} = \rule{1.5cm}{0.15mm}Hz\]


\underline{Question 2}  Calculate the first overtones (2nd harmonics) of each of the three
fundamentals by doubling the frequencies from the preceding question.
\[f_{200} = \rule{1.5cm}{0.15mm}m  \rule{1.5cm}{0.0mm}  f_{020} = \rule{1.5cm}{0.15mm}m \rule{1.5cm}{0.0mm}  f_{002} = \rule{1.5cm}{0.15mm}m\]

\underline{Question 3}  Write down the range of these first six frequencies.

\subsection{Acoustics Box}

Instead of studying the lecture room in more detail, we use a cubical
box or ``model room” with identical dimensions 
$L_x =L_y = L_z \equiv L$.
See Figure~\ref{f:1} for a similar but non-cubical box. 
%
\begin{figure}[hbtp]
\begin{center}
\includegraphics[width=.7\textwidth]{fig8_1}
\caption{Acoustics box to
simulate the acoustics of a room (built by Arnold Fernandez).  The
small speaker at the top excites box resonances with a swept-sine 
excitation or white noise. 
The microphone inserted on the right records the resonances.}
\label{f:1}
\end{center}
\end{figure}
%
The frequency spectrum for this cubical ``room” is much simpler than
for a real room with different dimensions, surfaces, furniture, etc.
Equation~(\ref{e:1}) above simplifies to
%
\be
f_{N_1N_2N_3} = \frac{v}{2L}\sqrt{N_1^2 + N_2^2 + N_3^2}
\label{e:3}
\ee
%
The integers $N_1$, $N_2$, $N_3$ in the formula are the mode numbers. 
For example, the lowest mode with an air resonance in only the 
$x$-direction is $(N_1, N_2, N_3) = (1, 0, 0)$. 
The corresponding resonance frequency is
%
\be
f_{1,0,0} = \frac{v}{2L}
\ee
%
For a cubical box, we obviously have the same resonance frequency 
$f_{1,0,0}$ for the three modes $(N_1, N_2, N_3) = (1,0,0)$, $(0,1,0)$, $(0,0,1)$.

\underline{Question 4} Measure the length L of your box. Calculate the lowest resonance frequency corresponding to modes $(1,0,0)$, $(0,1,0)$, $(0,0,1)$.
\\
\\
\\
\\
\\

\underline{Question 5}  Calculate the next four lowest resonance frequencies above the first resonance frequency and assign the appropriate values of $N_1$, $N_2$, $N_3$.
\\
\\
\\
\\
\\

Acquire some resonance spectra with the “FFT Analyzer” in the FEaT software. Excite the resonances
with white noise or sine sweep and determine which yields the better spectra.

\underline{Question 6} Read the five lowest frequencies from the resonance spectra. How do these measured values compare to your calculated values.
\\
\\
\\
\\
\\

%
\begin{figure}[hbtp]
\begin{center}
\includegraphics[width=.9\textwidth]{fig8_2}
\caption{Resonances in a cubical plywood box of inside dimension 
$L = 362~{\rm mm}$. 
The lowest resonances are clearly seen and the vibrational mode
numbers can be identified.
The resonances were excited with broadband white noise.}
\label{f:2}
\end{center}
\end{figure}
%



\subsection{Calculation of the Reverberation Time}

The {\em reverberation time} is one of the most important characteristics of
a room. Just as in the case of the resonant frequencies, the
reverberation time depends on the geometry of the room, on the choice of
sound absorbing materials, and on the persons and furniture in the room.
The reverberation time can be estimated from the formula 
%
\be
T_{\rm R} = 0.050\,\frac{V}{A_{\rm sabin}}~{\rm s}, \rule{2.5cm}{0.0mm} where \rule{0.2cm}{0.0mm} A_{\rm sabin} = aA
\label{e:TR}
\ee
%
where $V$ is the volume of the room in ${\rm ft}^3$ and $A_{\rm
sabin}$ is the total {\em effective area} of the room, also called
``absorption”, in units of {\em sabin},  a the absorption coefficient of the wall material, and A the physical area square feet. 

The unit {\em sabin} is named after Wallace C.~Sabine, founder of
architectural acoustics. One sabin is equal to a square foot of
perfectly absorbing material. For instance, a $3~{\rm ft}^2$ 
hole in a wall is a perfect sound absorber as it reflects no sound 
and thus corresponds to an effective area 
$A_{\rm eff} = 3~{\rm sabin}$. 
Table~\ref{t:1} lists the absorption
coefficients of several common materials. For example, a piece of
carpet with an area of $3~{\rm ft}^2$ at a sound frequency of 
500~Hz has an absorption coefficient $a = 0.3$ and an effective 
area $A_{\rm eff} = a A = 0.3\times 3= 0.90~{\rm sabin}$.

%
\begin{table}[hbtp]
\begin{center}
\includegraphics[width=.8\textwidth]{tab8_1}
\caption{Absorption coefficients a of various materials. 
(Values from ``The Physics of Sound" by R.E.~Berg and D.G.~Stork.)}
\label{t:1}
\end{center}
\end{table}
%
The effective area of an adult person is $A_{\rm eff} = 4.2~{\rm sabin}$.

In order to find the total effective area entering in
equation~(\ref{e:3}), each surface area $A$ is multiplied by 
its absorption coefficient $a$, resulting in the
product $a\, A$ for each surface. 
The total effective area then is the sum over all areas
%
\be
A_{\rm eff} = a_1\,A_1 + a_2\,A_2 + a_3\,A_3 + \cdots
~(\textrm{in sabin})
\label{e:Aeff}
\ee

\underline{Question 7} Calculate the reverberation time of our lecture room, which has
approximate
length, width, and height $L_x=29~{\rm ft}$, $L_y=24~{\rm ft}$, and 
$L_z=9.5~{\rm ft}$, respectively.
Use the absorption coefficients from Table~\ref{t:1} at 500~Hz
for concrete/brick for the walls and floors, and acoustical board
for the ceiling.
Calculate the effective area $A_{\rm eff}$ from
equation~(\ref{e:Aeff}) and the volume from $V = L_xL_yL_z$. 
Finally, use your values for $A_{\rm eff}$ and $V$ in
equation~(\ref{e:TR}) to
obtain the reverberation time.
\\
\\
\\
\\
\[A_{\rm sabin} = \rule{2.5cm}{0.15mm}sabin, \rule{2.5cm}{0.0mm} T_{\rm reverb} = \rule{2.5cm}{0.15mm}s\]

\underline{Question 8} Compare your values with those in the Course Guide (see chapter 
on ``Room Acoustics”).

\[A_{\rm sabin} = \rule{2.5cm}{0.15mm}sabin, \rule{2.5cm}{0.0mm} T_{\rm reverb} = \rule{2.5cm}{0.15mm}s\]

\subsection{Reverberation Time of The Laboratory Room}

Use the signal generator tool and select the Spectrogram Mode in the software. Set the signal
generator to 500 Hz or use “white noise”. Record a colored spectrogram, also called a sonogram,
where the vertical axis on the display is the frequency, the horizontal axis is time, and the color
indicates sound intensity. Run the spectrogram for a few seconds, then turn off the signal
generator to stop recording. Observe the time within which the 500 Hz line or white noise fade
out. This happens within about one second. This gives a good indication of the reverberation
time. You will have to guess on the spectrogram display when the reverberating sound has faded
into the background. You may also do this experiment by clapping instead of using a frequency
generator or white noise

\underline{Question 9} Write down the reverberation time for our laboratory room as obtained from the sonogram:
\[T_{\rm reverb} = \rule{2.5cm}{0.15mm}s\]

\underline{Question 10} Compare this reverberation time with the value you calculated for our classroom (see
Questions 7 and 8). Does the classroom or the laboratory room has a shorter reverberation time?
Give reasons for the difference.



\subsection{Calculation of the Reverberation Time of the
Cubical Acoustics Box}

Consider the much simpler case of the cubical acoustics box. 
Take its length L to be equal to the box used earlier in the lab.


\underline{Question 11} Calculate the reverberation time for the box 
assuming the absorption coefficient for plywood at 500~Hz 
in Table~\ref{t:1}.
\\
\\
\\
\\
\\

\underline{Question 12} What are the least and most sound absorbing materials in 
Table~\ref{t:1} at 500~Hz?
Calculate the reverberation time of the cubical box for 
those two materials instead of plywood.
\\
\\
\\
\\
\\

\underline{Question 13} Based on your results for the previous question, describe how you
can adjust the reverberation time of a room with the proper choice
of materials.
\\
\\
\\
\\
\\



\end{document}

% extras
%
\begin{figure}[hbtp]
\begin{center}
\includegraphics[width=.7\textwidth]{fig8_3}
\caption{Resonances of the cubical box with white noise excitation on
an extended frequency scale including the higher modes. The modes
become denser with increasing frequency.}
\label{f:3}
\end{center}
\end{figure}
%
%
\begin{figure}[hbtp]
\begin{center}
\includegraphics[width=.7\textwidth]{fig8_4}
\caption{Entry Hall of the TTU Southwest Collection/Special 
Collections Library.}
\label{f:4}
\end{center}
\end{figure}
%
%
\begin{figure}[hbtp]
\begin{center}
\includegraphics[width=.7\textwidth]{fig8_5}
\caption{Campus Circle at Texas Tech University.}
\label{f:5}
\end{center}
\end{figure}
%


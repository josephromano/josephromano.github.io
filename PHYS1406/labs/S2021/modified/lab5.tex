\documentclass[11pt]{NSF}
\sloppy

\usepackage{latexsym}
\usepackage{graphicx}
\usepackage{draftcopy}
\usepackage{longtable}
\usepackage{hyperref}
\usepackage{amsmath}

% some definitions
 
% begin equation, itemize, etc.

\def\be{\begin{equation}}
\def\ee{\end{equation}}
\def\bi{\begin{itemize}}
\def\ei{\end{itemize}}
\def\ben{\begin{enumerate}}
\def\een{\end{enumerate}}
\def\i{\item{}}

%%%%%%%%%%%%%%%%%%%%%%%%%%%%%%%%%%%%%%%%%%%%%%%%%%%%%%%%%%%%%%
\begin{document}
     
\section{5. Fourier Analysis and Synthesis of Waveforms}

PURPOSE AND BACKGROUND

The simplest sound is a pure sine wave with a single frequency and amplitude.
Most sound sources and instruments do not produce such simple waves. Usually
their sound contains many sine waves with higher frequencies, called {\em harmonics.}
These act together according to the {\em superposition principle} to produce a
complex tone. This addition of sine waves with suitable amplitudes and phases
is called {\em Fourier synthesis of sound}. 
The opposite, the decomposition of sound
into its sine-wave components, is called {\em Fourier analysis}. 
Periodic sound can
be synthesized or analyzed with a sufficient number of sine waves. 
A {\em pure} tone is a sine wave with a single frequency. 
Many sine waves added together form a
complex tone and waveform periodic in time. This laboratory is about the
analysis and synthesis of sound and how electronic 
{\em synthesizers} can mimic real instruments.

\subsection{Fourier Synthesis of Waveforms}

For our experiment, we will use the 
Fourier Series Applet, which is available online at 
\url{https://www.falstad.com/fourier/}.
Listen to several available wave forms (e.g.,  sine, triangle, 
square, and sawtooth waves) at a fundamental frequency of 
$f_1 = 500~{\rm Hz}$.
To do so, you will need to adjust the ``Playing Frequency" 
slider as close as you can to 500~Hz,  
and the check the ``Sound" box to listen to the sounds.

\subsubsection*{Questions}
\ben
\i Draw sketches of the four waveforms.

\i Which waveform most resembles a pure sine wave?

\i Which waveform least resembles a pure sine wave?

\i Which tone sounds least like the pure sine wave?
\een

Complex waveforms are produced by adding sine waves of different frequencies
and amplitudes. The tone heard in all four cases has the same {\em pitch} or
fundamental frequency $f_1= 500~{\rm Hz}$. For a pure tone (sine wave), the fundamental
is the only frequency present. For complex tones, sine waves with integer
multiples of the fundamental frequency and suitable amplitudes are added
together. For example, the next integer multiples of the fundamental 
$f_1 = 500~{\rm Hz}$ are 
$f_2 = 2 f_1= 1000~{\rm Hz}$, 
$f_3 = 3 f_1 = 1500~{\rm Hz}$, and so on.

These higher frequencies are called {\em overtones} or {\em harmonics}. 
Just like the
fundamental, each overtone has a single frequency. A complex waveform can be
produced with the fundamental plus higher harmonics of suitable amplitudes.
This process is called {\em superposition} of waves or, mathematically speaking,
{\em Fourier synthesis} of waves. Conversely, you can take a complex waveform apart
by decomposing it with a spectrum analyzer into its individual harmonics. This is
called {\em Fourier analysis} of waves.

%%%%%%%%%%%%%%%%%%%%%%%%%%%%%%%%%%%%%%%%
\subsubsection{Sawtooth Waveform}

The harmonics of the sawtooth wave follow a simple pattern. All harmonics
exist from $N=1$ to $N=\infty$, with amplitudes given by 
$A_N = A_1/N$.
Thus all integer multiples of the fundamental frequency
contribute to the waveform. 
Since in practice we cannot add an infinite number
of harmonics, we shall only use the first five or six harmonics
and add them up.

Using the online app, start by clicking the ``Sine" box.
You should see a single white dot, sticking up above the rest,
at a height corresponding to the amplitude of the first harmonic.
The second harmonic $N=2$, $f_2=1000~{\rm Hz}$ should have an 
amplitude $A_2=A_1/2$ for a sawtooth wave. 
Add this harmonic to the fundamental by adjusting the height 
of the second white dot to half the height of the first white dot. 
Take a look at and listen to the waveform generated.

\subsubsection*{Questions}
\ben
\i Find the frequencies of the next three higher harmonics 
and their relative amplitudes in percent. 
Complete the entries in Table 1.
%
\begin{table}[hbtp]
\begin{center}
\includegraphics[width=.35\textwidth]{tab5_1}
\caption{Sawtooth waveform: Harmonic numbers, frequencies, and relative
amplitudes.}
\label{t:1}
\end{center}
\end{table}
%

\i What would be the frequency and amplitude of the $N = 10$ 
harmonic for a sawtooth waveform of fundamental frequency 
$f_1 = 500~{\rm Hz}$.
\een

Continue adding harmonics (3rd, 4th, 5th, etc.) by adjusting the
heights of their white dots appropriately.
Note the changes in the tone and the waveform.
With each addition of a harmonic, the wave should look more and 
more like a sawtooth.

%%%%%%%%%%%%%%%%%%%%%%%%%%%%%%%%%%%%%%%%
\subsubsection{Square Wave}

A square or rectangular waveform is similar to the sawtooth 
in that the amplitudes of the harmonics follow the 
$A_N=A_1/N$ dependence. 
However, the major difference is that only the {\em odd}
harmonics $N=1$, $N=3$, $N=5$, etc., contribute.

\subsubsection*{Questions}
\ben

\i Use this information and complete the entries in Table~2
for the square wave.
%
\begin{table}[hbtp]
\begin{center}
\includegraphics[width=.35\textwidth]{tab5_2}
\caption{Square wave: Harmonic numbers, frequencies, and relative amplitudes.}
\label{t:2}
\end{center}
\end{table}
%
\een

Synthesize a square wave using the online app by starting as
before with just a ``Sine", and then successively adding the
higher harmonics with amplitudes given in Table~2.
Note the changes in tone and shape of the waveform as more
harmonics are added.

%%%%%%%%%%%%%%%%%%%%%%%%%%%%%%%%%%%%%%%%
\subsubsection{Triangular Waveform}

\subsubsection*{Questions}

\ben
\i 
%
\begin{table}[hbtp]
\begin{center}
\includegraphics[width=.35\textwidth]{tab5_3}
\caption{Triangular waveform: Harmonic numbers, frequencies, and relative amplitudes.}
\label{t:3}
\end{center}
\end{table}
%
\een

%%%%%%%%%%%%%%%%%%%%%%%%%%%%%%%%%%%%%%%%
\subsection{Fourier Analysis of Waveforms}


%
\begin{figure}[hbtp]
\begin{center}
\includegraphics[width=.7\textwidth]{fig5_1}
\caption{}
\label{f:} 
\end{center} 
\end{figure}
%
\end{document}


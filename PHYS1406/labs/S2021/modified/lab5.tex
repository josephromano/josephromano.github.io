\documentclass[11pt]{NSF}
\sloppy

\usepackage{latexsym}
\usepackage{graphicx}
\usepackage{draftcopy}
\usepackage{longtable}
\usepackage{hyperref}
\usepackage{amsmath}

% some definitions
 
% begin equation, itemize, etc.

\def\be{\begin{equation}}
\def\ee{\end{equation}}
\def\bi{\begin{itemize}}
\def\ei{\end{itemize}}
\def\ben{\begin{enumerate}}
\def\een{\end{enumerate}}
\def\i{\item{}}

%%%%%%%%%%%%%%%%%%%%%%%%%%%%%%%%%%%%%%%%%%%%%%%%%%%%%%%%%%%%%%
\begin{document}
     
\section{5. Fourier Analysis and Synthesis of Waveforms}

PURPOSE AND BACKGROUND

The simplest sound is a pure sine wave with a single frequency and amplitude.
Most sound sources and instruments do not produce such simple waves. Usually
their sound contains many sine waves with higher frequencies, called {\em harmonics.}
These act together according to the {\em superposition principle} to produce a
complex tone. This addition of sine waves with suitable amplitudes and phases
is called {\em Fourier synthesis of sound}. 
The opposite, the decomposition of sound
into its sine-wave components, is called {\em Fourier analysis}. 
Periodic sound can
be synthesized or analyzed with a sufficient number of sine waves. 
A {\em pure} tone is a sine wave with a single frequency. 
Many sine waves added together form a
complex tone and waveform periodic in time. This laboratory is about the
analysis and synthesis of sound and how electronic 
{\em synthesizers} can mimic real instruments.

\subsection{Fourier Synthesis of Waveforms}

For our experiment, we will use the 
Fourier Series Applet, which is available online at 
\url{https://www.falstad.com/fourier/}.
Listen to several available waveforms (e.g.,  sine, triangular, 
square, and sawtooth waves) at a fundamental frequency of 
$f_1 = 500~{\rm Hz}$.
To do so, you will need to adjust the ``Playing Frequency" 
slider as close as you can to 500~Hz,  
and then check the ``Sound" box to listen to the sounds.

\subsubsection*{Questions}
\ben
\i Draw sketches of the four waveforms.

\i Which waveform most resembles a pure sine wave?

\i Which waveform least resembles a pure sine wave?

\i Which tone sounds least like the pure sine wave?
\een

Complex waveforms are produced by adding sine waves of different frequencies
and amplitudes. The tone heard in all four cases has the same {\em pitch} or
fundamental frequency $f_1= 500~{\rm Hz}$. For a pure tone (sine wave), the fundamental
is the only frequency present. For complex tones, sine waves with integer
multiples of the fundamental frequency and suitable amplitudes are added
together. For example, the next integer multiples of the fundamental 
$f_1 = 500~{\rm Hz}$ are 
$f_2 = 2 f_1= 1000~{\rm Hz}$, 
$f_3 = 3 f_1 = 1500~{\rm Hz}$, and so on.

These higher frequencies are called {\em overtones} or {\em harmonics}. 
Just like the
fundamental, each overtone has a single frequency. A complex waveform can be
produced with the fundamental plus higher harmonics of suitable amplitudes.
This process is called {\em superposition} of waves or, mathematically speaking,
{\em Fourier synthesis} of waves. Conversely, you can take a complex waveform apart
by decomposing it with a spectrum analyzer into its individual harmonics. This is
called {\em Fourier analysis} of waves.

%%%%%%%%%%%%%%%%%%%%%%%%%%%%%%%%%%%%%%%%
\subsubsection{Sawtooth Waveform}

The harmonics of the sawtooth wave follow a simple pattern. All harmonics
exist from $N=1$ to $N=\infty$, with amplitudes given by 
$A_N = A_1/N$, where $A_1$ is the amplitude of the fundamental frequency.
Thus all integer multiples of the fundamental frequency
contribute to the waveform. 
Since in practice we cannot add an infinite number
of harmonics, we shall use only the first five non-zero
harmonics and add them up.

Using the online app, start by clicking the ``Sine" box.
You should see a single white dot, sticking up above the rest,
at a height corresponding to the amplitude of the first harmonic.
The second harmonic $N=2$, $f_2=1000~{\rm Hz}$ should have an 
amplitude $A_2=A_1/2$ for a sawtooth wave. 
Add this harmonic to the fundamental by adjusting the height 
of the second white dot to half the height of the first white dot. 
Take a look at and listen to the waveform generated.

Find the frequencies of the next three higher harmonics 
and their relative amplitudes in percent. 
Complete the entries in Table 1.
%
\begin{table}[hbtp]
\begin{center}
Table 1: Sawtooth waveform\\
\includegraphics[width=.35\textwidth]{tab5_1}
%\caption{Sawtooth waveform: Harmonic numbers, frequencies, and relative amplitudes.}
\label{t:1}
\end{center}
\end{table}
%
%\i What would be the frequency and amplitude of the $N = 10$ 
%harmonic for a sawtooth waveform of fundamental frequency 
%$f_1 = 500~{\rm Hz}$.

Continue adding harmonics (3rd, 4th, 5th, etc.) by appropriately
adjusting the heights of the white dots.
Note the changes in the tone and the waveform.
With each addition of a harmonic, the waveform should look more and 
more like a sawtooth.

%%%%%%%%%%%%%%%%%%%%%%%%%%%%%%%%%%%%%%%%
\subsubsection{Square Wave}

The square or rectangular waveform is similar to the sawtooth 
in that the amplitudes of the harmonics follow the 
$A_N=A_1/N$ dependence. 
However, the major difference is that only the {\em odd}
harmonics $N=1$, $N=3$, $N=5$, etc., contribute.

Use this information and complete the entries in Table~2
for the square wave.
%
\begin{table}[hbtp]
\begin{center}
Table 2: Square wave\\
\includegraphics[width=.35\textwidth]{tab5_2}
%\caption{Square wave: Harmonic numbers, frequencies, and relative amplitudes.}
\label{t:2}
\end{center}
\end{table}

Synthesize a square wave using the online app by starting as
before with just a ``Sine", and then successively adding the
higher harmonics with the amplitudes given in Table~2.
Note the changes in tone and shape of the waveform as more
harmonics are added.

%%%%%%%%%%%%%%%%%%%%%%%%%%%%%%%%%%%%%%%%
\subsubsection{Triangular Waveform}

The triangular wave is similar to the square wave in that 
it too consists of odd harmonics only. However, 
the amplitudes no longer follow the $A_N=A_1/N$ dependence, 
%but rather a $A_N = (-1)^{(N-1)/2} A_1/N^2$ dependence. 
but rather a $A_N = \pm A_1/N^2$ dependence. 
The sign of the amplitude alternates +, $-$, +, $-$, etc.,
for the 1st, 3rd, 5th, 7th harmonics, etc.
For instance, given an amplitude of the first harmonic of 100\%, 
the amplitude of the third harmonic now is
$A_3 = -A_1/3^2 = -11.11\%$.

Complete the entries in Table 3 for the triangular waveform.
%
\begin{table}[hbtp]
\begin{center}
Table 3: Triangular waveform\\
\includegraphics[width=.35\textwidth]{tab5_3}
%\caption{Triangular waveform: Harmonic numbers, frequencies, and relative amplitudes.}
\label{t:3}
\end{center}
\end{table}
%

Use the completed Table 3 to synthesize a triangular 
waveform using the online app and listen to the result.

\subsubsection*{Questions}
\ben
\i Of the three waveforms, which had the least noticeable 
contributions from its overtones to the overall form and tone?

\i Which of the three waveforms had the most 
noticeable contributions from its overtones?

\i How could you get sharper ``edges” on the square and 
sawtooth waveform than those created with just five non-zero
harmonics?

\i Why can you hear a 1~Hz square wave?
\een

%%%%%%%%%%%%%%%%%%%%%%%%%%%%%%%%%%%%%%%%
\subsection{Fourier Analysis of Waveforms}

The ``FFT Analyzer Tool" in the Electroacoustic Toolbox 
analyzes an incoming signal with a mathematical operation 
called a {\em Fast Fourier Transform} (FFT) to 
identify the different frequencies in the signal. 
The display is a frequency spectrum---see Figure 1.
For a sine wave, the FFT tool will show a frequency spectrum with one peak for
the only frequency present (the fundamental), 
with the amplitude being the height of the peak.
For non-sinusoidal periodic waveforms you will see many peaks.
The location of the peaks and their relative amplitudes 
follow the theoretical expressions that we used above to synthesize
sawtooth, square, and triangular waveforms.

\subsubsection{Musical Synthesizers}

Modern keyboards are capable of simulating sounds from real 
instruments quite well. 
They work on the basis of {\em Fourier analysis and synthesis}. 
Every tone from a given instrument has its own {\em timbre} 
and Fourier spectrum. 
The fundamental frequency determines the {\em pitch} of the tone. 
Often the fundamental does not have the highest amplitude. 
Some higher harmonics may be stronger. 
Nonetheless, the ear discerns the frequency of the fundamental 
as the pitch of the tone. 
Musical instruments produce sound with complex Fourier spectra. 
These change with every note. 
For example, the Fourier spectra of ``middle C” ($f = 261.63~{\rm Hz}$) 
from a violin and a viola or bassoon look quite different.

\subsubsection{Real and Synthesized Sound of a Didgeridoo}

An example of the spectrum from a didgeridoo and the 
corresponding synthesized tone is shown in Figure~\ref{f:1}. 
The synthesized tone sounds similar to the actual one, but not quite the same.
%
\begin{figure}[hbtp]
\begin{center}
\includegraphics[width=.7\textwidth]{fig5_1}
\caption{Top: Actual sound spectrum of the note D2 from a didgeridoo. 
The odd harmonics dominate, as expected for a ``closed tube”. 
Bottom: Synthesized sound spectrum, using only the first four odd harmonics 
$N = 1$, 3, 5, 7.}
\label{f:1} 
\end{center} 
\end{figure}
%

\subsubsection*{Questions}
\ben
\i Why does the synthesized tone not sound
exactly like the real tone from a didgeridoo?
\een

\end{document}


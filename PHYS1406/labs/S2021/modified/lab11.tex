\documentclass[11pt]{NSF}
\sloppy

\usepackage{latexsym}
\usepackage{graphicx}
\usepackage{draftcopy}
\usepackage{longtable}
\usepackage{hyperref}

% some definitions
 
% begin equation, itemize, etc.

\def\be{\begin{equation}}
\def\ee{\end{equation}}
\def\bi{\begin{itemize}}
\def\ei{\end{itemize}}
\def\ben{\begin{enumerate}}
\def\een{\end{enumerate}}
\def\i{\item{}}

%%%%%%%%%%%%%%%%%%%%%%%%%%%%%%%%%%%%%%%%%%%%%%%%%%%%%%%%%%%%%%
\begin{document}
     
\section{11. Musical Scales, Temperament, Elementary Music Theory}

PURPOSE AND BACKGROUND

The arrangement of musical notes today is the result of centuries of
changes in both musical style and taste. As music became more complex,
the tuning of instruments, for instance the piano, took various forms.
The older styles of tuning are primarily of historical interest, but
it is still worthwhile to understand how music has arrived at the
modern {\em equal temperament}. 
In tuning any instrument, a starting pitch
must be chosen. For a violin, this pitch is concert A$_4$ 
($f=440~{\rm Hz}$).
The rest of the strings are tuned to this pitch. The manner in which
they are tuned is called the {\em musical temperament}. 
A violin has only
four strings and therefore tuning is not difficult. But how is a piano
tuned with so many keys? What should be the starting pitch, and how do
you tune the other keys? These questions arose from the need for
standardization. Here we will discuss briefly some features of
older temperaments, their benefits and drawbacks. 
Equal temperament will be treated in more detail.

\subsection{Pythagorean Temperament}

When music started using multiple parts, chords became integral
objects of melodic structure. Initially, {\em perfect fifths} were used to
tune all notes on a keyboard. A perfect fifth is rather easy to
discern, even by the musically challenged. The term {\em fifth} refers to
the fifth note in a major scale. For example, in the C-major scale C,
D, E, F, G, A, B, the note “G” is the fifth. The frequency ratio
between the perfect fifth (G) and the {\em tonic} (C) is 3:2.

The perfect fifth was the first ratio to be used in tuning. 
Early piano tuners
would first tune middle C$_4$, then the notes G$_4$, D$_5$, A$_5$,
E$_6$, B$_6$, F$^\sharp_7$ all in perfect fifths above C$_4$. 
They also tuned F$_3$, B$^\flat_2$, E$^\flat_2$, A$^\flat_1$, D$^\flat_1$ in
perfect fifths below C$_4$. 
F$^\sharp_6$ is then tuned by a perfect octave down
from F$^\sharp_7$, and C$^\sharp_7$ is tuned a perfect fifth above
F$^\sharp_6$. In this way, the
twelve different keys in the {\em chromatic scale} were determined. 
All other keys were tuned from these keys by octaves with frequencies in
the ratio 2:1. The resulting temperament was known as the {\em Pythagorean
temperament}.

\subsubsection*{Questions}
\ben
\item
{\em Middle C} is commonly taken as ${\rm C}_4 = 261.63~{\rm Hz}$. 
What are the frequencies of a perfect fifth above middle C (G$_4$) 
and a perfect fifth below middle C (F$_3$)?

\item
Consider the notes D$^\flat_1$ and C$^\sharp_8$ tuned by perfect 
fifths according to the Pythagorean temperament. 
Since all 12 keys in the chromatic scale are
determined in this manner, there are 12 jumps of 3/2 in frequency, or
twelve perfect fifths, from D$^\flat_1$ to C$^\sharp_8$. 
Calculate the frequency ratio C$^\sharp_8$/D$^\flat_1$.

\item 
D$^\flat$ and C$^\sharp$ are the same key on the piano. 
We thus would expect D$^\flat_8$ and
C$^\sharp_8$ to have the same pitch. 
(Note that C$^\sharp_8$ is one half step above the
highest key on the piano, but answer this question anyway.) 
Going from D$^\flat_1$ to D$^\flat_8$ in octaves spans 7 octaves. 
Each octave increases the frequency by a factor of 2. 
Calculate the frequency ratio D$^\flat_8$/D$^\flat_1$.

\item
Questions 2 and 3 arrive at the same notes by two different means, one
by 12 perfect fifths, the other by 7 octaves. 
How much do the frequency ratios in questions 2 and 3 differ?  
\een

The discrepancy is the {\em Pythagorean comma}. 
Octaves are retained as being perfect, with a ratio of 2:1. 
But the fifths must be re-tuned for 12 fifths to equal 7 octaves. 
In the Pythagorean temperament, all fifths are kept perfect but the last one. 
The final fifth sounds very bad and is known as the “wolf fifth” 
because of its growling sound and beats. 
This became unacceptable as music evolved.

\subsection{Just Temperament and Mean Tone Temperament}

When music began to have more differentiated harmonies, the inclusion
of a {\em perfect third} with a frequency ratio 5:4 became increasingly
important. The Pythagorean temperament had particularly bad thirds and
so its use was slowly discarded and replaced by other temperaments.

The {\em just temperament} tuned a perfect fifth above and below middle C,
resulting in G$_4$ and F$_3$, respectively. 
It also tuned perfect thirds above F$_3$ and above middle C. 
In this way, {\em major triads}, composed of the tonic, third, and 
fifth notes on a scale, sounded perfectly in tune. 
Tuning the other eight keys in just temperament is slightly more
complicated than in the Pythagorean temperament.

\subsubsection*{Questions}
\ben
\item
Calculate the frequencies of a perfect third above middle C (E$_4$) 
and a perfect third below middle C (A$^\flat_3$).
\een

As music expanded into {\em minor keys}, one of the serious weaknesses of
just temperament became painfully apparent. Minor triads sound
especially bad in this temperament. The error again is all
concentrated in one area. In this case, above E$^\flat$, 
the pair F$^\sharp$ and D$^\flat$
should be a perfect fifth, but each note is reached by two different
routes. Therefore this fifth is off by a significant amount. The 
{\em mean tone temperament} was created to compensate for the error by
spreading it over all the fifths, not just one. This too, however, caused
problems when musical keys farther from the key of C where used.

\subsection{Equal Temperament}

All these problems eventually led to a resolution with the
introduction of {\em equal temperament} or the {\em equal-tempered
scale}. 
Here the error is spread {\em equally across all twelve notes} 
in the chromatic sequence. 
All keys then sound the same. However, there are no perfect 
fifths or thirds or any other chords, except the octave. The twelve
keys in equal temperament are spaced by equal frequency ratios. Since
an octave must have a 2:1 ratio, the interval between keys must be
multiplied 12 times in order to give a value of two. This interval
therefore is the 12th root of two, namely $2^{1/12}$ . 
The frequency of
each note is multiplied by this number to give the next note one half
step or semitone higher.
%
\begin{figure}[hbtp]
\begin{center}
\includegraphics[width=.95\textwidth]{fig11_1}
\caption{Frequencies of the equal temperament scale.
(From ``Physics of Sound," by R.E.~Berg and D.G.~Stork.)}
\label{f:1}
\end{center}
\end{figure}
%

\subsubsection*{Questions}
\ben
\item
Use a calculator to compute the value of $2^{1/12}$. 
Calculate the frequencies of the remaining 11 notes of the chromatic
scale, starting with middle C. 
Insert the values in Table~\ref{t:1} below.
%
\begin{table}[hbtp]
\begin{center}
\includegraphics[width=.45\textwidth]{tab11_1}
\caption{Frequencies of the notes of the chromatic scale in equal 
temperament, starting with C$_4$.}
\label{t:1}
\end{center}
\end{table}
%

\item
Compare your calculated values in Table~\ref{t:1} with the frequencies shown
in Figure~\ref{f:1} for the piano keyboard.
Equal temperament is the way all pianos are tuned today, including
modern electronic keyboards.

\item 
Compare the frequencies of the C-major triad in just temperament and
equal temperament. For just temperament, take the answers for the
perfect fifth G$_4$ from Part I, Question 1 and perfect third E4 
from Part II, Question 1. For equal temperament, use values from Table~\ref{t:1}. 
Insert all values in Table~\ref{t:3} below.
%
\begin{table}[hbtp]
\begin{center}
\includegraphics[width=.45\textwidth]{tab11_3}
\caption{Frequencies of the major triad based on C$_4$, 
in just and equal temperament.}
\label{t:3}
\end{center}
\end{table}
%

\item
Which triad would you think sounds “better”, the one tuned in just
temperament or the one tuned in equal temperament? 
Why?

\een

\subsection{Tuning a Violin and Beats}

String instruments in chamber music often are tuned to Pythagorean
temperament. The A string of a violin is first tuned {\em beatless} to
concert ${\rm A}_4= 440~{\rm Hz}$ by comparing, for instance, with a tuning fork. 
The term beatless refers to the absence of ``beats” that would otherwise be
heard if a note were slightly out of tune with another. 
(Recall that if two notes with frequencies $f_1\approx f_2$ are played
simultaneously, you can hear two things: A tone with a
frequency equal to the average of $f_1$ and $f_2$, and a slow
amplitude variation ``beating” with the difference frequency 
$\Delta f=|f_2 -f_1|$.
This difference frequency is the {\em beat frequency}.)

Once the A$_4$ string on the violin is tuned to 440~Hz, 
the E$_5$ string, a fifth higher than A$_4$, is tuned by beats
as follows: 
The 3rd harmonic of A$_4$, i.e., ${\rm E}_6 = 1320~{\rm Hz}$, 
is tuned beatless with the 2nd harmonic of the E$_5$ string,
which again is ${\rm E}_6 = 1320~{\rm Hz}$. 
In the same manner, the D$_4$ string, which is a fifth down 
from the A$_4$ string, is tuned beatless with the A$_4$ string. 
Finally, the G$_3$ string, which is a fifth down from the D$_4$
string, is tuned beatless with the D$_4$ string. 
An experienced string-instrument player can easily hear the beats 
between two strings that
are out of tune and thus tune them to be beatless. All strings are
tuned by {\em perfect fifths} in this way according to the Pythagorean
temperament. The resulting sound from a string ensemble can be very
clean and pleasing. But slight dissonances may arise when string
instruments tuned to Pythagorean temperament and an equal-tempered
piano play together.

\subsubsection*{Questions}
\ben
\item
What are the frequencies of the four strings on the violin tuned to
Pythagorean temperament, starting with ${\rm A}_4 = 440~{\rm Hz}$? 
Put your entries in the first column of Table~\ref{t:4} below.
For the second column, use the relevant frequency values from Figure~\ref{f:1}.

\item
Start with middle ${\rm C}_4 = 261.63~{\rm Hz}$ and calculate 
the frequency of E$_4$ in Pythagorean temperament. 
(Hint: To get E$_4$ from C$_4$ in Pythagorean temperament,
you need to first go up 4 fifths to E$_6$, and then come 
back down 2 octaves.)
Compare this frequency with the value of E$_4$ in equal 
temperament (you can get this value from  Figure~\ref{f:1}).
If both of these notes were played simultaneously, would you 
expect to hear beats?
If so, what would you hear for the beat frequency?

\een
%
\begin{table}[hbtp]
\begin{center}
\includegraphics[width=.35\textwidth]{tab11_4}
\caption{Frequencies of the four open violin strings in Pythagorean
temperament and comparison with equal temperament.}
\label{t:4}
\end{center}
\end{table}
%


\end{document}
%%%%%%%%%%%%%%%%%%%%%%%%
% extras
%
\begin{table}[hbtp]
\begin{center}
\includegraphics[width=.4\textwidth]{tab11_2}
\caption{Three notes on the keyboard.}
\label{t:2}
\end{center}
\end{table}
%


\documentclass[11pt]{NSF}
\sloppy

\usepackage{latexsym}
\usepackage{graphicx}
\usepackage{draftcopy}
\usepackage{longtable}
\usepackage{hyperref}

% some definitions
 
% begin equation, itemize, etc.

\def\be{\begin{equation}}
\def\ee{\end{equation}}
\def\bi{\begin{itemize}}
\def\ei{\end{itemize}}
\def\ben{\begin{enumerate}}
\def\een{\end{enumerate}}
\def\i{\item{}}

%%%%%%%%%%%%%%%%%%%%%%%%%%%%%%%%%%%%%%%%%%%%%%%%%%%%%%%%%%%%%%
\begin{document}
     
\section{1. Topic}

PURPOSE AND BACKGROUND

text

\subsection{XXX}

text

\ben
\item
question 1?

\item
question 2?
\een

text

\subsection{YYY}

\ben
\item
question 1?

\een

%
\begin{figure}[hbtp]
\begin{center}
\includegraphics[width=.95\textwidth]{fig11_1}
\caption{Frequencies of the equal temperament scale.
(From ``Physics of Sound," by R.E.~Berg and D.G.~Stork.)}
\label{f:1}
\end{center}
\end{figure}
%
%
\begin{table}[hbtp]
\begin{center}
\includegraphics[width=.45\textwidth]{tab11_1}
\caption{Frequencies of the notes of the chromatic scale in equal 
temperament, starting with C$_4$.}
\label{t:1}
\end{center}
\end{table}
%

\end{document}


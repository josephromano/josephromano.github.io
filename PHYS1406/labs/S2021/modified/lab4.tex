\documentclass[11pt]{NSF}
\sloppy

\usepackage{latexsym}
\usepackage{graphicx}
\usepackage{draftcopy}
\usepackage{longtable}
\usepackage{hyperref}
\usepackage{amsmath}

% some definitions
 
% begin equation, itemize, etc.

\def\be{\begin{equation}}
\def\ee{\end{equation}}
\def\bi{\begin{itemize}}
\def\ei{\end{itemize}}
\def\ben{\begin{enumerate}}
\def\een{\end{enumerate}}
\def\i{\item{}}

%%%%%%%%%%%%%%%%%%%%%%%%%%%%%%%%%%%%%%%%%%%%%%%%%%%%%%%%%%%%%%
\begin{document}
     
\section{4. Air Resonance}

PURPOSE AND BACKGROUND

The concept of {\em resonance} in a pipe is similar to that of a string. The waves in
pipes consist of compressions and rarefactions of the air, with back-and-forth
motion of the air molecules in the direction of propagation or against it. The
waves in air thus are {\em longitudinal} waves. In this laboratory we study standing
waves in a pipe. They are the result of two waves traveling in opposite
directions inside the pipe, with each wave being reflected at the ends of the
pipe. In this way the superposition of two waves yields a standing wave,
provided that in addition the {\em resonance conditions} are met.

For a pipe with both ends open, resonance at the {\em lowest} frequency 
({\em fundamental frequency} or {\em first harmonic}) 
occurs when there are anti-nodes of the air motion
at the ends – and only there, with a single velocity node at the center, see
Figure~\ref{f:1}. 
The motion of air molecules is highest at the anti-nodes and lowest at the nodes.
%
\begin{figure}[hbtp]
\begin{center}
\includegraphics[width=.4\textwidth]{fig4_1}
\caption{Air molecule displacement in an open and closed pipe.}
\label{f:1} 
\end{center} 
\end{figure}
%
For a pipe with one end closed and one end open, resonance at the lowest
frequency occurs when we have a velocity node at the closed end and an
anti-node at the open end. Plotted in Figure~\ref{f:1} is the displacement or velocity
of air molecules as a function of position along the pipe. The two curves for
each pipe in Figure~\ref{f:1} are one-half period of oscillation apart.

For the pipe with both ends open, we have $L = \lambda/2$ according to 
Figure~\ref{f:1}. 
For the closed pipe we have $L = \lambda/4$. 
The fundamental frequency is given by
%
\be
f_1 = \frac{v}{\lambda} = \frac{v}{2L}
\quad(\text{both ends open})\,,
\qquad
f_1 = \frac{v}{\lambda} = \frac{v}{4L} 
\quad(\text{one end closed})\,,
\ee
%
where $v$ is the velocity of sound.

\subsubsection*{Questions}
\ben
\i For a pipe with both ends open, what are the formulas for the fundamental
frequency and the frequencies $f_2$, $f_3$, $f_4$ of the next three overtones or harmonics?
(Hint: Higher harmonics have frequencies that are integer multiples of the
fundamental, and all integers are allowed for a pipe with both ends open.)

\i
For a pipe with one end closed and one end open, we have $L = \lambda/4$ 
according to Figure~\ref{f:1}. 
Write down the equations for the fundamental frequency and for the
first three existing overtones. 
(Hint: {\em Only odd integers are allowed} as you can see by extending 
the drawings in Figure~\ref{f:1} to higher harmonics.)

\een

EXPERIMENTAL PROCEDURE

In the lab, we can record the frequency spectrum of sound in a pipe
using the setup shown in Figure~\ref{f:2}.
(This figure is specifically for a tube that is open at both ends.)
The speaker produces {\em white noise} and the ``Electroacoustic Toolbox"
can be used to record spectrum.
Figure~\ref{f:3} shows sample recorded frequency spectra for both 
types of tubes.
%
\begin{figure}[hbtp]
\begin{center}
\includegraphics[width=.65\textwidth]{fig4_2}
\caption{Set up of the resonance tube in the ``open tube” configuration. White
noise from the Mac mini is applied to the speaker. The sound
enters the tube on the left and excites the resonances. The microphone on the
right records them for display in the computer. In the ``closed tube” 
configuration the speaker and microphone must be on the same (right) side.}
\label{f:2} 
\end{center} 
\end{figure}
%
%
\begin{figure}[hbtp]
\begin{center}
\includegraphics[width=.7\textwidth]{fig4_3}
\caption{Resonances of a PASCO resonance tube excited with white noise.
Upper figure: Tube open at both ends with an effective length 
$L_{\rm eff,\ open} = 1.18~{\rm m}$.
Lower figure: Tube closed at one end with an effective length 
$L_{\rm eff,\ closed} = L_{\rm eff,\ open}/2 = 0.59~{\rm m}$ 
(with a plug in the tube to shorten its length). 
The fundamental frequency for both tubes is $f_1 = 146~{\rm Hz}$, 
but only the odd harmonics are observed in the tube closed at one end.}
\label{f:3} 
\end{center} 
\end{figure}
%

\subsection{Pipe Length Correction}

If you were to compare the calculated and observed
fundamental frequencies from an actual experiment,
you would find that they do not agree very well.
This has to do with the fact that in pipes, waves 
reflect from the ends of the tube by sticking out a little bit. 
There is an end correction that increases the wavelength. 
This correction is proportional to the radius of the tube. 
Therefore, the larger the tube radius, the more the wave 
will ``stick out” and cause an increase in wavelength. 
The correction results in an extra length $\Delta L$, 
given from theory by $\Delta L = 0.61 R$ for each open end, 
where $R$ is the radius of the pipe. 
Thus for a closed pipe and open pipe of length $L$ and 
radius $R$, the effective lengths are, respectively,
%
\be
L_{\rm eff,\ closed} = L + 0.61 R\,,
\qquad
L_{\rm eff,\ open} = L + 1.22 R\,.
\ee

\subsubsection*{Questions}
\ben
\i Calculate the effective lengths of a closed and open
tube if $L=131~{\rm cm}$ and the diameter of the tube
is $D=14.4~{\rm cm}$.  
(Recall: The diameter  $D$ equals twice the radius $R$.)
\een

%%%%%%%%%%%%%%%%%%%%%%%%%%%%%%%%%%%%%%
\subsection{Determination of the Velocity of Sound}

%
\begin{figure}[hbtp]
\begin{center}
\includegraphics[width=.7\textwidth]{fig4_4}
\caption{Frequency spectrum from a cylindrical PASCO packing tube, 
tapped on the floor with the closed end. 
The large peak is the $N =1$ harmonic (fundamental frequency), 
the small peak is the $N =3$ harmonic. 
The $N =2$ harmonic is missing, as is to be expected for a tube closed at one end.}
\label{f:4} 
\end{center} 
\end{figure}
%

%
\begin{figure}[hbtp]
\begin{center}
\includegraphics[width=.7\textwidth]{fig4_5}
\caption{Decaying waveform of the packing tube.}
\label{f:5} 
\end{center} 
\end{figure}
%
%
\begin{figure}[hbtp]
\begin{center}
\includegraphics[width=.65\textwidth]{fig4_6}
\caption{Experimental setup for spherical Helmholtz resonator.}
\label{f:6} 
\end{center} 
\end{figure}
%

%
\begin{figure}[hbtp]
\begin{center}
\includegraphics[width=.7\textwidth]{fig4_7}
\caption{Helmholtz resonance curve from a large aluminum sphere. The measured
and calculated values of the resonance frequency at the peak are 92 Hz and 93
Hz, respectively.}
\label{f:7} 
\end{center} 
\end{figure}
%
%
\begin{figure}[hbtp]
\begin{center}
\includegraphics[width=.7\textwidth]{fig4_8}
\caption{Helmholtz resonance from a 0.75 liter wine bottle.}
\label{f:8} 
\end{center} 
\end{figure}
%


\end{document}


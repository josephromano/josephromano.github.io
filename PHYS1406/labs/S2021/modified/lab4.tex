\documentclass[11pt]{NSF}
\sloppy

\usepackage{latexsym}
\usepackage{graphicx}
\usepackage{draftcopy}
\usepackage{longtable}
\usepackage{hyperref}
\usepackage{amsmath}

% some definitions
 
% begin equation, itemize, etc.

\def\be{\begin{equation}}
\def\ee{\end{equation}}
\def\bi{\begin{itemize}}
\def\ei{\end{itemize}}
\def\ben{\begin{enumerate}}
\def\een{\end{enumerate}}
\def\i{\item{}}

%%%%%%%%%%%%%%%%%%%%%%%%%%%%%%%%%%%%%%%%%%%%%%%%%%%%%%%%%%%%%%
\begin{document}
     
\section{4. Air Resonance}

PURPOSE AND BACKGROUND

The concept of {\em resonance} in a pipe is similar to that of a string. The waves in
pipes consist of compressions and rarefactions of the air, with back-and-forth
motion of the air molecules in the direction of propagation or against it. The
waves in air thus are {\em longitudinal} waves. In this laboratory we study standing
waves in a pipe. They are the result of two waves traveling in opposite
directions inside the pipe, with each wave being reflected at the ends of the
pipe. In this way the superposition of two waves yields a standing wave,
provided that in addition the {\em resonance conditions} are met.

For a pipe with both ends open, resonance at the {\em lowest} frequency 
({\em fundamental frequency} or {\em first harmonic}) 
occurs when there are anti-nodes of the air motion
at the ends – and only there, with a single velocity node at the center, see
Figure~\ref{f:1}. 
The motion of air molecules is highest at the anti-nodes and lowest at the nodes.
%
\begin{figure}[hbtp]
\begin{center}
\includegraphics[width=.4\textwidth]{fig4_1}
\caption{Air molecule displacement in an open and closed pipe.}
\label{f:1} 
\end{center} 
\end{figure}
%
For a pipe with one end closed and one end open, resonance at the lowest
frequency occurs when we have a velocity node at the closed end and an
anti-node at the open end. Plotted in Figure~\ref{f:1} is the displacement or velocity
of air molecules as a function of position along the pipe. The two curves for
each pipe in Figure~\ref{f:1} are one-half period of oscillation apart.

For the pipe with both ends open, we have $L = \lambda/2$ according to 
Figure~\ref{f:1}. 
For the closed pipe we have $L = \lambda/4$. 
The fundamental frequency is given by
%
\be
f_1 = \frac{v}{\lambda} = \frac{v}{2L}
\quad(\text{both ends open})\,,
\qquad
f_1 = \frac{v}{\lambda} = \frac{v}{4L} 
\quad(\text{one end closed})\,,
\ee
%
where $v$ is the velocity of sound.

%
\subsubsection*{Questions}

%
\begin{figure}[hbtp]
\begin{center}
\includegraphics[width=.65\textwidth]{fig4_2}
\caption{}
\label{f:2} 
\end{center} 
\end{figure}
%
%
\begin{figure}[hbtp]
\begin{center}
\includegraphics[width=.7\textwidth]{fig4_3}
\caption{}
\label{f:3} 
\end{center} 
\end{figure}
%
%
\begin{figure}[hbtp]
\begin{center}
\includegraphics[width=.7\textwidth]{fig4_4}
\caption{}
\label{f:4} 
\end{center} 
\end{figure}
%
%
\begin{figure}[hbtp]
\begin{center}
\includegraphics[width=.7\textwidth]{fig4_5}
\caption{}
\label{f:5} 
\end{center} 
\end{figure}
%
%
\begin{figure}[hbtp]
\begin{center}
\includegraphics[width=.65\textwidth]{fig4_6}
\caption{}
\label{f:6} 
\end{center} 
\end{figure}
%
%
\begin{figure}[hbtp]
\begin{center}
\includegraphics[width=.7\textwidth]{fig4_7}
\caption{}
\label{f:7} 
\end{center} 
\end{figure}
%
%
\begin{figure}[hbtp]
\begin{center}
\includegraphics[width=.7\textwidth]{fig4_8}
\caption{}
\label{f:8} 
\end{center} 
\end{figure}
%


\end{document}


\documentclass[11pt]{NSF}
\sloppy

\usepackage{latexsym}
\usepackage{graphicx}
\usepackage{draftcopy}
\usepackage{longtable}
\usepackage{hyperref}

% some definitions
 
% begin equation, itemize, etc.

\def\be{\begin{equation}}
\def\ee{\end{equation}}
\def\bi{\begin{itemize}}
\def\ei{\end{itemize}}
\def\ben{\begin{enumerate}}
\def\een{\end{enumerate}}
\def\i{\item{}}

%%%%%%%%%%%%%%%%%%%%%%%%%%%%%%%%%%%%%%%%%%%%%%%%%%%%%%%%%%%%%%
\begin{document}
     
\section{9. Electrical Energy and Work, Acoustical Power}

PURPOSE AND BACKGROUND

Electricity is one of the most important energy forms. 
Here we study electrical energy, work, power, voltage, current,
and resistance. 
We will compare the power consumption of a conventional
incandescent light bulb with more efficient 
compact fluorescent and light-emitting diode light bulbs
(CFL and LED light bulbs for short).
We will determine the energy and dollar savings of using a CFL 
or LED light bulb instead of an incandescent light bulb. 
We will also investigate the acoustic power radiated by
a loudspeaker by finding the sound intensity in front of
the speaker. 
The speaker efficiency then follows from the acoustic power 
divided by the electric power. 
We judge how loud a speaker sounds for a given acoustic power 
and find out how acute our sense of hearing is.

\subsection{Some Theory Concerning Power, Energy, Work, and
Electricity}

{\em Energy} is the ability to do work.

{\bf Example:} 1 gallon of gasoline contains energy to do work. An
automobile engine does work and moves a car 30 miles with this energy.

Unit of energy and work: 1~Joule (J)

{\em Power} is the rate at which work is done:
%
\be
{\rm Power} = {\rm Work/Time~Interval}
\quad{\rm or}\quad
P = W/\Delta t
\ee
%
\be
{\rm Work} = {\rm Power}\cdot{\rm Time~Interval}
\quad{\rm or}\quad
W = P\,\Delta t
\ee

Unit of power: 1~J/s = 1 Watt (W)

{\em Ohm's law of electricity}:
%
\be
V=IR
\ee
%
where
$V$ is the voltage across a load
(for instance a light bulb or loudspeaker) in volt (V); 
$I$ is the current through the load in Ampere (A); and
$R$ is the resistance of the load in Ohm ($\Omega$).

Electric power:
\be
P = VI\quad{\rm or,\ equivalently,}\quad
P = I^2 R\quad{\rm and}\quad 
P = V^2/R
\ee

Common unit of energy: 1~kilowatt-hour (kWh)

Conversion: $1~{\rm kWh} = 1000~{\rm W}\times 3600~{\rm s}
=3,600,000~{\rm W}\cdot{\rm s} 
=3,600,000~{\rm J}
=3.6\times 10^6~{\rm J}$

{\bf Example:} A sedentary person consumes 2000~kcal (kilocalories) of food
energy per day. The conversion is $1~{\rm kcal} = 4184~{\rm J}$. 
Therefore $2000~{\rm kcal} = 8,370,000~{\rm J}$. 
This amount of energy is consumed in a time interval 
$\Delta t = 24~{\rm hr} = 86,400~{\rm s}$. 
Hence the rate of energy consumption is 
$P = W/\Delta t =8,370,000~{\rm J}/86,400~{\rm s} = 97~{\rm J/s} = 97~{\rm W}$ 
or about 100~W.  This is typical
for the resting metabolic rate of a person. You may know this rate as
``2000 kcal per day” rather than 100~Watt. When we sit around doing
little, we burn food energy at the rate of 100~W, i.e., about the same
as an old-style 100~W incandescent light bulb consumes in the form of
electrical energy.

\subsection{Power and Energy Consumption in Light Bulbs}

An incadescent light bulb, CFL, and LED light bulb are connected to a 
triple light bulb fixture, which supplies a household voltage 
$V=120~{\rm volt}$ to each bulb.
The power rating of the three bulbs are $P=40~{\rm W}$, 9~W, and 6~W,
respectively, but the bulbs are equally bright.

\subsubsection*{Questions}
\ben
\item 
Suppose you turn the light bulbs on for 5~hours each day. 
Calculate the energy used by each bulb in a 30-day month. 
Express your answers first in Joules, and then convert to kWh.

\item 
Electricity costs about 13~cents/kWh. 
What is the monthly electric bill for each of the light bulbs?

\item 
An incandescent light bulb costs \$0.75 (if still available), 
a CFL \$1.50, and an LED light bulb \$5.00. 
How long does it take to make up for the extra initial cost of 
the CFL and LED light bulbs over the incandescent light bulb?
(Hint: The number of months needed to make up the extra initial 
cost is equal to the difference in their initial costs divided 
by the difference in their monthly costs.)

\item 
How much money is saved over the lifetime of 10,000~hours of a 
CFL and 20,000~hours of an LED light bulb, 
compared to the 2000 hours for an incandescent light bulb?
Include in your calculation the number of incandescent light bulbs 
you would need during the lifetime of a CFL and LED light bulb.
Hint: Use the formula
%
\be
{\rm savings~(in\ dollars)}
=\$0.75\left(\frac{T}{2000~{\rm hr}}\right)
- C_{\rm initial} + \frac{\$0.13}{1000~{\rm
  W}\cdot{\rm hr}}(40~{\rm W}-P)\,{T}
\ee
%
where $T=10,000~{\rm hr}$, $C_{\rm initial}=\$1.50$, and 
$P=9~{\rm W}$ for a CFL; 
or $T=20,000~{\rm hr}$, $C_{\rm initial}=\$5.00$, and 
$P=6~{\rm W}$ for a LED light bulb.

\item What are the energy savings in percent when using a CFL and 
LED light bulb instead of an incandescent light bulb? 
(Hint: Compare the wattages of the three bulbs.)

\een

\subsection{Electric Power to a Loudspeaker}

A sine-wave signal generator is connected to a loudspeaker as shown in
Figure~\ref{f:1}. 
A multi-meter, set to the ammeter mode, is connected in-line between the 
loudspeaker and the signal generator, and 
another multi-meter, set to the voltmeter mode, is connected in parallel 
to the speaker inputs.
%
\begin{figure}[hbtp]
\begin{center}
\includegraphics[width=.5\textwidth]{fig9_1}
\caption{Schematic of the speaker connections to a signal generator,
voltmeter, and ammeter.}
\label{f:1}
\end{center}
\end{figure}
%
When the amplitude of the signal generator is adjusted to produce
a comfortable loudness at $500~{\rm Hz}$, one observes a current 
$I=0.3~{\rm Ampere}$ and voltage $V=5~{\rm volt}$.

\subsubsection*{Questions}
\ben
\item 
Calculate the power to the loudspeaker from the formula $P = IV$.

\item 
Loudspeakers of hi-fi systems often are rated at 100~W or higher. 
How does your answer for our loudspeaker compare with such ratings?

\item 
Do you think a power of several hundred Watt is necessary? Why or why not?
\een

\subsection{Resistance or Impedance of a Loudspeaker, Power Continued}

The reaction of a loudspeaker to an applied alternating current (AC) 
voltage is called {\em impedance}, labeled with the letter $Z$. 
Impedance is not the same as resistance because it also includes 
capacitance and inductance. 
But we will ignore this distinction here and use resistance and impedance
interchangeably.
This means that we will use Ohm's law $V=IR$ to calculate the
resistance from $R=V/I$, and then use this value of $R$ to get an
estimate of the impedance $Z$.

\subsubsection*{Questions}
\ben
\item
Obtain the impedance of the loudspeaker from the voltage $V$ and 
current $I$ given in Part III above.
Compare this with the specification of $16~\Omega$ on the loudspeaker 
enclosure.

\item
Obtain the power to the loudspeaker from the expression $P = I^2 R$.
Compare your answer with the result from Part III, Question 1.

\een

%\subsection{Loudspeaker Power Measured Directly with a Power Meter}

%The loudspeaker power can also be measured directly by using 
%a power meter.
%To do so, one feeds the output from the signal generator directly 
%into the power meter without the loudspeaker, voltmeter, and 
%ammeter in the circuit. 
%The impedance of the power meter should be set to the same value 
%as the impedance found in Part IV above.
%Doing so leads to a power reading of $1.5~{\rm W}$.

\subsection{Acoustical Power and Loudspeaker Efficiency}

Using a sound level meter in setting “A” (which corresponds to
the response of the human ear), we can measure the sound 
loudness level (in phon)  
at various locations in front of and close to the speaker. 
Averaging these measurements at a distance of 1~m from the 
speaker, we find an average loudness level of 90~phon.
%
\begin{figure}[hbtp]
\begin{center}
\includegraphics[width=.9\textwidth]{fig9_2}
\caption{Fletcher-Munson curves of equal loudness.
(From ``Physics of Sound," by R.E.~Berg and D.G.~Stork.)}
\label{f:2}
\end{center}
\end{figure}
%

\subsubsection*{Questions}
\ben
\item 
Consulting the Fletcher-Munson equal loudness curves in Figure~\ref{f:2}, 
what is the sound intensity $I$ in ${\rm W/m}^2$ corresponding to an
average loudness level of 90~phon at a frequency of 500~Hz?

\item 
To calculate the total acoustical power $P_{\rm acoustical}$ 
radiated by the loudspeaker, we need to know the area into which 
the power is radiated.
Assume that this area is the base of a cone in front of the speaker
with area $A = 0.6\pi r^2$, where $r=1~{\rm m}$.
Given this area, calculate the acoustical power using the formula 
$P_{\rm acoustical} = IA$.

\item 
Using the above answer for $P_{\rm acoustical}$ and your result 
from Part III, Question 1 for the electrical power $P_{\rm electrical}$ 
to the speaker, calculate the {\em loudspeaker efficiency}
$(P_{\rm acoustical}/P_{\rm electrical})\times 100\%$.

\item 
Based on your answer to the previous question, 
comment on the conversion of electrical power to acoustical power.

\item 
Compare the acoustical power output from a speaker with the 
light power output from a CFL. 
Do this by assuming that the same electrical power 
(e.g., 9~W) goes into both the speaker and the CFL. 
For the CFL assume a conversion efficiency of 20\% from 
electrical power to light power, so 
$P_{\rm light} = 0.20\,P_{\rm electrical}$. 
For the acoustical power, you will need to increase the 
value that you found in Question 2 above by a factor of 
$9/1.5=6$ to take in account the increase in electrical 
power from 1.5~W to 9~W.
(You should find that the emitted acoustical power is
much lower than the light power output from a CFL.)

\een

\end{document}


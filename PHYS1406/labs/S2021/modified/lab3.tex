\documentclass[11pt]{NSF}
\sloppy

\usepackage{latexsym}
\usepackage{graphicx}
\usepackage{draftcopy}
\usepackage{longtable}
\usepackage{hyperref}

% some definitions
 
% begin equation, itemize, etc.

\def\be{\begin{equation}}
\def\ee{\end{equation}}
\def\bi{\begin{itemize}}
\def\ei{\end{itemize}}
\def\ben{\begin{enumerate}}
\def\een{\end{enumerate}}
\def\i{\item{}}

%%%%%%%%%%%%%%%%%%%%%%%%%%%%%%%%%%%%%%%%%%%%%%%%%%%%%%%%%%%%%%
\begin{document}
     
\section{3. String Resonance}

PURPOSE AND BACKGROUND

THEORY AND EXPERIMENT

%
\begin{figure}[hbtp]
\begin{center}
\includegraphics[width=.85\textwidth]{fig3_1}
\caption{Vibrations of a string. The wavelengths of the standing wave resonance
modes are $\lambda_N = 2L/N$ and the frequencies are 
$f_N = v/\lambda_N = Nv/2L = N f_1$, where $N$ is
the harmonic number, $v$ the velocity of the wave along the string,
and $f_1$ the fundamental frequency.} 
\label{f:1} 
\end{center} 
\end{figure}
%
%
\begin{figure}[hbtp] 
\begin{center} 
\includegraphics[width=\textwidth]{fig3_2}
\caption{PASCO Sonometer Model WA-9611 for studying vibrating strings. The
string is excited with a Driver Coil and the vibrational modes are analyzed
with a Detector Coil. The sine wave generator activates the Driver Coil. The
vibrating string induces a voltage in the Detector Coil. The latter is
connected to the Mac computer. The``Oscilloscope Tool” in the
Electroacoustics Toolbox is used to observe the signal.} 
\label{f:2} 
\end{center} 
\end{figure}
%
%
\begin{figure}[hbtp] 
\begin{center} 
\includegraphics[width=.8\textwidth]{fig3_3}
\caption{Sonometer string excited with a driver coil placed near one of the 
two bridges and harmonics recorded with an electromagnetic detector coil.}
\label{f:3} 
\end{center} 
\end{figure}
%
%
\begin{figure}[hbtp] 
\begin{center} 
\includegraphics[width=.75\textwidth]{fig3_4}
\caption{Sound spectrum form the G3 open string of a violin. 
Top figure: Spectrum from bowed string. 
Bottom figure: Spectrum from plucked string. 
Note that the bowed string has more pronounced higher harmonics 
resulting in a richer sound.}
\label{f:4} 
\end{center} 
\end{figure}
%

%
\begin{table}[hbtp]
\begin{center}
\includegraphics[width=.65\textwidth]{tab3_1}
%\caption{}
\label{t:1}
\end{center}
\end{table}
%
%
\begin{table}[hbtp]
\begin{center}
Table 2: Standing waves on a string\\
\includegraphics[width=.95\textwidth]{tab3_2}
%\caption{}
\label{t:2}
\end{center}
\end{table}
%

\end{document}


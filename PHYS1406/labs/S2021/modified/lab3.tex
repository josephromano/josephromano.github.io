\documentclass[11pt]{NSF}
\sloppy

\usepackage{latexsym}
\usepackage{graphicx}
\usepackage{draftcopy}
\usepackage{longtable}
\usepackage{hyperref}
\usepackage{amsmath}

% some definitions
 
% begin equation, itemize, etc.

\def\be{\begin{equation}}
\def\ee{\end{equation}}
\def\bi{\begin{itemize}}
\def\ei{\end{itemize}}
\def\ben{\begin{enumerate}}
\def\een{\end{enumerate}}
\def\i{\item{}}

%%%%%%%%%%%%%%%%%%%%%%%%%%%%%%%%%%%%%%%%%%%%%%%%%%%%%%%%%%%%%%
\begin{document}
     
\section{3. String Resonance}

PURPOSE AND BACKGROUND

Standing waves on stretched strings and in pipes offer a convenient way to
study vibrations, including the fundamental frequency and harmonics
(overtones). For strings in particular, the frequency depends on the tension,
the mass of the string per unit length ({\em linear mass density}), 
and the total length of the string.
In wind instruments, with air as the vibrating medium, the frequency is defined
by the {\em speed of sound} and the effective length of the pipe. Once the
fundamental frequency is known the higher harmonics are found as simple integer
multiples of that frequency.

THEORY AND EXPERIMENT

When a string is plucked, a {\em transverse} standing wave is 
created on the string---see Figure 1. 
In the simplest case, we have only one {\em anti-node} with maximum
movement in the center. The points at the two ends of the string do not move
and are called {\em nodes}. 
The standing waves result from two waves traveling in {\em opposite}
directions along the string. The superposition of the two waves yields
a standing wave, provided that the {\em resonance conditions} are met.
%
\begin{figure}[hbtp]
\begin{center}
\includegraphics[width=.85\textwidth]{fig3_1}
\caption{Vibrations of a string. The wavelengths of the standing wave resonance
modes are $\lambda_N = 2L/N$ and the frequencies are 
$f_N = v/\lambda_N = Nv/2L = N f_1$, where $N$ is
the harmonic number, $v$ the velocity of the wave along the string,
and $f_1$ the fundamental frequency.} 
\label{f:1} 
\end{center} 
\end{figure}
%

The first 3 vibrational modes of a string are shown in Figure 1. For the
fundamental mode (harmonic number $N = 1$), the wavelength is 
$\lambda = 2L$, where $L$ is the length of the string. 
For the next higher mode, the first overtone or
second harmonic ($N = 2$), the wavelength $\lambda = L$.

The velocity $v$ of the wave on the string 
(not the speed of sound in air!) is given by
%
\be
v = \sqrt{\frac{F}{\mu}}\,,
\ee
%
where $F$ is the {\em tension} on the string and $\mu$ its 
{\em linear mass density} (mass per unit length or kg/m).

For example, a typical metal guitar string has a mass per 
unit length of $\mu = 6.3\times 10^{-3}~{\rm kg/m}$. 
For a tension $F =73.3~{\rm N}$, the velocity of the wave 
along the string is
%
\be
v=\sqrt{\frac{73.3~{\rm N}}{6.3\times 10^{-3}~{\rm kg/m}}}
=108.0~\frac{{\rm m}}{{\rm s}}\,.
\ee
%
The {\em fundamental frequency} $f$ is given by
%
\be
f=\frac{v}{\lambda} = \frac{v}{2L}
\quad\Rightarrow\quad
f=\frac{1}{2L}\sqrt{\frac{F}{\mu}}
\quad(\text{Mersenne's Law})\,.
\ee
%
Note that this frequency also defines the {\em pitch} of 
the sound from the string. Strings on a classical
guitar have a length of $L= 0.65~{\rm m}$. 
With the velocity known, the fundamental frequency is
$f= 108.0/(2\times 0.65)=83.0~{\rm Hz}$.
This is close to the frequency of the E-string of a guitar.

\subsubsection*{Questions}
\ben
\i Discuss how the tension, mass, material, diameter, and 
length of the string affect the fundamental frequency and 
the wave velocity on the string.

\een

%%%%%%%%%%%%%%%%%%%%%%%%%%%%%%%%%%%%%%%%%%%%%%%%%%%%
\subsection{String Vibration Experiments}

One can illustrate standing waves on a string by stretching 
a horizontal string made of flexible fabric over a pulley. 
We place a suitable weight on the vertical end of the string,
and fasten the horizontal end of the string to a vibrator. 
The vibrator is connected to a frequency generator. 
By tuning the frequency of the vibrator to the 
frequency of the fundamental vibrational mode,
we can produce the fundamental (1st harmonic) 
oscillation of the string. 
Then by increasing the frequency of the vibrator,
we can produce the higher
vibrational modes (2nd, 3rd, $\cdots$ harmonics).

\subsubsection*{Questions}
\ben
\i Suppose we change the weight attached to the end of 
the string.
How would that change the fundamental frequency?

\i Suppose we keep the weight the same but change
the length of the string.
How would that change the fundamental frequency?
\een

%%%%%%%%%%%%%%%%%%%%%%%%%%%%%%%%%%%%%%%%%%%%%%%%%%%
\subsection{Sonometer Experiments}

We can also study the resonance modes and frequencies of a 
stretched string using a sonometer.
As shown in Figure~\ref{f:2}, 
the string is excited with a driver coil 
and the vibrational modes are analyzed with a detector coil.
A sample frequency spectrum is shown in Figure~\ref{f:3}.
%
\begin{figure}[hbtp] 
\begin{center} 
\includegraphics[width=\textwidth]{fig3_2}
\caption{PASCO Sonometer Model WA-9611 for studying vibrating strings. The
string is excited with a driver coil and the vibrational modes are analyzed
with a detector coil. The sine wave generator activates the driver coil. The
vibrating string induces a voltage in the detector coil. The latter is
connected to the Mac computer. The ``Oscilloscope Tool” and ``FFT Analyzer
Tool" in the Electroacoustics Toolbox are used to observe the signal.}
\label{f:2} 
\end{center} 
\end{figure}
%
%
\begin{figure}[hbtp] 
\begin{center} 
\includegraphics[width=.7\textwidth]{fig3_3}
\caption{Sound spectrum of a 
sonometer string excited with a driver coil placed near one of the 
two bridges and harmonics recorded with an electromagnetic detector coil.}
\label{f:3} 
\end{center} 
\end{figure}
%

\subsubsection*{Questions}
\ben
\i From Figure~\ref{f:3}, estimate the fundamental frequency (1st harmonic)
and the first three overtones (2nd, 3rd, and 4th harmonic) of the vibrating
string.
Is there a simple relationship between these frequencies?

\i If the length of the string is $L=68.6~{\rm cm}$, what is the wave velocity $v$?
Express your answer in m/s.

\i If the string has a linear mass density of $\mu = 1.5\times 10^{-3}~{\rm kg/m}$,
calculate the tension $F$ in the string needed to 
produce that wave velocity.
\een

%%%%%%%%%%%%%%%%%%%%%%%%%%%%%%%%%
\subsection{Plucked versus Bowed String}

Figure~\ref{f:4} shows the sound  spectrum of the G3 open string of a violin
when it is bowed (top plot) and plucked (bottom plot).
%
\begin{figure}[hbtp] 
\begin{center} 
\includegraphics[width=.7\textwidth]{fig3_4}
\caption{Sound spectrum from the G3 open string of a violin. 
Top figure: Spectrum from bowed string. 
Bottom figure: Spectrum from plucked string.}
%Note that the bowed string has more pronounced higher harmonics 
%resulting in a richer sound.}
\label{f:4} 
\end{center} 
\end{figure}
%

\subsubsection*{Questions}
\ben

\i How many harmonics can you see for the bowed string (Figure~\ref{f:4}, top)? 
Note the relative amplitudes of the harmonics and the 
overall shape of the spectrum.

\i Same as the previous question for the plucked string (Figure~\ref{f:4}, bottom).
Discuss the similarities and differences in the spectra of the plucked and bowed string. 

\i List some reasons why the spectra from the plucked and bowed string are different.

\een

\end{document}


\documentclass[11pt]{NSF}
\sloppy

\usepackage{latexsym}
\usepackage{graphicx}
\usepackage{draftcopy}
\usepackage{longtable}
\usepackage{hyperref}

% some definitions
 
% begin equation, itemize, etc.

\def\be{\begin{equation}}
\def\ee{\end{equation}}
\def\bi{\begin{itemize}}
\def\ei{\end{itemize}}
\def\ben{\begin{enumerate}}
\def\een{\end{enumerate}}
\def\i{\item{}}

%%%%%%%%%%%%%%%%%%%%%%%%%%%%%%%%%%%%%%%%%%%%%%%%%%%%%%%%%%%%%%
\begin{document}
     
\section{2. Wave Phenomena in Water and Air}

PURPOSE AND BACKGROUND

Wave motion is responsible for the propagation of sound. In this laboratory we
study various wave characteristics and how they are related to the production
and propagation of sound. We will take a look at {\em reflection}, 
{\em refraction}, {\em interference}, and {\em diffraction}. 
A ``ripple tank” with water waves is used to
simulate the properties of sound waves. A light source at the ripple tank
illuminates the waves so that they are visible. For actual sound waves we use a
two-speaker system to demonstrate interference.

THEORY AND EXPERIMENT

A wave is a periodic disturbance or fluctuation in a medium about its
equilibrium position. We use water waves as a good example. Waves transport
energy and can do work. A simple {\em sine wave} 
can be used to demonstrate important properties of waves. 
Figure~\ref{f:1} shows the {\em displacement} of the vibrating medium
(e.g., air, water, or a string) as a function of time. The horizontal axis is
time, and the vertical axis is displacement. The equilibrium position
is at $y=0$. The period $T$ is the time for one complete cycle, in other words the
time for a system to return to its initial position.
%
\begin{figure}[hbtp]
\begin{center}
\includegraphics[width=.6\textwidth]{fig2_1}
\caption{Displacement of the medium (e.g., water) of a wave from equilibrium 
as a function of time, for a fixed point of observation.}
\label{f:1}
\end{center}
\end{figure}
%

The frequency of oscillation is defined as the inverse of the period,
$f=1/T$.

The physical unit of frequency is {\em Hertz}, abbreviated Hz. 
The oscillation in Figure~\ref{f:1} 
has a period of $T = 0.10~{\rm s}$ and a frequency of $f = 10~{\rm Hz}$. 
For water, the molecules move up and down (transversely), while the wave 
itself travels in a direction perpendicular to the up and down motion. 
This kind of wave is called a {\em transverse traveling wave}. 
The {\em wavelength} $\lambda$ is the distance the wave travels 
during one ``up and down” cycle. 
It is the distance from any one crest
to the next nearest crest, or from wave trough to next trough, or between any
two corresponding points having the same {\em phase}---see Figure 2. 
If we call $v$ the wave speed, then we have
\be
v = \frac{\lambda}{T} = f\lambda\,.
\label{e:v=f*lambda}
\ee
%
\begin{figure}[hbtp]
\begin{center}
\includegraphics[width=.6\textwidth]{fig2_2}
\caption{A transverse wave traveling with velocity $v$ and wavelength 
$\lambda$. 
We see a snapshot at a fixed time with vertical direction 
showing the displacement of
the medium and horizontal direction the position of the wave.}
\label{f:2}
\end{center}
\end{figure}
%

%%%%%%%%%%%%%%%%%%%%%%%%%%%%%%%%%%%%%%%%%
\subsection{Wave Velocity}

We can use a PASCO ripple tank to produce plane waves in the ``strobe
light" setting.
The direction of the traveling waves is perpendicular to the 
crests and troughs of the waves.
For a frequency setting of $f=20~{\rm Hz}$, we measure the
length of 5 consecutive crests and determine the wavelength $\lambda$
by dividing that distance by 5.

\subsubsection*{Questions}
\ben

\i 
If the distance between 5 consecutive crests is 11~cm, what is the 
velocity $v$ of the waves according to equation (\ref{e:v=f*lambda})?
Write your answer in units of m/s.
\een

%%%%%%%%%%%%%%%%%%%%%%%%%%%%%%%%%%%%%%%%%%%%%%%%%
\subsection{Reflection}
%
\begin{figure}[hbtp]
\begin{center}
\includegraphics[width=.6\textwidth]{fig2_3}
\caption{Incoming wave reflected off a smooth surface.}
\label{f:3}
\end{center}
\end{figure}
%
When sound waves hit a barrier such as a wall, some of the sound is reflected
(with the rest absorbed by the wall). 
Waves obey the {\em law of reflection}. 
A line
drawn perpendicular to a point on the wall is called the ``surface normal”---see
Figure 3. The angle that the incoming wave makes with the normal is called the
{\em angle of incidence}. The law of reflection states that for a wave approaching a
barrier, the wave will be reflected from the surface at an angle equal to the
angle of incidence. For the law of reflection to hold, the surface roughness
must be small compared to the size of the wavelength. In other words, we need a
smooth, or optically speaking, a ``mirror- like” surface. This is the case in
our experiments with the ripple tank.

\subsubsection*{Questions}
\ben
\i Suppose a traveling plane water wave is incident 
on a concave piece of plastic which acts as a 
``mirror" for the water waves.
Draw a diagram showing the direction of the incident and 
reflected waves for different incident locations on the concave surface.
You should see focusing of the waves in analogy to
an optical mirror.

\een

%%%%%%%%%%%%%%%%%%%%%%%%%%%%%%%%%%%%%%%%%%%%%%
\subsection{Refraction}

Refraction means a change in the direction in which a wave travels 
(see Figure~\ref{f:4}). 
This happens for instance in water where the depth changes, and the wave
speed changes as a consequence. 
(For the velocity for shallow water waves, we have 
$v = \sqrt{gd}$, where $g = 9.8~{\rm m/s}^2$ and $d$ is
the depth of the water.)
Although refraction has only limited
applications to sound propagation in enclosed rooms such as our laboratory, it
accounts for some interesting atmospheric phenomena (see below). Refraction
also occurs with light waves, where it accounts for the action of optical
lenses. In all cases where refraction occurs, the wave speed and direction of
propagation change.
%
\begin{figure}[hbtp]
\begin{center}
\includegraphics[width=.6\textwidth]{fig2_4}
\caption{Refraction of a wave at the interface between two media, where 
the wave speed is different in the two media.}
\label{f:4}
\end{center}
\end{figure}
%

\subsubsection{Refraction of Sound Waves in the Atmosphere}

The speed of sound depends on the temperature of the air. Cooler, denser air
will transmit sound more slowly than warmer air. Under normal conditions, the
air near the ground is warmer than the air above it. This is the reason why you
may not always hear the thunder from a lightning strike several miles away: The
sound traveling through the cold air higher up travels more slowly than through
the warm air closer to the ground. The sound therefore is refracted upwards and
may not reach you. In contrast, a {\em temperature inversion}, where the air is
cooler closer to the ground, produces the opposite effect. A cool lake at night
and in the morning hours can cause such a temperature inversion: Sound is
refracted downwards towards the listener, effectively amplifying the direct
sound across the lake. This makes the sound, for instance from people on the
opposite shore, sound louder and closer than it actually is.

\subsubsection*{Questions}
\ben
\i Draw a diagram illustrating how the direction of sound propagation changes
for the case of a temperature inversion.

\een

%%%%%%%%%%%%%%%%%%%%%%%%%%%%%%%%%%%%%%%%%%
\subsection{Interference of Waves}

When two or more wave trains move through the same region of space, the waves
interfere with each other at any given spot. 
{\em Constructive interference} occurs
when two waves with the same phase, such as two wave crests, align at the same
location. The two amplitudes add together to create a ``hotspot” of twice the
amplitude and thus a maximum in intensity---see Figure~\ref{f:5}.
%
\begin{figure}[hbtp]
\begin{center}
\includegraphics[width=.95\textwidth]{fig2_5}
\caption{Interference and superposition of two waves. 
The diagram on the left
shows {\em constructive} interference and on the right 
{\em destructive} interference.}
\label{f:5}
\end{center}
\end{figure}
%

On the other hand, if the wave crest of one wave meets with a wave trough of
another wave, the two waves are completely out-of-phase and suffer 
{\em destructive interference}. The resultant amplitude is nearly zero, and so is the wave
intensity---again see Figure~\ref{f:5}.

\subsubsection*{Questions}
\ben

\i Draw a diagram similar to Figure~\ref{f:5} 
showing the interference of two sine waves that are 90~degrees
out of phase with one another 
(i.e., so that the crest of one wave occurs at the zero of the other wave, etc.)

\een

\subsubsection{Interference of Water Waves}

We can study the interference maxima and minima with water waves in the 
``ripple tank” using the wave generator with two small point-like dippers, 
each producing circular waves. 
If we space the plungers a few wavelengths apart, and choose a 
frequency between 15 and 20 Hz, we observe the interference pattern
similar to that shown in Figure~\ref{f:6}.
The radial lines correspond to regions of destructive interference
where the two circular waves cancel each other out (e.g., at the top 
black dot in Figure~\ref{f:6}).
Between those radial lines are regions of constructive interference
(e.g., at the bottom black dot in Figure~\ref{f:6}).
%
\begin{figure}[hbtp]
\begin{center}
\includegraphics[width=.5\textwidth]{fig2_6}
\caption{Interference of circular water waves.
[Image from {\tt chegg.com}.]}
\label{f:6}
\end{center}
\end{figure}
%

\subsubsection*{Questions}

\ben
\i Suppose we change the frequency of the dippers.
Describe how the interference pattern changes.
(For example, does increasing the frequency of the 
dippers increase or decrease the number of radial 
lines in the interference pattern.)

\een

%%%%%%%%%%%%%%%%%%%%%%%%%%%%%%%%
\subsubsection{Interference of Sound Waves}

Sound waves from two speakers behave exactly like the 
interfering water waves in the ripple tank shown
in Figure~\ref{f:6}.
For interference from the speakers to be clearly audible, 
the wavelength should be somewhat smaller than the distance between the speakers.
By walking in front of the speakers, you can hear the
interference maxima (constructive interference) and 
minima (destructive interference) as you walk.

\subsubsection*{Questions}

\ben
\i At the midpoint in front of the speakers (i.e., at the 
same distance from each speaker) would you hear 
constructive or destructive interference? Why?

\i Would the number of audible maxima and minima 
increase or decrease if the frequency is increased?

\i Do you think the interference of sound waves is desirable 
or undesirable in rooms and concert halls?  
Why? How would you address such problems?

\een

%%%%%%%%%%%%%%%%%%%%%%%%%%%%%%%%%%%%%%%%%%%%%%
\subsection{Diffraction of Waves}

Diffraction is a wave phenomenon with direct applications to sound propagation
when there is a barrier, opening, or corner. The effect is pronounced when the
wavelength is comparable to the size of the obstacle. In such cases diffraction
enables one to hear sound ``around corners”---see Figure~\ref{f:7}. 
We all have heard
sound from a door opening when we were outside a room but not in the
line-of-sight of the sound source inside.
%
\begin{figure}[hbtp]
\begin{center}
\includegraphics[width=.95\textwidth]{fig2_7}
\caption{Diffraction of sound waves after passing through an opening, 
around a barrier, and around an edge.}
\label{f:7}
\end{center}
\end{figure}
%

The amount of diffraction depends on the relative size of the wavelength
of the waves and the size of the opening or barrier obstructing the 
passage of the waves.
For example, suppose a wave approaches a barrier with a width large
compared to the wavelength.
Then there is a ``shadow” region without waves behind the barrier, 
as expected. 
If, however, the wavelength is comparable to the barrier width, the 
``shadow” region behind the barrier becomes small and waves travel into 
that region.

P.S. Water waves are an example of {\em transverse waves}, 
whereas sound waves are {\em longitudinal waves}. 
In the first case the medium (water) oscillates {\em transversely} to 
the direction of wave propagation; in the second case (air) the 
oscillations are {\em longitudinal} along and against the 
propagation direction. 
These differences do not affect the basic study of wave behavior in this laboratory.

\subsubsection*{Questions}
\ben

\i For the case of plane waves incident on an opening, sketch the 
diffracted waves for the case where the wavelength of the waves is 
small compared to the size of the opening.

\i Same as the previous question but for the case where the 
wavelength of the waves is comparable to the size of the opening.
\een

\end{document}


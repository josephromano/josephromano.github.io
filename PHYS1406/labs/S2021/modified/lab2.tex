\documentclass[11pt]{NSF}
\sloppy

\usepackage{latexsym}
\usepackage{graphicx}
\usepackage{draftcopy}
\usepackage{longtable}
\usepackage{hyperref}

% some definitions
 
% begin equation, itemize, etc.

\def\be{\begin{equation}}
\def\ee{\end{equation}}
\def\bi{\begin{itemize}}
\def\ei{\end{itemize}}
\def\ben{\begin{enumerate}}
\def\een{\end{enumerate}}
\def\i{\item{}}

%%%%%%%%%%%%%%%%%%%%%%%%%%%%%%%%%%%%%%%%%%%%%%%%%%%%%%%%%%%%%%
\begin{document}
     
\section{2. Wave Phenomena in Water and Air}

PURPOSE AND BACKGROUND

Wave motion is responsible for the propagation of sound. In this laboratory we
study various wave characteristics and how they are related to the production
and propagation of sound. We will take a look at {\em reflection}, 
{\em refraction}, {\em interference}, and {\em diffraction}. 
A ``ripple tank” with water waves is used to
simulate the properties of sound waves. A light source at the ripple tank
illuminates the waves so that they are visible. For actual sound waves we use a
two-speaker system to demonstrate interference and diffraction.

THEORY AND EXPERIMENT

A wave is a periodic disturbance or fluctuation in a medium about its
equilibrium position. We use water waves as a good example. Waves transport
energy and can do work. A simple {\em sine wave} 
can be used to demonstrate important properties of waves. 
Figure~\ref{f:1} shows the {\em displacement} of the vibrating medium
(e.g., air, water, or a string) as a function of time. The horizontal axis is
the time, and the vertical axis is the displacement. The equilibrium position
is at $x=0$. The period $T$ is the time for one complete cycle, in other words the
time for a system to return to its initial position.
%
\begin{figure}[hbtp]
\begin{center}
\includegraphics[width=.6\textwidth]{fig2_1}
\caption{Displacement of the medium (e.g., water) of a wave from equilibrium 
as a function of time, for a fixed point of observation.}
\label{f:1}
\end{center}
\end{figure}
%

The frequency of oscillation is defined as the inverse of the period,
$f=1/T$.

The physical unit of the frequency is {\em Hertz}, abbreviated Hz. 
The oscillation in Figure~\ref{f:1} 
has a period of $T = 0.10~{\rm s}$ and a frequency of $f = 10~{\rm Hz}$. 
For water, the molecules move up and down (transversely), while the wave 
itself travels in a direction perpendicular to the up and down motion. 
This kind of wave is called a {\em transverse traveling wave}. 
The {\em wavelength} $\lambda$ is the distance the wave travels 
during one ``up and down” cycle. 
It is the distance from any one crest
to the next nearest crest, or from wave trough to next trough, or between any
two corresponding points having the same {\em phase}---see Figure 2. 
If we call $v$ the wave speed, then we have
\be
v = \frac{\lambda}{T} = f\lambda\,.
\label{e:v=f*lambda}
\ee
%
\begin{figure}[hbtp]
\begin{center}
\includegraphics[width=.6\textwidth]{fig2_2}
\caption{A transverse wave traveling with velocity $v$ and wavelength 
$\lambda$. 
We see a snapshot at a fixed time with vertical direction 
showing the displacement of
the medium and horizontal direction the position of the wave.}
\label{f:2}
\end{center}
\end{figure}
%

%%%%%%%%%%%%%%%%%%%%%%%%%%%%%%%%%%%%%%%%%
\subsection{Wave Velocity}

We can use the PASCO ripple tank to produce plane waves in the ``strobe
light" setting.
The direction of the traveling waves is perpendicular to the straight
crests and troughs of the waves.
For a frequency setting of $f=20~{\rm Hz}$, we measure the
length of 5 consecutive crests and determine the wavelength $\lambda$
by dividing that distance by 5.

\subsubsection*{Questions}
\ben

\i 
If the distance between 5 consecutive crests is 11~cm, what is the 
velocity $v$ of the waves according to equation (\ref{e:v=f*lambda})?
Write your answer in units of $m/s$.
\een

%%%%%%%%%%%%%%%%%%%%%%%%%%%%%%%%%%%%%%%%%%%%%%%%%
\subsection{Reflection}
%
\begin{figure}[hbtp]
\begin{center}
\includegraphics[width=.6\textwidth]{fig2_3}
\caption{Incoming wave reflected off a smooth surface.}
\label{f:3}
\end{center}
\end{figure}
%
When sound waves hit a barrier such as a wall, some of the sound is reflected
(with the rest absorbed by the wall). 
Waves obey the {\em law of reflection}. 
A line
drawn perpendicular to a point on the wall is called the ``surface normal”---see
Figure 3. The angle that the incoming wave makes with the normal is called the
{\em angle of incidence}. The law of reflection states that for a wave approaching a
barrier, the wave will be reflected from the surface at an angle equal to the
angle of incidence. For the law of reflection to hold, the surface roughness
must be small compared to the size of the wavelength. In other words, we need a
smooth, or optically speaking, a ``mirror- like” surface. This is the case in
our experiments with the ripple tank.

\subsubsection*{Questions}
\ben
\i Suppose a traveling plane water wave is incident 
on a concave piece of plastic which acts as a 
``mirror" for the water waves.
Draw a diagram showing the direction of the incident and 
reflected waves for different locations on the concave surface.
You should see focusing of the waves in analogy to
an optical mirror.

\een

%%%%%%%%%%%%%%%%%%%%%%%%%%%%%%%%%%%%%%%%%%%%%%
\subsection{Refraction}

Refraction means a change in the direction in which a wave travels 
(see Figure~\ref{f:4}). 
This happens for instance in water where the depth changes, and the wave
speed changes as a consequence. Although refraction has only limited
applications to sound propagation in enclosed rooms such as our laboratory, it
accounts for some interesting atmospheric phenomena (see below). Refraction
also occurs with light waves, where it accounts for the action of optical
lenses. In all cases where refraction occurs, the wave speed and direction of
propagation change.
%
\begin{figure}[hbtp]
\begin{center}
\includegraphics[width=.6\textwidth]{fig2_4}
\caption{Refraction of wave on a barrier such as in water.}
\label{f:4}
\end{center}
\end{figure}
%

REFRACTION OF SOUND WAVES IN THE ATMOSPHERE

The speed of sound depends on the temperature of the air. Cooler, denser air
will transmit sound more slowly than warmer air. Under normal conditions, the
air near the ground is warmer than the air above it. This is the reason why you
may not always hear the thunder from a lightning strike several miles away: The
sound traveling through the cold air higher up travels more slowly than through
the warm air closer to the ground. The sound therefore is refracted upwards and
may not reach you. In contrast, a {\em temperature inversion}, where the air is
cooler closer to the ground, produces the opposite effect. A cool lake at night
and in the morning hours can cause such a temperature inversion: Sound is
refracted downwards towards the listener, effectively amplifying the direct
sound across the lake. This makes the sound, for instance from people on the
opposite shore, sound louder and closer than it actually is.

%%%%%%%%%%%%%%%%%%%%%%%%%%%%%%%%%%%%%%%%%%
\subsection{Interference of Waves}

When two or more wave trains move through the same region of space, the waves
interfere with each other at any given spot. 
{\em Constructive interference} occurs
when two waves with the same phase, such as two wave crests, align at the same
location. The two amplitudes add together to create a ``hotspot” of twice the
amplitude and thus a maximum in intensity---see Figure~\ref{f:5}.
%
\begin{figure}[hbtp]
\begin{center}
\includegraphics[width=.95\textwidth]{fig2_5}
\caption{Interference and superposition of two waves. 
The diagram on the left
shows {\em constructive} interference and on the right 
{\em destructive} interference.}
\label{f:5}
\end{center}
\end{figure}
%

On the other hand, if the wave crest of one wave meets with a wave trough of
another wave, the two waves are completely out-of-phase and suffer 
{\em destructive interference}. The resultant amplitude is nearly zero, and so is the wave
intensity---again see Figure~\ref{f:5}.

\subsubsection*{Questions}
\ben

\i Draw a diagram similar to Figure~\ref{f:5} 
showing the interference of two sine waves that are 90~degrees
out of phase with one another 
(so that the crest of one wave occurs at the zero of the other wave, etc.)

\een

INTERFERENCE OF WATER WAVES

We can study the interference maxima and minima with water waves in the 
``ripple tank” using the wave generator with two small point-like dippers, 
each producing circular waves. 
If we space the plungers a few wavelengths apart, and choose a 
frequency between 15 and 20 Hz, we observe the interference pattern
similar to that shown in Figure~\ref{f:6}.
The radial lines correspond to regions of destructive interference
where the two circular waves cancel each other out (e.g., the 
top black top in Figure~\ref{f:6}).
Between those radial lines are regions of constructive interference
(e.g.,the bottom black dot in Figure~\ref{f:6}).
%
\begin{figure}[hbtp]
\begin{center}
\includegraphics[width=.6\textwidth]{fig2_6}
\caption{Interference of circular water waves.
Image from {\tt chegg.com}.}
\label{f:6}
\end{center}
\end{figure}
%

%%%%%%%%%%%%%%%%%%%%%%%%%%%%%%%%%%%%%%%%%%%%%%
\subsection{Diffraction of Waves}


%
\begin{figure}[hbtp]
\begin{center}
\includegraphics[width=.95\textwidth]{fig2_7}
\caption{Diffraction of sound waves after passing through an opening, 
around a barrier, and around an edge.}
\label{f:6}
\end{center}
\end{figure}
%

\end{document}


\documentclass[11pt]{NSF}
\sloppy

\usepackage{latexsym}
\usepackage{graphicx}
\usepackage{draftcopy}
\usepackage{longtable}
\usepackage{hyperref}

% some definitions
 
% begin equation, itemize, etc.

\def\be{\begin{equation}}
\def\ee{\end{equation}}
\def\bi{\begin{itemize}}
\def\ei{\end{itemize}}
\def\ben{\begin{enumerate}}
\def\een{\end{enumerate}}
\def\i{\item{}}

%%%%%%%%%%%%%%%%%%%%%%%%%%%%%%%%%%%%%%%%%%%%%%%%%%%%%%%%%%%%%%
\begin{document}
     
\section{2. }

PURPOSE AND BACKGROUND

Wave motion is responsible for the propagation of sound. In this laboratory we
study various wave characteristics and how they are related to the production
and propagation of sound. We will take a look at {\em reflection}, 
{\em refraction}, {\em interference}, and {\em diffraction}. 
A ``ripple tank” with water waves is used to
simulate the properties of sound waves. A light source at the ripple tank
illuminates the waves so that they are visible. For actual sound waves we use a
two-speaker system to demonstrate interference and diffraction.

THEORY AND EXPERIMENT

A wave is a periodic disturbance or fluctuation in a medium about its
equilibrium position. We use water waves as a good example. Waves transport
energy and can do work. A simple {\em sine wave} 
can be used to demonstrate important properties of waves. 
Figure~\ref{f:1} shows the {\em displacement} of the vibrating medium
(e.g., air, water, or a string) as a function of time. The horizontal axis is
the time, and the vertical axis is the displacement. The equilibrium position
is at $x=0$. The period $T$ is the time for one complete cycle, in other words the
time for a system to return to its initial position.

%
\begin{figure}[hbtp]
\begin{center}
\includegraphics[width=.6\textwidth]{fig2_1}
\caption{Displacement of the medium (e.g., water) of a wave from equilibrium 
as a function of time, for a fixed point of observation.}
\label{f:1}
\end{center}
\end{figure}
%

%
\begin{figure}[hbtp]
\begin{center}
\includegraphics[width=.6\textwidth]{fig2_2}
\caption{A transverse wave traveling with velocity $v$ and wavelength 
$\lambda$. 
We see a snapshot at a fixed time with vertical direction 
showing the displacement of
the medium and horizontal direction the position of the wave.}
\label{f:2}
\end{center}
\end{figure}
%

%
\begin{figure}[hbtp]
\begin{center}
\includegraphics[width=.6\textwidth]{fig2_3}
\caption{Incoming wave reflected off a smooth surface.}
\label{f:3}
\end{center}
\end{figure}
%

%
\begin{figure}[hbtp]
\begin{center}
\includegraphics[width=.6\textwidth]{fig2_4}
\caption{Refraction of wave on a barrier such as in water.}
\label{f:4}
\end{center}
\end{figure}
%

%
\begin{figure}[hbtp]
\begin{center}
\includegraphics[width=.95\textwidth]{fig2_5}
\caption{Interference and superposition of two waves. 
The diagram on the left
shows {\em constructive} interference and on the right 
{\em destructive} interference.}
\label{f:5}
\end{center}
\end{figure}
%

%
\begin{figure}[hbtp]
\begin{center}
\includegraphics[width=.95\textwidth]{fig2_6}
\caption{Diffraction of sound waves after passing through an opening, 
around a barrier, and around an edge.}
\label{f:6}
\end{center}
\end{figure}
%

\end{document}


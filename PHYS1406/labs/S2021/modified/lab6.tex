\documentclass[11pt]{NSF}
\sloppy

\usepackage{latexsym}
\usepackage{graphicx}
\usepackage{draftcopy}
\usepackage{longtable}
\usepackage{hyperref}

% some definitions
 
% begin equation, itemize, etc.

\def\be{\begin{equation}}
\def\ee{\end{equation}}
\def\bi{\begin{itemize}}
\def\ei{\end{itemize}}
\def\ben{\begin{enumerate}}
\def\een{\end{enumerate}}
\def\i{\item{}}

%%%%%%%%%%%%%%%%%%%%%%%%%%%%%%%%%%%%%%%%%%%%%%%%%%%%%%%%%%%%%%
            
\begin{document}

\section{6. Spectrum Analysis of Instruments and Voice}

PURPOSE AND BACKGROUND

We continue with our discussion of harmonics. 
All musical instruments produce tones that
are unique and have a characteristic {\em timbre} or quality of sound. The
frequency spectra tell the harmonic content of a tone and how it can
be synthesized. Electronic keyboards make use of this to reproduce
sounds. We analyze the sound from a variety of string instruments, 
wind instruments, and the human voice.

\subsection{String Instruments}

All musical instruments use a driving force to set an oscillator into
motion. Stringed instruments use a bow or plucking for exciting the
vibrations. 
The strings of a violin are tuned to the notes G3, D4, A4, and E5;
for the guitar, they are E2, A2, D3, G3, B3 and E4. You see that the
note G3 is common to both instruments. 
Figure~\ref{f:1} shows an example of a frequency spectrum from the 
open G3 strings of a guitar (top) and violin (bottom).
%
\begin{figure}[hbtp]
\begin{center}
\includegraphics[width=.6\textwidth]{fig6_1}
\caption{Frequency spectra of the plucked open G3 strings of a guitar
(top) and violin (bottom). The fundamental frequency
(pitch) and frequencies of the harmonics are the same. But the timbre
(quality of sound) is very different because of the different relative
amplitudes of the harmonics.}
\label{f:1}
\end{center}
\end{figure}


\ben
\item 
What differences do you see in the frequency spectra for the guitar
and the violin?

\item Placing a finger down on the fingerboard reduces the effective 
length of the string and increases the pitch.
How do you think the two spectra would change if you increase the 
pitch of the G3 strings on the guitar and violin?
What will be similar?

\item How do you think the spectrum for the violin would change if 
the violin string were {\em bowed} instead of plucked?

\item The lowest and highest notes on a violin have fundamental
frequencies of 196~Hz (open G3 string) and 2093~Hz (C7 played on 
the E3 string).  
Approximately how many octaves apart are these two notes?
\een

\subsection{Wind Instruments} 
Wind instruments use air as the vibrating medium. Brass instruments
have closed pipes, with the closed end near the mouth. Woodwinds such
as the clarinet, oboe and bassoon also are closed pipes, with a reed
at the closed end. The lowest harmonics of these instruments are
primarily the odd harmonics, as is expected for closed pipes. This
applies especially to the clarinet due to its straight cylindrical
bore. For other wind instruments, all harmonics are present without a
special dominance of odd or even harmonics. Flutes, piccolos,
recorders and, more exotically, an ocarina, are open pipes with even
and odd harmonics.  

Figure~\ref{f:2} shows the simple, almost purely sinusoidal, frequency
spectrum from a slide whistle.
%
\begin{figure}[hbtp]
\begin{center}
\includegraphics[width=.6\textwidth]{fig6_2}
\caption{Frequency spectrum of a slide whistle with 
$f =880~\textrm{Hz}$.}
\label{f:2}
\end{center}
\end{figure}

\ben
\item
How does the frequency spectrum of the slide whistle compare to that 
of the guitar or violin?

\item Since a slide whistle is closed at one end, what harmonics might 
you produce if you ``over-blow" the lowest note?

\een

\subsection{Voice}

The human vocal tract is an intricate system for producing sound. The
voice of each person is unique. The sound is produced, and its quality
determined, by the vocal tract consisting of throat, nasal cavity, and
mouth. Each of these components acts as a resonator with
characteristic resonant frequencies. The different vowel sounds come
from different regions of the vocal tract. This allows for a large
variety of sounds, but some general characteristics exist.

The frequency regions where several neighboring harmonics have high
amplitudes are called {\em vocal formants}. Most people have similar
formants because of the similar size and shape of their vocal tracts. 
The individual resonators of the tract produce the different formants.
They can be adjusted by a change in size and shape of the throat,
nasal cavity, and mouth. How this is done distinguishes a great singer
from a bad one. Vocal formants are what we listen to in order to
recognize persons. Adjusting the cavities of the vocal tract changes
the formant regions. Adjusting the tension in the vocal cords changes
the pitch and associated harmonics.

Figure~\ref{f:4} shows the frequency spectra of a male voice (top) 
and female voice (bottom) making the vowel sound ``ee".
Note the dominant formant regions.
%
\begin{figure}[hbtp]
\begin{center}
\includegraphics[width=.6\textwidth]{fig6_4}
\caption{Vowel sound ``ee” from a male voice (top) and 
female voice (bottom). 
The female voice has purer harmonics. 
Note the formant regions.}
\label{f:4}
\end{center}
\end{figure}

\ben

\item The human throat has a typical overall length of 17 cm. Consider
it as a simple pipe, with one end closed at the vocal folds and the
other open at the mouth. What is the fundamental resonance frequency?
What are the frequencies of the next three harmonics?

\item Identify any formant regions in the male and female voice in
Figure~\ref{f:4}, where the amplitudes are pronounced.
Are they located around similar freqeuncies?

\item How can you effectively change the resonant frequencies of the
vocal tract (formant regions)?
How can you change the frequencies of the vocal folds?

\item Telephones transmit frequencies only in the approximate range
300-3000~Hz. Why do you think this frequency range is 
sufficient for most purposes?

\een

\end{document}

%%%%%%%%%%%%%%%%%%%
\begin{table}[hbtp]
\begin{center}
\begin{tabular}{ |p{0.8in} | p{0.8in} | p{0.9in}|}
\hline
Sound & 1st formant region (Hz) & 2nd formant region (Hz) \\
\hline
oo & & \\  
\hline
ah & & \\  
\hline
ee & & \\  
\hline
\end{tabular}
\caption{}
\label{t:}
\end{center}
\end{table}

\begin{figure}[hbtp]
\begin{center}
\includegraphics[width=.6\textwidth]{fig6_3}
\caption{}
\label{f:}
\end{center}
\end{figure}

\begin{figure}[hbtp]
\begin{center}
\includegraphics[width=.6\textwidth]{fig6_5}
\caption{}
\label{f:}
\end{center}
\end{figure}



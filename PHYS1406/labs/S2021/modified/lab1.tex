\documentclass[11pt]{NSF}
\sloppy

\usepackage{latexsym}
\usepackage{graphicx}
\usepackage{draftcopy}
\usepackage{longtable}
\usepackage{hyperref}

% some definitions
 
% begin equation, itemize, etc.

\def\be{\begin{equation}}
\def\ee{\end{equation}}
\def\bi{\begin{itemize}}
\def\ei{\end{itemize}}
\def\ben{\begin{enumerate}}
\def\een{\end{enumerate}}
\def\i{\item{}}

%%%%%%%%%%%%%%%%%%%%%%%%%%%%%%%%%%%%%%%%%%%%%%%%%%%%%%%%%%%%%%
\begin{document}
     
\section{1. Topic}

PURPOSE AND BACKGROUND

In order to understand sound and music, we need to understand periodic motion
and how it gives rise to sound. Periodic motion is any sort of movement that
repeats itself after an amount of time called the period. For example, a violin
string or the reed of a bassoon exhibit periodic motion when playing a
sustained tone. A grandfather clock exhibits periodic motion as the pendulum
swings back and forth, and so does a Ferris wheel that rotates at a constant
speed.

Simple harmonic motion (SHM) is the purest form of periodic motion. Two
conditions have to be met:

1) There exists a stable equilibrium position. If the system is at rest it will
stay at rest there. It will tend to return to that position if displaced from
it.

2) There exists a restoring force towards the equilibrium position. This force
is proportional to the amount of displacement from equilibrium. For example, if
a mass hanging from a spring originally is at rest and then pulled down a small
distance, the mass will oscillate up and down with SHM when let go. The spring
provides a restoring force to bring the mass back to the equilibrium position.
If the mass instead is pulled twice as far, the spring provides twice the force
to bring it back. In this manner, the system is linear and it is said to obey
Hooke’s Law.

Much of music and sound is generated from periodic vibrations of the air or
solid material in musical instruments. Examples are the vibrating strings of a
violin and the reeds of woodwind instruments. In practice, however, very few
musical tones come from pure SHM. That sound actually would be rather boring.
Instead, musical tones consist of a combination of harmonics - see a tone from
a plucked violin string in Figure 1. The lowest frequency corresponding to the
first peak is called the fundamental frequency. This is the only frequency
present in SHM. The peaks at the higher frequencies in Figure 1 are the higher
harmonics or overtones that make up the tone. We shall discuss this in more
detail in later laboratories.

\subsection{XXX}

text

\ben
\item
question 1?

\item
question 2?
\een

text

\subsection{YYY}

\ben
\item
question 1?

\een

%
\begin{figure}[hbtp]
\begin{center}
\includegraphics[width=.95\textwidth]{fig1_1}
\caption{Frequency spectrum of a plucked string showing the 
fundamental frequency and higher harmonics (overtones).}
\label{f:1}
\end{center}
\end{figure}
%
%
\begin{figure}[hbtp]
\begin{center}
\includegraphics[width=.4\textwidth]{fig1_2}
\caption{Pendulum and spring.}
\label{f:2}
\end{center}
\end{figure}
%
%
\begin{figure}[hbtp]
\begin{center}
\includegraphics[width=.75\textwidth]{fig1_3}
\caption{Sonometer with two vibrating strings.}
\label{f:3}
\end{center}
\end{figure}
%
%

\begin{table}[hbtp]
\begin{center}
Pendulum Period $T$ (seconds)\\
\begin{tabular}{| c | c | c | c | c | }
\hline
&\multicolumn{2}{c}{$L=20.0~{\rm cm}$} \vrule
&\multicolumn{2}{c}{$L=80.0~{\rm cm}$} \vrule\\
\hline
Trial & \phantom{ }Pb\phantom{ } & Al & \phantom{ }Pb\phantom{ }\  & Al \\
\hline
\# 1 &  &  &  &  \\
\hline
\# 2 &  &  &  &  \\
\hline
\# 3 &  &  &  &  \\
\hline
avg  &  &  &  &  \\
\hline
\end{tabular}
%\caption{Table}
\label{t:1}
\end{center}
\end{table}

%
\begin{table}[hbtp]
\begin{center}
Spring Period\\
\begin{tabular}{| c | c | c | c | c | }
\hline
Trial & Mass $m$ (g) & Period $T$ (s) & Mass $m$ (g) & Period $T$ (s) \\
\hline
\# 1 &  &  &  &  \\
\hline
\# 2 &  &  &  &  \\
\hline
\# 3 &  &  &  &  \\
\hline
avg  &  &  &  &  \\
\hline
\end{tabular}
%\caption{Table}
\label{t:1}
\end{center}
\end{table}
%

%
\begin{table}[hbtp]
\begin{center}
String Period\\
\begin{tabular}{| c | c | c | }
\hline
Tension mass $m$ (g) & Frequency $f$ (Hz) & Period $T$ (s) \\
\hline
500 & 94 & \\
\hline
2000 & 181 & \\
\hline
\end{tabular}
%\caption{Table}
\label{t:1}
\end{center}
\end{table}
%

\end{document}


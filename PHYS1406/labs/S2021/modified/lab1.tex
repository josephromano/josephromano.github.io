\documentclass[11pt]{NSF}
\sloppy

\usepackage{latexsym}
\usepackage{graphicx}
\usepackage{draftcopy}
\usepackage{longtable}
\usepackage{hyperref}

% some definitions
 
% begin equation, itemize, etc.

\def\be{\begin{equation}}
\def\ee{\end{equation}}
\def\bi{\begin{itemize}}
\def\ei{\end{itemize}}
\def\ben{\begin{enumerate}}
\def\een{\end{enumerate}}
\def\i{\item{}}

%%%%%%%%%%%%%%%%%%%%%%%%%%%%%%%%%%%%%%%%%%%%%%%%%%%%%%%%%%%%%%
\begin{document}
     
\section{1. Simple Harmonic Motion (SHM)}

PURPOSE AND BACKGROUND

In order to understand sound and music, we need to understand periodic motion
and how it gives rise to sound. Periodic motion is any sort of movement that
repeats itself after an amount of time called the {\em period}. 
For example, a violin
string or the reed of a bassoon exhibits periodic motion when playing a
sustained tone. A grandfather clock exhibits periodic motion as the pendulum
swings back and forth, and so does a Ferris wheel that rotates at a constant
speed.

Simple harmonic motion (SHM) is the purest form of periodic motion. Two
conditions have to be met:

1) There exists a stable {\em equilibrium position}. 
If the system is at rest it will
stay at rest there. It will tend to return to that position if displaced from
it.

2) There exists a {\em restoring force} towards the equilibrium position. 
This force
is proportional to the amount of displacement from equilibrium. For example, if
a mass hanging from a spring originally is at rest and then pulled down a small
distance, the mass will oscillate up and down with SHM when let go. The spring
provides a restoring force to bring the mass back to the equilibrium position.
If the mass instead is pulled twice as far, the spring provides twice the force
to bring it back. In this manner, the system is {\em linear} 
and it is said to obey {\bf Hooke’s Law}.

Much of music and sound is generated from periodic vibrations of the air or
solid material in musical instruments. Examples are the vibrating strings of a
violin and the reeds of woodwind instruments. In practice, however, very few
musical tones come from pure SHM. That sound actually would be rather boring.
Instead, musical tones consist of a combination of harmonics---see a tone from
a plucked violin string in Figure 1. The lowest frequency corresponding to the
first peak is called the {\em fundamental frequency}. 
This is the only frequency present in SHM. 
The peaks at the higher frequencies in Figure 1 are the higher
harmonics or {\em overtones} that make up the tone. 
We shall discuss this in more detail in later laboratories.
%
\begin{figure}[hbtp]
\begin{center}
\includegraphics[width=.75\textwidth]{fig1_1}
\caption{Frequency spectrum of a plucked string showing the 
fundamental frequency and higher harmonics (overtones).}
\label{f:1}
\end{center}
\end{figure}
%

%\newpage
THEORY AND EXPERIMENT

A basic property of simple harmonic motion and any periodic motion is the
{\em period} $T$ and directly related to it the {\em frequency} $f$. 
The period is the time for one complete cycle of motion. 
The frequency is the number of cycles during that time. 
These two quantities are inversely related. 
For example, if it takes a mass on the spring two seconds to complete a 
cycle, then $T = {\rm period} = 2~{\rm s}$, 
and the frequency is one cycle per two seconds. As a formula we can
write 
\be
f = \frac{1}{T} = \frac{1}{2~{\rm s}} = 0.5~{\rm Hz}\,.
\ee
The unit of frequency is {\em Hertz}, abbreviated Hz, and is the number of cycles per second. 
A cycle can be one revolution, a completion of a periodic process, or one oscillation.

%%%%%%%%%%%%%%%%%%%%%%%%%%%%%%%%%%%%%%%%%%
\subsection{Pendulum}

For the first part of this lab, we make a pendulum using a 
string and either a lead (Pb) or aluminum (Al) ball as the
mass (see Figure~\ref{f:2}).
%
\begin{figure}[hbtp]
\begin{center}
\includegraphics[width=.5\textwidth]{fig1_2}
\caption{Pendulum (left) and spring (right).}
\label{f:2}
\end{center}
\end{figure}
%
The length of the string is initially $L = 20.0~{\rm cm}$ 
from the support to the center of the mass ball. 
To obtain the time it takes for one period of 
oscillation using a small displacement, we measure the 
time it takes for ten oscillations, and then divide 
the total time by 10 to get the average time for one period. 
We repeat this process three times and record the average 
of the three trials. 
We repeat the whole process for the other mass using 
the same length of $L = 20.0~{\rm cm}$, and 
then repeat for both the lead and aluminum balls for a
string of length $L=80.0~{\rm cm}$.
The results are given in the following table:
%
\begin{table}[hbtp]
\begin{center}
Pendulum Period $T$ (seconds)\\
\begin{tabular}{| c | c | c | c | c | }
\hline
&\multicolumn{2}{c}{$L=20.0~{\rm cm}$} \vrule
&\multicolumn{2}{c}{$L=80.0~{\rm cm}$} \vrule\\
\hline
Trial & \phantom{ }Pb\phantom{ } & Al & \phantom{ }Pb\phantom{ }\  & Al \\
\hline
\# 1 & 0.91 & 0.89 & 1.77 & 1.76 \\
\hline
\# 2 & 0.89 & 0.91 & 1.76 & 1.76 \\
\hline
\# 3 & 0.88 & 0.87 & 1.78 & 1.77 \\
\hline
avg  & 0.89 & 0.89 & 1.77 & 1.76 \\
\hline
\end{tabular}
%\caption{Table}
\label{t:1}
\end{center}
\end{table}

\subsubsection*{Questions}
%
\ben
\i Compare your results for the different masses and lengths. 
Which variable had an effect on the period? Which had no effect? 
(This might puzzle you and is different from the spring below.)

\i Note that the long pendulum was four times longer than the shorter one. 
Compare the periods of the longer and shorter pendulums.

\i If the length of the long pendulum were $9\times$ longer 
than the short pendulum (i.e., $L=180~{\rm cm}$), what do you think
the period would be?
\een

We know from basic mechanics that the period $T$ is proportional 
to the square root of the length $L$ of the pendulum according to the formula
\be
T=2\pi\sqrt{\frac{L}{g}}\propto \sqrt{L}\,.
\ee
So, if the long pendulum is four times as long as the short one, 
the period $T$ is only twice as long. 
(The quantity $g$ is the acceleration in Earth’s gravitational field, 
given by $g = 980~{\rm cm/s}^2$.)

%%%%%%%%%%%%%%%%%%%%%%%%%%%%%%%%%%%%%%%%%%
\subsection{Spring}

For spring oscillations, we start by attaching a 
$50~{\rm g}$ mass to the spring (see Figure~\ref{f:2}).
Pulling down slightly on the spring, we let go, and measure the time for ten 
oscillations, dividing by 10 again to obtain the average period.
We repeat this process three times and record the average of the three trials.
We then repeat with a mass of 200~g.
The results are given in the following table:
%
\begin{table}[hbtp]
\begin{center}
Spring Period $T$ (seconds)\\
\begin{tabular}{| c | c | c | }
\hline
Trial & $m=50~{\rm g}$ & $m=200~{\rm g}$ \\
\hline
\# 1 & 0.59 & 1.07 \\
\hline
\# 2 & 0.58 & 1.05  \\
\hline
\# 3 & 0.59 & 1.06 \\
\hline
avg  & 0.59 & 1.06 \\
\hline
\end{tabular}
%\caption{Table}
\label{t:2}
\end{center}
\end{table}
%
\subsubsection*{Questions}
%
\ben
\item 
How does the period $T$ of oscillation of a spring depend 
on the mass $m$ suspended from it? 
Write a simple proportionality to describe your observation.
\een

For a mass attached to a spring, the formula for the period of 
oscillation is
\be
T=2\pi\sqrt{\frac{m}{k}}\,,
\ee
where $m$ is the mass suspended from the spring and $k$ is the so-called
{\em spring constant}.

%%%%%%%%%%%%%%%%%%%%%%%%%%%%%%%%%%%%%%%%%%
\subsection{Vibrating Strings}

We can also study the simple harmonic motion of a vibrating string. 
Guitars and other string instruments have strings under tension. 
We use a so-called {\em sonometer},
which is an apparatus with strings whose tension can be adjusted. 
A string is fastened at one end to a tension meter and led over a 
bridge near the other end (see Figure~\ref{f:3}).
%
\begin{figure}[hbtp]
\begin{center}
\includegraphics[width=.75\textwidth]{fig1_3}
\caption{Sonometer with two vibrating strings.}
\label{f:3}
\end{center}
\end{figure}
%

The tension meter on the sonometer measures the equivalent of ``mass”. 
For instance, when the tension scale reads 6~kg, it is the equivalent 
of having the string attached to a 6~kg mass hanging over the edge 
of the table.

We use the microphone connected to the Mac mini-computer and 
record the frequency spectrum of the vibrating string with the 
spectrum analyzer in our ``Electroacoustics Toolbox" software. 
See Figure~\ref{f:1} again, which shows the fundamental frequency 
as well as the full frequency spectrum from the vibrating string. 
(In this figure, the sound intensity from the microphone is
displayed on the $y$-axis, versus the frequency on the $x$-axis.) 
Several frequencies are present when a string is plucked, namely 
the fundamental and the higher harmonics.

For two different tension masses, 500~g and 2000~g, we record the 
fundamental frequency from the spectrum analyzer mode on the computer.
We also calculate the period of the string oscillation using the 
formula $T= 1/f$.
The results are given in the following table:
%
\begin{table}[hbtp]
\begin{center}
String Frequency and Period\\
\begin{tabular}{| c | c | c | }
\hline
Tension mass $m$ (g) & Frequency $f$ (Hz) & Period $T=1/f$ (s) \\
\hline
500 & 94 & 0.011 \\
\hline
2000 & 181 & 0.0055 \\
\hline
\end{tabular}
%\caption{Table}
\label{t:1}
\end{center}
\end{table}
%
\subsubsection*{Questions}
%
\ben
\i How does the period change with increasing tension? 

\i How does the fundamental frequency (pitch) change as the tension 
increases?
\een

%%%%%%%%%%%%%%%%%%%%%%%%%%%%%%%%%%%%
\subsection{Pythagorean Intervals and String Division}

For the last part of the lab, we use the sonometer again and 
divide its strings with a wedge.
We move the wedge under the string, and pluck the two sections 
of the string, listening for when the two resulting tones sound consonant.
We do this for four different string divisions.
We write down the lengths of the two string sections, $L_1$ and 
$L_2$, and take their ratio, written as a decimal fraction with 
3 significant digits.
The results are given in the following table:
%
\begin{table}[h!btp]
\begin{center}
String Division and Ratio\\
\begin{tabular}{| c | c | c | }
\hline
$L_1$ (cm) & $L_2$ (cm) & ratio $L_2/L_1$ \\
\hline
400 & 400 & 1.00 \\
\hline
355 & 445 & 1.25 \\
\hline
319 & 481 & 1.51 \\
\hline
201 & 399 & 1.99 \\ 
\hline
\end{tabular}
%\caption{Table}
\label{t:4}
\end{center}
\end{table}
%

\subsubsection*{Questions}
%
\ben
\i What fraction of two integers are these ratios closest to,
and what musical intervals do these ratios correspond to?
(For example, 1.51 is close to 3/2, which corresponds to a musical 
fifth.)

\een

\end{document}


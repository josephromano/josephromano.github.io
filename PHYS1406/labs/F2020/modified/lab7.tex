\documentclass[11pt]{NSF}
\sloppy

\usepackage{latexsym}
\usepackage{graphicx}
\usepackage{draftcopy}
\usepackage{longtable}
\usepackage{hyperref}

% some definitions
 
% begin equation, itemize, etc.

\def\be{\begin{equation}}
\def\ee{\end{equation}}
\def\bi{\begin{itemize}}
\def\ei{\end{itemize}}
\def\ben{\begin{enumerate}}
\def\een{\end{enumerate}}
\def\i{\item{}}

%%%%%%%%%%%%%%%%%%%%%%%%%%%%%%%%%%%%%%%%%%%%%%%%%%%%%%%%%%%%%%
            
\begin{document}

\section{7. Sound Intensity, Hearing, Just Noticeable Difference (JND)} 

PURPOSE AND BACKGROUND

We can hear a wide range of sound intensities and frequencies. The
intensity between the threshold of hearing and the threshold of pain
varies by a factor of $10^{12}$, i.e., by 12 orders of magnitude or 120
decibel. The corresponding range in the amplitude of {\em air pressure 
fluctuations} is a factor of $10^6$. 
In view of this extreme range in sound intensity level, 
numbers are most conveniently expressed in power-of-ten
notation and with a decibel or dB-scale.  

Here we study sound intensity levels (SIL) and the frequency response of
the human ear. We also discuss ``just noticeable differences” (JND) in
intensity and frequency that the ear can discern.  

The ear is sensitive to a range in frequencies from about 20~Hz to 20~kHz.  
This audible range thus covers a factor of 1000 or $10^3$ in frequency, 
which is not nearly as large as the intensity range of $10^{12}$.  
In order to cover these large ranges, the ear
response is compressed or logarithmic with respect to both frequency
and sound intensity.

\subsection{Theory and Experiment}

The amplitude of a sound wave corresponds to air pressure fluctuations
(compressions and rarefactions of the air) in a longitudinal wave.

The {\em threshold of hearing} is a sound intensity at the ear of 
%
\be
I_0=1\times 10^{-12}~\textrm{W/m}^2
\quad{\rm at}\quad
f = 1000~\textrm{Hz}
\ee
%
This is the reference intensity for sound intensity measurements.
 
The {\em sound intensity level} (SIL) is defined by comparing
any intensity $I$ to the threshold of hearing $I_0$ 
according to 
%
\be
\rm{SIL} = 10\,\log({I}/{I_0})~{\rm dB}
\label{e:SIL}
\ee
%
where the logarithm is taken to the base 10.
The inverse equation is
%
\be
I=I_0\,10^{({\rm SIL}/10~{\rm dB})}
\label{e:inverse}
\ee
%
SIL is measured in {\em decibels} or dB.

For example, let the sound intensity in a room be 
$I = 1\times 10^{-6}~{\rm W/m}^2$. 
The SIL is then 
\be
\rm{SIL} 
= 10\,\log\left(\frac{1\times 10^{-6}}{1\times 10^{-12}}\right)~{\rm dB}
=10\, \log 10^6~{\rm dB} = 10\times 6~{\rm dB} = 60~{\rm dB}
\ee

The SIL also can be used to express a change in intensity from one 
value to another, without referring to the threshold of hearing $I_0$. 
We are then dealing with a {\em change} in SIL, denoted by 
$\Delta\,{\rm SIL}$, and not the SIL itself. 

For instance, if the intensity $I$ doubles to $2I$, we have
%
\be
\Delta\,{\rm SIL} = 10\,\log (2I/I)~{\rm dB}
= 10\,\log 2~{\rm dB} 
= 10\times 0.3~{\rm dB} 
= 3~{\rm dB}
\ee
Therefore, a doubling in intensity corresponds to an increase of 3 dB in the SIL.

\ben

\item
The sound intensity level in a typical environment is generally much higher 
than the threshold of hearing. 
For example, a typical sound intensity level that one might measure
for the background noise in the laboratory is ${\rm SIL}=70~{\rm dB}$.
What is the sound intensity $I$ of this background noise, expressed in 
units of ${\rm W/m}^2$? Hint: Use equation~(\ref{e:inverse}) to find $I$.

\item 
Assume that the sound intensity level for one student clapping in the
laboratory is ${\rm SIL}_1=75~{\rm dB}$. 
Calculate the theoretical increase in the sound intensity level,
$\Delta\,{\rm SIL}$, if the intensity $I_{10}$ for ten students clapping 
is ten times the intensity $I_1$ for one student clapping.

\item 
Using your answer to the previous question, what would you expect 
the sound intensity level ${\rm SIL}_{10}$ to be for 10 students 
clapping?
Hint: $\Delta\,{\rm SIL} = {\rm SIL}_{10}- {\rm SIL}_1$

\item 
At a frequency $f =1000~{\rm Hz}$, an intensity of $I = 1~{\rm W/m}^2$ 
becomes quite painful to the ear.
What is the sound intensity level in dB of a 1000~Hz sinusoidal 
tone at the threshold of pain?

\een

\subsection{Frequency Response of the Ear}

The ear can hear sound over a wide range of frequencies 
from about 20~Hz to 20~kHz. 
However, the perceived loudness varies quite dramatically
with frequency. The so-called {\em Fletcher-Munson curves} in
Figure~\ref{f:1} show
lines of equal perceived loudness. 
%
\begin{figure}[hbtp]
\begin{center}
\includegraphics[width=.9\textwidth]{fig7_1}
\caption{Fletcher-Munson curves of equal loudness. 
(From ``Physics of Sound” by R.A.~Berg and D.G.~Stork.)}
\label{f:1}
\end{center}
\end{figure}
%
The curve at the bottom marked 
``0~phons” represents the threshold of hearing, and the line marked 
``120~phons” represents the threshold of pain. 
Each curve has a ``phon” designation and
indicates equal perceived loudness as a function of frequency. 
The ``decibel” and ``phon” scales agree by convention at a frequency
of 1000~Hz (see Figure~\ref{f:1}). For example, if a loudspeaker produces a
1000~Hz tone with ${\rm SIL} = 60~{\rm dB}$ at your location, 
you perceive this sound intensity as a loudness  of 60~phon. 
If on the other hand the speaker produces
a tone at 100~Hz with the same ${\rm SIL} = 60~{\rm dB}$, 
you hear this as less loud than the 1000~Hz tone. 
In order for the two frequencies to sound
equally loud, the speaker must produce the 100 Hz tone at about 
${\rm SIL} = 70~{\rm dB}$ instead. 
Verify this on the curve labeled ``60 phons”.

You can also see from Figure~\ref{f:1} that the human ear is most sensitive to
sound around 4000 Hz, where the Fletcher-Munson curves dip lowest.
Therefore, if you follow a Fletcher-Munson curve from 4000 Hz to lower
frequencies, the sound intensity must be raised to be perceived as
equally loud. The same applies to higher frequencies above 4000 Hz.

\ben
\item
A loudspeaker produces a $1000~{\rm Hz}$ tone at an ${\rm
SIL}=40~{\rm dB}$.
From the Fletcher-Munson curve labeled ``40 phon" in Figure~\ref{f:1},
what SIL would be needed at $200~{\rm Hz}$ for the tone to be 
perceived as equally loud?

\item 
What SIL would be needed at 4000~Hz?

\een

\subsection{Just Noticeable Difference in Intensity}

The just noticeable difference (JND) in intensity is the smallest
change in SIL that the ear can discern. Usually a 25\% or 1~dB change in
intensity is detected. This depends somewhat on sound intensity and
frequency as can be seen in Figure 2. As the intensity or frequency
decreases, the ear becomes less sensitive to changes in intensity.
%
\begin{figure}[hbtp]
\begin{center}
\includegraphics[width=.7\textwidth]{fig7_2}
\caption{Just noticeable difference curves in intensity 
for 70~Hz, 200~Hz, and 1000~Hz sinusoidal tones. 
(From ``Physics of Sound” by R.A.~Berg and D.G.~Stork.)}
\label{f:2}
\end{center}
\end{figure}

\ben
\item 
Express a 25\% change in intensity $I$ as a change in dB. 
Hint: Calculate $\Delta\,{\rm SIL}$ for $I_2 = 1.25 I_1$.

\item
From Figure~\ref{f:2}, what is the value of the JND in 
intensity for the 1000 Hz curve at 80 dB?

\item
Suppose you compare the intensity of a square wave at 
$f=1000~{\rm Hz}$ to that of a sine wave at $f=1000~{\rm Hz}$.
For which do you get a smaller JND in intensity---i.e., 
for which can you hear smaller differences in SIL? 
Can you give a reason for this? 
(Hint: Consider the harmonics in the square wave.)
\een

\subsection{Just Noticeable Difference in Frequency}

In addition to being able to discern changes in sound intensity, 
we have an even better ability to notice changes in frequency. 
Figure~\ref{f:jnd_pitch} shows the JND in frequency, comparing 
it to the size of the critical bands on the cochlea.
%
\begin{figure}[hbtp]
\begin{center}
\includegraphics[width=.4\textwidth]{freqJNDa.jpg}
\caption{Just noticeable difference in frequency,
comparing it to the size of the critical bands on the cochlea.
(From ``Science of Sound," by Rossing, Moore, and Wheeler.)}
\label{f:jnd_pitch}
\end{center}
\end{figure}
%

To experimentally determine the just noticeable difference in
frequency, we play two pure tones one right after the other,
starting with the same frequency. 
We then increase one frequency slightly and keep playing both tones 
in succession.
The JND in frequency is when you can first discern a difference in 
the frequency (i.e., pitch) of the two tones.
One can express the JND in frequency as the either the 
difference between the two frequencies or as a percentage relating
the frequency difference to the starting frequency.

\ben
\item 
According to Figure~\ref{f:jnd_pitch} what is the JND in frequency
at a frequency of 200~Hz?
(Express the JND both as a difference of frequencies and as a 
percentage.)

\item What is the JND in frequency at a frequency of 2000~Hz?
\een

\subsection{Loudness in Sones}

The decibel values that we have discussed above are based on 
{\em objective} measurements of the sound intensity. 
There also exists a {\em subjective} ``sone" scale that tells 
what sounds ``twice as loud” to many persons. 
Such a ``twice as loud curve” is shown as a straight line in 
Figure~\ref{f:3}. 
On the sone scale, 1~sone corresponds to a loudness level of 
40~phon for a pure sine wave with $f = 1000~{\rm Hz}$. 
(Recall that for the special case of a pure tone at a frequency 
of 1000~Hz, the number of phon is the same 
as the number of dB.)
%
\begin{figure}[hbtp]
\begin{center}
\includegraphics[width=.6\textwidth]{fig7_3}
\caption{Sone scale, with ``twice as loud” meaning a doubling in the
sone number. The reference is 1~sone at a loudness level of 40~phon.
The phon scale is the same as the dB scale for a pure tone at 1000~Hz.}
\label{f:3}
\end{center}
\end{figure}

Figure~\ref{f:3} shows that, in order for sound to be perceived as twice as
loud, the sound intensity level must be higher by 10 phon (or 10~dB at
1000~Hz).
For example, for an increase in loudness from 1 sone to 2 sone, the 
sound intensity increases by 10~phon from 40~phon to 50~phon. 
Generally, for every increase in sound intensity by 10~phon, the sone 
number doubles. 
Example: For a doubling in loudness from 4 to 8 sone, the sound intensity 
increases from 60 to 70~phon.

\ben
\item
According to Figure~\ref{f:3}, what is the increase in phon for a 
doubling in loudness from 10 to 20~sone?

\item
How many times louder does a 90~phon tone sound than a 60~phon tone?

\een

Application:
The sone scale is used for specifying the loudness of fans and
appliances. For instance, quiet bathroom fans have a rating of 1 to 
2~sones; louder ones have a rating of 3 to 4~sones or more.

\end{document}


\documentclass[11pt]{NSF}
\sloppy

\usepackage{latexsym}
\usepackage{graphicx}
\usepackage{draftcopy}
\usepackage{longtable}
\usepackage{hyperref}

% some definitions
 
% begin equation, itemize, etc.

\def\be{\begin{equation}}
\def\ee{\end{equation}}
\def\bi{\begin{itemize}}
\def\ei{\end{itemize}}
\def\ben{\begin{enumerate}}
\def\een{\end{enumerate}}
\def\i{\item{}}

%%%%%%%%%%%%%%%%%%%%%%%%%%%%%%%%%%%%%%%%%%%%%%%%%%%%%%%%%%%%%%
            
\begin{document}

\begin{center}
PHYS 1406: ``Physics of sound and music"\\
Spring 2021
\end{center}

Welcome to ``Physics of sound and music"!

Please read the Course Syllabus and Course Calendar for the upcoming semester. 
These are posted on the course website:
\url{https://josephromano.github.io/PHYS1406/}
Relevant information will also be posted via Blackboard.

Due to covid-19 all lectures and labs will be held remotely via zoom.

In this class, we will discuss principles of sound, acoustics, music,
and musical instruments.  We will also have a few performances by
guest musicians %faculty from the TTU School of Music
and student volunteers from class. 

You do not have to have a background in physics and music for this
course. A high school background in science and math should be
sufficient.

The following textbook is required for the course:

{\em Physics of Sound and Music, PHYS 1406,
Course Guide and Laboratory Manual, Spring 2021}, by Prof Walter L. Borst.

This text contains the course topics, as well as the laboratory manual 
and sample homework, quizzes, and exams (from previous years).
The book will be available at The CopyOutlet, 2402 Broadway, Lubbock, 
Tel. 744-7772 starting Thursday Jan 21st.

Bring the Course Guide with you to lectures and to the labs.  
It is a good idea to read ahead in the book before
class, and to add your notes in the book.

%Please sign the attendance sheet on the front desk in the classroom 
%before each class. Come on time for signing in and finding a seat.

The lab sections of the course start on Thursday Feb 11th
(see the Course Calendar for details).
Due to covid-19 all of the labs will also be held remotely via zoom,
and have been modified accordingly, relative to what is in the Course Guide.
The modified labs are posted on the course website:
\url{https://josephromano.github.io/PHYS1406/}

I am looking forward to seeing you!

\end{document}


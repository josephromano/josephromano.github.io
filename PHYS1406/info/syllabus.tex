\documentclass[11pt]{NSF}
\sloppy

\usepackage{latexsym}
\usepackage{graphicx}
\usepackage{draftcopy}
\usepackage{longtable}
\usepackage{hyperref}

% some definitions
 
% begin equation, itemize, etc.

\def\be{\begin{equation}}
\def\ee{\end{equation}}
\def\bi{\begin{itemize}}
\def\ei{\end{itemize}}
\def\ben{\begin{enumerate}}
\def\een{\end{enumerate}}
\def\i{\item{}}

%%%%%%%%%%%%%%%%%%%%%%%%%%%%%%%%%%%%%%%%%%%%%%%%%%%%%%%%%%%%%%
            
\begin{document}

\begin{center}
COURSE SYLLABUS\\
PHYS 1406: ``Physics of sound and music"\\
Spring 2022
\end{center}

The course satisfies the 4-hour lab-science requirement in the Natural Science Core Curriculum.

{\bf Instructor:}
Joseph D. Romano\\
Office: Science Bldg, Room 14\\
Phone: 806-834-6522\\
E-mail: joseph.d.romano@ttu.edu

{\bf Office hours:}
Virtual (preferred) and in-person office hours by appointment.

{\bf Lectures:}
PHYS 1406: TTh 11:00am-12:20pm (SC 010)

{\bf Labs:} You must be signed up for one of the following lab
sections, which are held in SC 130:

PHYS 1406-501: Wed 4:00-5:50pm\\
PHYS 1406-502: Thu 4:00-5:50pm\\
PHYS 1406-503: Fri 4:00-5:50pm

Attendance and turned-in lab assignments are required.
A minimum lab grade of 75\% is needed to pass the course.
See Course Calendar for lab dates.

{\bf Required textbooks:}
\ben

\item 
{\em Physics of Sound and Music, PHYS 1406, 
Course Guide and Laboratory Manual}, by Prof Walter L.~Borst.
This book contains detailed information about the various course
topics (see below).
It also contains the lab manual and sample homework, quizzes, and exams
from previous years. 
Copies of the course guide will be available at the SUB campus bookstore.

\item
{\em How Music Works: The Science and Psychology of Beautiful Sounds, 
from Beethoven to the Beatles and Beyond Kindle Edition}, by John Powell.
Unfortunately, I didn't request it in time to have it available at 
the campus bookstore.  
But you can get it from e.g., amazon.com (a new paperback version is 
\$16.49, while the kindle version is only \$5.99).

\een

During the semester, you will be assigned readings from these texts, which you
will need to do before coming to class.  
I plan to run the class more as a discussion section than as a standard lecture.  
So for you to participate in class and profit from these discussions, 
{\em you will need to do the readings in advance}.

{\bf Course Objectives and Outcomes:}
Become familiar with some physical principles of sound, acoustics, and music. 
Enjoy music!
No prior knowledge of sound and music is required. 
A general high school background in science and mathematics is assumed.

{\bf Course Topics:} 
Distinguishing between general sound, music, and noise;
harmonics, quality of sound; sound analysis and synthesis;
the physics of oscillations and waves;
production and perception of sound (instruments, voice, hearing);
room acoustics and electrical recording and reproduction of sound.
{\em Also, class demonstrations and experiments, and performances by 
students and guest musicians.}

\newpage
{\bf Grading:}\\
10 laboratories: 20\%\\
Exams: $3\times 20\% = 60\%$\\
Final exam: 20\%

{\bf Grade Scale:} 100--A--86--B--72--C--58--D--44--F--0
 
{\bf Exams:}
There will be three exams plus a cumulative final.
Homework questions will not be graded, but you are encouraged
to do past homework problems and sample quizzes and exams
to help you do well on the exams.
You are allowed to bring one handwritten page of notes to the exams.
{\em A scientific calculator and ruler is required for the exams.}

{\bf Make-ups:}
Make-up exams and quizzes are not given without a {\em prior} valid excuse.
In case of a serious emergency, please discuss with me how a missed
grade can be made up.

{\bf Recommendation:} 
Study 3-5 hours each week outside of class for a likely grade of A or B.

{\bf REQUIRED SYLLABUS LANGUAGE:}

{\bf Vaccinations:} 
Texas Tech University strongly recommends students adhere to CDC guidelines on
COVID-19, including obtaining COVID-19 vaccinations. If you were unable to
obtain a vaccination prior to your arrival on campus, the COVID-19 vaccine is
available at Student Health Services by appointment. You can find additional
information about the vaccine and campus vaccine clincs at:
\url{https://www.ttu.edu/commitment/covid-19-vaccine/index.php}.

{\bf Face Covering Policy:} 
As of May 19, 2021, face coverings are optional in TTU facilities and
classrooms but, based on CDC guidelines, are recommended and welcome,
especially for those who have not been vaccinated for COVID-19 or who may have
susceptibilities to the virus. Face coverings are required in public
transportation (e.g., Citibus) and in the Student Health Clinic. 

{\bf Seating Charts and Social Distancing:}
There is not a mandated social distancing protocol for classroom seating, but
using a seating chart and taking attendance are recommended in support of
campus contact tracers if needed. Social distancing is recommended in rooms
that will enable it.

{\bf Illness-related Absences:} 
In general, student absences due to illness are to be considered as they were
prior to the pandemic, with consideration given to the fact that students who
are isolating with COVID-19 and students who are quarantining for symptoms or
direct exposure may have extended days of absence. Makeup opportunities should
be provided in a reasonable timeframe. Guidance for students is available at 
the COVID-19 page:
\url{https://www.depts.ttu.edu/communications/emergency/coronavirus/}.

{\bf In-Person Office Hours:} In-person office hours will be provided upon
request if a virtual (online) meeting is not feasible. For in-person 
office visits, I strongly recommend that students wear masks and
practice social distancing to help reduce the spread of COVID-19.  Outdoor
meeting venues (weather permitting) is also an option.

{\bf Personal Hygiene:} 
We all should continue to practice frequent hand washing, use hand sanitizers
after touching high-touch points (e.g., door handles, shared keyboards, etc.),
and cover faces when coughing or sneezing.

{\bf Potential Changes:} 
The University will continue to monitor CDC, State, and TTU System guidelines
in continuing to manage the campus implications of COVID-19. Any changes
affecting class policies or delivery modality will be in accordance with those
guidelines and announced as soon as possible. If Texas Tech University campus
operations are required to change because of health concerns related to the
COVID-19 pandemic, it is possible that this course will move to a fully online
delivery format. Should that be necessary, students will be advised of
technical and equipment requirements, such as web cam, microphone, and remote
proctoring software.

{\bf Academic Honesty (OP 34.12)}:
It is the aim of the faculty of Texas Tech University to foster a
spirit of complete honesty and high standard of integrity. The attempt
of students to present as their own any work not honestly performed is
regarded by the faculty and administration as a most serious offense
and renders the offenders liable to serious consequences, possibly
suspension. 

“Scholastic dishonesty” includes, but it not limited to,
cheating, plagiarism, collusion, falsifying academic records,
misrepresenting facts, and any act designed to give unfair academic
advantage to the student (such as, but not limited to, submission of
essentially the same written assignment for two courses without the
prior permission of the instructor) or the attempt to commit such an
act.  

The full policy is available at
\url{http://www.depts.ttu.edu/opmanual/OP34.12.pdf} 

{\bf Special Accommodation for Students with Disabilities 
(OP 34.22):} 
Any student who, because of
a disability, may require special arrangements in order to meet the
course requirements should contact the instructor as soon as possible
to make any necessary arrangements. Students should present
appropriate verification from Student Disability Services during the
instructor's office hours. Please note: instructors are not allowed to
provide classroom accommodations to a student until appropriate
verification from Student Disability Services has been provided. For
additional information, please contact Student Disability Services in
West Hall or call 806-742-2405.  

The full policy is available at
\url{http://www.depts.ttu.edu/opmanual/OP34.22.pdf}

{\bf Student Absence for
Observance of a Religious Holy Day (OP 34.19):}

\ben
\i “Religious holy day”
means a holy day observed by a religion whose places of worship are
exempt from property taxation under Texas Tax Code §11.20.  

\i A student who intends to observe a religious holy day should make that
intention known in writing to the instructor prior to the absence. A
student who is absent from classes for the observance of a religious
holy day shall be allowed to take an examination or complete an
assignment scheduled for that day within a reasonable time after the
absence.  

\i A student who is excused under section 2 may not be
penalized for the absence; however, the instructor may respond
appropriately if the student fails to complete the assignment
satisfactorily.  
\een

The full policy is available at
\url{http://www.depts.ttu.edu/opmanual/OP34.19.pdf}

{\bf TTU Resources for Discrimination, Harassment, and Sexual
Violence:}
Texas Tech University is committed to providing and strengthening an
educational, working, and living environment where students, faculty,
staff, and visitors are free from gender and/or sex discrimination of
any kind. Sexual assault, discrimination, harassment, and other Title
IX violations are not tolerated by the University. 

Report any
incidents to the Office of Student Rights \& Resolution, 806-742-SAFE
(7233) or file a report online at 
\url{http://www.depts.ttu.edu/titleix/}. 

Faculty and staff
members at TTU are committed to connecting you to resources on campus.
Some of these available resources are: 

\bi
\i {\bf TTU Student Counseling Center}\\
Phone: 806-742-3674\\
Website: \url{https://www.depts.ttu.edu/scc/}\\
(Provides confidential support on campus.) 

\i {\bf TTU 24-hour Crisis Helpline}\\
Phone: 806-742-5555\\
(Assists students who are experiencing a mental health or
interpersonal violence crisis. If you call the helpline, you will
speak with a mental health counselor.) 

\i {\bf Voice of Hope Lubbock Rape Crisis Center}\\
Phone: 806-742-7273\\
Website: \url{http:voiceofhopelubbock.org}\\
(24-hour hotline
that provides support for survivors of sexual violence.) 

\i {\bf The Risk, Intervention, Safety and Education (RISE) Office}\\
Phone: 806-742-2110\\
Website: \url{http://www.depts.ttu.edu/rise/}\\
(Provides a range of resources and support options
focused on prevention, education, and student wellness.) 

\i {\bf Texas Tech Police Department}\\
Phone: 806-742-3931\\ 
Website: \url{http://www.depts.ttu.edu/ttpd/}\\ 
(To report criminal activity that occurs on or near Texas Tech campus.)

\i {\bf LGBTQIA:}
Within the Center for Campus Life, the Office of LGBTQIA
serves the Texas Tech community through facilitation and leadership of
programming and advocacy efforts. This work is aimed at strengthening
the lesbian, gay, bisexual, transgender, queer, intersex, and asexual
(LGBTQIA) community and sustaining an inclusive campus that welcomes
people of all sexual orientations, gender identities, and gender
expressions.

\ei
\end{document}


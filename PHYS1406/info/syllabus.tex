\documentclass[11pt]{NSF}
\sloppy

\usepackage{latexsym}
\usepackage{graphicx}
\usepackage{draftcopy}
\usepackage{longtable}
\usepackage{hyperref}

% some definitions
 
% begin equation, itemize, etc.

\def\be{\begin{equation}}
\def\ee{\end{equation}}
\def\bi{\begin{itemize}}
\def\ei{\end{itemize}}
\def\ben{\begin{enumerate}}
\def\een{\end{enumerate}}
\def\i{\item{}}

%%%%%%%%%%%%%%%%%%%%%%%%%%%%%%%%%%%%%%%%%%%%%%%%%%%%%%%%%%%%%%
            
\begin{document}

\begin{center}
COURSE SYLLABUS\\
PHYS 1406: ``Physics of sound and music"\\
Spring 2023
\end{center}

The course satisfies the 4-hour lab-science requirement in the Natural Science Core Curriculum.

{\bf Instructor:}
Joseph D. Romano\\
Office: Science Bldg, Room 14\\
Phone: 806-834-6522\\
E-mail: joseph.d.romano@ttu.edu

{\bf Office hours:}
Zoom (preferred) and in-person office hours by appointment.

{\bf Lectures:}
PHYS 1406: TTh 11:00am-12:20pm (SC 010)

{\bf Labs:} You must be signed up for one of the following lab
sections, which are held in SC 130:

PHYS 1406-501: Wed 4:00-5:50pm\\
PHYS 1406-502: Thu 4:00-5:50pm\\
PHYS 1406-503: Fri 4:00-5:50pm

Attendance and turned-in lab assignments are required.
\emph{A minimum lab grade of 75\% is needed to pass the course.}
See Course Calendar for lab dates.

{\bf Required textbook:}
{\em How Music Works: The Science and Psychology of Beautiful Sounds, 
from Beethoven to the Beatles and Beyond}, by John Powell.

{\bf Course Objectives and Outcomes:}
Become familiar with some physical principles of sound, acoustics, and music. 
Enjoy music!
No prior knowledge of sound and music is required. 
A general high school background in science and mathematics is assumed.

{\bf Course Topics:} 
Distinguishing between general sound, music, and noise;
harmonics, quality of sound; sound analysis and synthesis;
the physics of oscillations and waves;
production and perception of sound (instruments, voice, hearing);
room acoustics and electrical recording and reproduction of sound.
Elementary music theory and tuning systems.
{\em Also, class demonstrations and experiments, and performances by 
students and guest musicians.}

{\bf Course website:}
\url{https://josephromano.github.io/PHYS1406/} with announcements
circulated via Blackboard \url{https://ttu.blackboard.com/}.

{\bf Class attendance:}
Please sign the attendance sheet each day upon entering 
the classroom.
You are only allowed four unexcused absences.
After that, you will lose points from your final grade.
For example, five unexcused absences means $-1$ point, six 
unexcused absences means $-2$ points, etc.
 
{\bf Class presentations:}
You will be asked to do one short (10 minute) presentation 
during class this semester, together with another student.
The topic should be related to sound or music, but it doesn't
have to be directly connected to the material we are learning
at that time.

{\bf Exams:}
There will be two midterm exams plus a cumulative final.
You are allowed to bring one handwritten page of notes to the exams.
{\em A scientific calculator and ruler is required for the exams.}
Make-up exams are not given without a {\em prior} valid excuse.
In case of a serious emergency, please discuss with me how a missed
grade can be made up.

{\bf Homework assignments:}
You are asked to read material from both ``How music works" and
the online lecture notes before coming to class.
Homework problems will not be given or graded, but you are
encouraged to do the homework problems at the end of the lecture notes.
Doing so will improve your performance on the exams.

{\bf Grading:}\\
10 laboratories: 20\%\\
Class attendance: 10\%\\
Class presentation: 10\%\\
Midterm x 2: 40\%\\
Final: 20\%

{\bf Grade Scale:} 100--A--86--B--72--C--58--D--44--F--0
 
{\bf REQUIRED AND RECOMMENDED SYLLABUS LANGUAGE:}

\url{https://www.depts.ttu.edu/tlpdc/RequiredSyllabusStatements.php}

\url{https://www.depts.ttu.edu/tlpdc/RecommendedSyllabusStatements.php}


\end{document}


\documentclass[11pt]{NSF}
\sloppy

\usepackage{latexsym}
\usepackage{graphicx}
\usepackage{draftcopy}
\usepackage{longtable}
\usepackage{hyperref}

% some definitions
 
% begin equation, itemize, etc.

\def\be{\begin{equation}}
\def\ee{\end{equation}}
\def\bi{\begin{itemize}}
\def\ei{\end{itemize}}
\def\ben{\begin{enumerate}}
\def\een{\end{enumerate}}
\def\i{\item{}}

%%%%%%%%%%%%%%%%%%%%%%%%%%%%%%%%%%%%%%%%%%%%%%%%%%%%%%%%%%%%%%
            
\begin{document}

\begin{center}
COURSE SYLLABUS\\
PHYS 1406: ``Physics of sound and music"\\
Spring 2021
\end{center}

The course satisfies the 4-hour lab-science requirement in the Natural Science Core Curriculum.

{\bf Instructor:}
Joseph D. Romano\\
Office: Science Bldg, Room 14\\
Phone: 806-834-6522\\
E-mail: joseph.d.romano@ttu.edu

{\bf Office hours (remote):}
TTh 2:00pm-3:30pm, and by appointment

{\bf Lectures:}
PHYS 1406: TTh 12:30-1:50pm (via zoom)

{\bf Labs:} You must be signed up for one of the following lab
sections

PHYS 1406-501: Thu 4:00-5:50pm\\
PHYS 1406-502: Thu 10:00-11:50am\\
PHYS 1406-503: Fri 10:00-11:50am

All lab sections will be held remotely (via zoom).

Attendance and turned in lab assignment are required.
A minimum lab grade of 75\% is needed for passing the course.
See Course Calendar for lab dates.

{\bf Required textbook:}
{\em Physics of Sound and Music, PHYS 1406, 
Course Guide and Laboratory Manual, Spring 2021}, by Prof Walter
L.~Borst

This book contains the course topics.
It also contains the lab manual and sample homework, quizzes, and exams
from previous years. 
Bring the book with you to class; add your notes in the book.
{\em The book is available at The CopyOutlet, 2402 Broadway, Lubbock,
Tel. 744-7772}.

{\bf Course Objectives and Outcomes:}
Become familiar with some physical principles of sound, acoustics, and music. 
Enjoy music!
No prior knowledge of sound and music is required. 
A general high school background in science and mathematics is assumed.

{\bf Course Topics:} 
Distinguishing between general sound, music, and noise;
harmonics, quality of sound; sound analysis and synthesis;
the physics of oscillations and waves;
production and perception of sound (instruments, voice, hearing);
room acoustics and electrical recording and reproduction of sound.
{\em Also, class demonstrations and experiments, and performances by 
students and guest musicians.}

{\bf Grading:}\\
10 laboratories: 20\%\\
Exams: $3\times 20\% = 60\%$\\
Final exam: 20\%

{\bf Grade Scale:} 100--A--86--B--72--C--58--D--44--F--0
 
{\bf Exams:}
There will be three exams plus a cumulative trial.
Homework questions will not be graded, but you are encouraged
to do past homework problems and sample quizzes and exams
to help you do well on the exams.
You are allowed to bring one handwritten page of notes to the exams.
{\em A scientific calculator and ruler is required for the exams.}

{\bf Make-ups:}
Make-up exams and quizzes are not given without a {\em prior} valid excuse.
In case of a serious emergency, please discuss with me how a missed
grade can be made up.

{\bf Recommendation:} 
Study 3-5 hours each week outside of class for a likely grade of A or B.

\newpage
{\bf REQUIRED SYLLABUS LANGUAGE:}

{\bf Academic Honesty (OP 34.12)}:
It is the aim of the faculty of Texas Tech University to foster a
spirit of complete honesty and high standard of integrity. The attempt
of students to present as their own any work not honestly performed is
regarded by the faculty and administration as a most serious offense
and renders the offenders liable to serious consequences, possibly
suspension. 

“Scholastic dishonesty” includes, but it not limited to,
cheating, plagiarism, collusion, falsifying academic records,
misrepresenting facts, and any act designed to give unfair academic
advantage to the student (such as, but not limited to, submission of
essentially the same written assignment for two courses without the
prior permission of the instructor) or the attempt to commit such an
act.  

The full policy is available at
\url{http://www.depts.ttu.edu/opmanual/OP34.12.pdf} 

{\bf Special Accommodation for Students with Disabilities 
(OP 34.22):} 
Any student who, because of
a disability, may require special arrangements in order to meet the
course requirements should contact the instructor as soon as possible
to make any necessary arrangements. Students should present
appropriate verification from Student Disability Services during the
instructor's office hours. Please note: instructors are not allowed to
provide classroom accommodations to a student until appropriate
verification from Student Disability Services has been provided. For
additional information, please contact Student Disability Services in
West Hall or call 806-742-2405.  

The full policy is available at
\url{http://www.depts.ttu.edu/opmanual/OP34.22.pdf}

{\bf Student Absence for
Observance of a Religious Holy Day (OP 34.19):}

\ben
\i “Religious holy day”
means a holy day observed by a religion whose places of worship are
exempt from property taxation under Texas Tax Code §11.20.  

\i A student who intends to observe a religious holy day should make that
intention known in writing to the instructor prior to the absence. A
student who is absent from classes for the observance of a religious
holy day shall be allowed to take an examination or complete an
assignment scheduled for that day within a reasonable time after the
absence.  

\i A student who is excused under section 2 may not be
penalized for the absence; however, the instructor may respond
appropriately if the student fails to complete the assignment
satisfactorily.  
\een

The full policy is available at
\url{http://www.depts.ttu.edu/opmanual/OP34.19.pdf}

{\bf TTU Resources for Discrimination, Harassment, and Sexual
Violence:}
Texas Tech University is committed to providing and strengthening an
educational, working, and living environment where students, faculty,
staff, and visitors are free from gender and/or sex discrimination of
any kind. Sexual assault, discrimination, harassment, and other Title
IX violations are not tolerated by the University. 

Report any
incidents to the Office of Student Rights \& Resolution, 806-742-SAFE
(7233) or file a report online at 
\url{http://www.depts.ttu.edu/titleix/}. 

Faculty and staff
members at TTU are committed to connecting you to resources on campus.
Some of these available resources are: 

\bi
\i {\bf TTU Student Counseling Center}\\
Phone: 806-742-3674\\
Website: \url{https://www.depts.ttu.edu/scc/}\\
(Provides confidential support on campus.) 

\i {\bf TTU 24-hour Crisis Helpline}\\
Phone: 806-742-5555\\
(Assists students who are experiencing a mental health or
interpersonal violence crisis. If you call the helpline, you will
speak with a mental health counselor.) 

\i {\bf Voice of Hope Lubbock Rape Crisis Center}\\
Phone: 806-742-7273\\
Website: \url{http:voiceofhopelubbock.org}\\
(24-hour hotline
that provides support for survivors of sexual violence.) 

\i {\bf The Risk, Intervention, Safety and Education (RISE) Office}\\
Phone: 806-742-2110\\
Website: \url{http://www.depts.ttu.edu/rise/}\\
(Provides a range of resources and support options
focused on prevention, education, and student wellness.) 

\i {\bf Texas Tech Police Department}\\
Phone: 806-742-3931\\ 
Website: \url{http://www.depts.ttu.edu/ttpd/}\\ 
(To report criminal activity that occurs on or near Texas Tech campus.)

\i {\bf LGBTQIA:}
Within the Center for Campus Life, the Office of LGBTQIA
serves the Texas Tech community through facilitation and leadership of
programming and advocacy efforts. This work is aimed at strengthening
the lesbian, gay, bisexual, transgender, queer, intersex, and asexual
(LGBTQIA) community and sustaining an inclusive campus that welcomes
people of all sexual orientations, gender identities, and gender
expressions.

\ei
\end{document}


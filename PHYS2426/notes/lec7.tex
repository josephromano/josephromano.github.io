\section{Capacitance}

%%%%%%%%%%%%%%%%%%%%%%%%%%%%%%%%%%%%%%%%%%%%%%%%%%%
\subsection{Capacitors}

\bi

\i A capacitor is a circuit element whose primary
function is to {\em store charge}.

\i Given two conductors (of any shape and size)
with charges $\pm Q$, respectively, the 
capacitance $C$ is defined by
%
\be
C \equiv Q/\Delta V
\ee
%
where $\Delta V$ is the potential difference between
the positively and negatively-charged conductors.

\i The value of $C$ is independent of $\Delta V$,
depending only on the geometrical arrangement of 
the conductors.

\i Examples:

- Parallel-plate capacitor, area $A$, separation $d$: 
$C = \epsilon_0 A/d$.

- Two concentric cylinders, length $L$, inner and 
outer radii $a$, $b$:
$C = 2\pi\epsilon_0 L/\ln(b/a)$.

- Two concentric spheres, inner and 
outer radii $a$, $b$:
$C = 4\pi\epsilon_0 ab/(b-a)$.

- Single, isolated sphere:
$C = 4\pi\epsilon_0 a$.

\ei

%%%%%%%%%%%%%%%%%%%%%%%%%%%%%%%%%%%%%%%%%%%%%%%%%%%
\subsection{Series and parallel combinations}
\bi

\i One can connect capacitors in {\em parallel} or 
{\em series}, or combinations of the two.
Such a combination of capacitors can be reduced to a 
single {\em equivalent} capacitance $C_{\rm eq}$.

\i For a parallel combination of $C_1$ and $C_2$:
\be
C_{\rm eq} = C_1 + C_2
\ee
The proof of this relation uses the fact that the 
potential differences are equal,
$\Delta V_1 = \Delta V_2 = \Delta V$,
while the charges add
$Q=Q_1+Q_2$.

\i For a series combination of $C_1$ and $C_2$:
\be
\frac{1}{C_{\rm eq}} = \frac{1}{C_1} + \frac{1}{C_2}
\ee
The proof of this relation uses the fact that the 
potential differences add,
$\Delta V = \Delta V_1+\Delta V_2$, while the charges 
are equal, $Q=Q_1=Q_2$.

\ei

%%%%%%%%%%%%%%%%%%%%%%%%%%%%%%%%%%%%%%%%%%%%%%%%%%%
\subsection{Energy stored in a capacitor}

\bi

\i To charge up a capacitor, one needs to do 
work against the electric field associated with 
the charge already deposited on the plates.

\i If $q$ is the current charge on the 
capacitor, then the associated potential 
difference of the plates is $q/C$.
To move additional charge $dq$ against 
that potential requires work $dW = dq\, q/C$.

\i The total work $W$ is given by the integral
%
\be
W = \int_0^Q dq\, \frac{q}{C} = 
\frac{1}{2}\frac{Q^2}{C}
\ee

\i This work is the electrostatic potential
energy $U_E$ stored in the capacitor, and it
can be written in several equivalent ways:
%
\be
U_E 
= \frac{1}{2}\frac{Q^2}{C} 
= \frac{1}{2}C(\Delta V)^2
= \frac{1}{2}Q\Delta V
\ee

\i For a parallel plate capacitor, which has
$C=\epsilon_0 A/d$ and $\Delta V= Ed$, one can show 
that the (volume) energy density 
$u_E\equiv {U_E}/{\rm volume}$ is given by
$u_E = \frac{1}{2}\epsilon_0 E^2$.
This expression is actually valid for {\em any}
electrostatic field, not just for the field 
between the plates of a parallel-plate capacitor.
\ei


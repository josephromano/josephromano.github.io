\section{Induced electric fields;
motors and generators}

\subsection{Induced electric fields}
\bi

\i Recall that emf is defined by
${\cal E} \equiv \oint_C\vec f_{\rm s}\cdot
d\vec s$, where $f_{\rm s}$ is the force
per unit charge exerted by the source,
which can be chemical reactions in a 
battery, a solar cell, etc.

\i For motional emf, $\vec f_{\rm s} =
\vec f_{\rm mag}$, the magnetic force per 
unit charge on a moving charge.

\i For a changing magnetic field
$\vec f_{\rm s} =\vec E$, an induced electric
field that acts on charges according to 
$\vec F = q\vec E$, just like electric fields
produced by electric charges.
But this induced electric field is
{\em non-conservative} since 
$\oint_C \vec E\cdot d\vec s\ne 0$.

\i So electric fields can be produced in 
two ways:
(i) by electric charges (which gives rise to 
a conservative field), or 
(ii) by changing magnetic fields 
(which gives rise to a non-conservative field).

\ei

%%%%%%%%%%%%%%%%%%%%%%%%%%%%%%%%%%%%%%%%
\subsection{Generators and motors}
\bi

\i A generator converts mechanical energy
into electrical energy.
A motor is a generator run in reverse;
it converts electrical energy into mechanical
energy.

\i AC (alternating current) generator:
Rotate a loop of wire in a permanent magnetic
field and get an AC current out.
(Mathematically: $\Phi_B = BA\cos\omega t$
implies ${\cal E} = -N d\Phi_B/dt
= NBA\omega\sin\omega t$.)

\i An AC generator uses a {\em slip ring} to 
connect the ends of the rotating loop to 
brushes connected to an external circuit.
A DC generator uses a {\em split ring}
instead of a slip ring; the split ring
switches the polarity every half cycle so that 
${\cal E} = NBA\omega|\sin\omega t|$.

\i A motor is constructed by sending a 
current through a coil of wire in a 
permanent magnetic field.
The current in the coil produces a magnetic
field that interacts with the external magnetic 
field---it feels a torque in the external 
field---and the coil 
rotates on account of that torque.

\i Demos:
(i) Hand-powered AC generator;
(ii) DC motor constructed from electromagnets;
(iii) Do-it-yourself DC motor constructed from 
a D-cell battery, a small magnet, paper 
clips, and a coil of (magnet) wire stripped 
on one side.

\ei

%%%%%%%%%%%%%%%%%%%%%%%%%%%%%%%%%%%%%%%%%%%%%
\subsection{Back-emf}

\bi

\i When a motor is turned on and starts 
running, there is a {\em back emf}
(and current) induced in the coil due to its 
motion, which opposes the applied emf (and
current) that originally got it moving.

\i The back emf can be a sizable fraction 
of the applied emf when the motor is running
at full speed, which decreases the current to the
load, $I = ({\cal E}-{\cal E}_{\rm back})/R$.

\i For a heavy load, the motor turns more slowly.
The back emf is then proportionally smaller, 
leading to a larger current to the load.
If the motor ``jams", the back emf is zero.
This leads to a potentially large current that 
can ``burn out" the motor.

\ei


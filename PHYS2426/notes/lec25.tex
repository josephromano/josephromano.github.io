\section{Image formation by flat and
spherical mirrors}

%%%%%%%%%%%%%%%%%%%%%%%%%%%%%%%%%%%%%%%%%%%%%%%%
\subsection{Flat mirrors}
\bi

\i Using angle of incidence equals angle of 
reflection, one can show that for a flat mirror,
a virtual image is produced, which is upright,
has the same size as the object ($h'=h$), and 
is as far behind the mirror as the object is 
in front of it ($q=-p$).

\i NOTE:

(i) The apparent left-right reversal (right hand
of a person seen as the left hand of the image)
is actually a front-back reversal.

(ii) You only need a mirror half your
height to see your full body, and it doesn't
depend on how far away you are from the mirror.

(iii) a double-mirror from from two flat mirrors
intersecting at a right angle allows you to 
see yourself as others do (since the doubly-reflected
image doesn't have any apparent left-right reversal).

\ei
%%%%%%%%%%%%%%%%%%%%%%%%%%%%%%%%%%%%%%%%%%%%%%%%
\subsection{Spherical mirrors}
\bi

\i Spherical mirrors are formed from spherical 
reflecting surfaces having radius of curvature $R$.

\i They can either be {\em concave}, like a shaving
mirror; or {\em convex}, like a passenger-side mirror
on a car.

\i Tracing rays of light from an object to the
image using the law of reflection and simple geometry,
one can show that
%
\be
\frac{1}{p}+ \frac{1}{q} = \frac{1}{f}\,,
\qquad
\frac{1}{f} \equiv \frac{2}{R} 
\ee
%
where $p$ and $q$ are the object and image location, 
respectively.

\i The second equation relates the radius of curvature
of the mirror to the {\em focal length} $f$.
In the {\em paraxial} approximation (where rays of 
light don't deviate much from the main axis of the
mirror), parallel rays of light converge
to or diverge from a single focal point $F$.

\i For spherical mirrors, parallel rays of light far 
from the main axis don't all meet at a single point.  
This is called {\em spherical aberration}.

\i For parabolic mirrors, {\em all} parallel rays meet
at a single point. 

\i The sign conventions for the mirror equation are:

$p>0$ if the object is in front of the reflecting 
surface of the mirror;

$q>0$ if the image is in front of the mirror;

$h'>0$ if the image is upright; 

$f, R>0$ if the mirror is concave.

\ei

%%%%%%%%%%%%%%%%%%%%%%%%%%%%%%%%%%%%%%%%%%%%%%%%
\subsection{Ray tracing for concave and convex mirrors}
\bi

\i For a concave mirror, parallel rays of light meet at the 
focal point in front of the mirror.

\i For a convex mirror, parallel rays of light
diverge from the focal point in back of the mirror.

\i By tracing rays of light reflecting off of a mirror,
one can show that:

(i) a concave mirror produces both
real and virtual images (if $p>f$ and $p<f$, respectively).

(ii) a convex mirror {\em always} produces virtual
images that are upright and reduced in size.

\i If $p=f$ for a concave mirror, no image is formed
(formally, $q\rightarrow\infty$).

\i Note that real images are always inverted (relative
to the orientation of the object), while virtual images 
are always upright.

\ei

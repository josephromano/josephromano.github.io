\section{Inductance}

\subsection{Self inductance}
\bi

\i When a circuit is first connected to a battery 
with emf ${\cal E}$, the changing magnetic flux 
through the circuit (due to the increasing current)
induces a back emf
%
\be
{\cal E}_L \equiv -L\,\frac{di}{dt}
\ee
which opposes the change in the flux.
The proportionality constant $L$ is called the 
{\em inductance} (or {\em self inductance}) of
the circuit; it has units of Henries (${\rm H}$), 
where $1~{\rm H} = 1~{\rm V}\cdot{\rm s}/{\rm A}$.

\i Since we can also write ${\cal E}=-N\,{d\Phi_B}/dt$,
we have $L\,{di}/{dt} = N\,{d\Phi_B}/{dt}$, or 
$L = N\Phi_B/i$, which depends only on the
geometry of the circuit.

\i Inductance corresponds to the {\em inertia} 
(or opposition) of
a circuit to changes in its current.
Inductance is analogous to mass, which is inertia 
to changes in the velocity of an object 
$\vec F = m\,d\vec v/dt$.

\i Examples:

(i) solenoid (length $l$, cross-sectional area $A$, number of
turns per length $n$):
$L = \mu_0 n^2 Al$.

(ii) coaxial cable (length $l$, inner radius $a$, outer radius $b$):
$L/l = (\mu_0/2\pi) \ln(b/a)$.

\ei
%%%%%%%%%%%%%%%%%%%%%%%%%%%%%%%%%%%
\subsection{Mutual inductance}
\bi

\i Consider two loops of wire with numbers of loops
$N_1$, $N_2$ and currents $i_1$, $i_2$.
Then a changing current $i_1$ in loop 1 induces a changing
flux $\Phi_{12}$ in loop 2, leading to 
${\cal E}_2 \equiv - M_{12}\, di_1/dt$,
where $M_{12} = N_2 \Phi_{12}/i_1$.
(Similarly, ${\cal E}_1 = - M_{21}\, di_2/dt$,
where $M_{21} = N_1 \Phi_{21}/i_2$.)

\i It turns out that $M_{12}=M_{21}\equiv M$,
called the {\em mutual inductance} of the two 
loops.  
The mutual inductance depends only the geometry
(size, relative orientation, etc.) 
of the two loops.

\i Examples

(i) Two coaxial solenoids with the longer solenoid
having length $l$, cross-sectional area $A$,
and number of turns $N_1$,
and the shorter solenoid having 
number of turns $N_2$:
$M = \mu_0 N_1 N_2 A/l$.
(For this problem, it is simplest to calculate
$M_{12}$ since the flux through solenoid 2 due
to solenoid 1 is uniform;
this is not so for the flux through solenoid 1
due to solenoid 2.)

(ii) Transformer with number of turns $N_1$,
$N_2$ for the input and output:
${\cal E}_1/N_1 = {\cal E}_2/N_2$.
(Thus a transformer {\em steps-up} the voltage
if $N_2>N_1$.)

\ei
%%%%%%%%%%%%%%%%%%%%%%%%%%%%%%%%%%%
\subsection{$RL$ circuits}

\bi
\i Consider a battery with emf ${\cal E}$ connected to 
a resistor $R$ and inductor $L$ in series.

\i Then applying the loop rule to the potential differences
around the circuit leads to a differential equation
${\cal E}-iR - L\,di/dt=0$, which has solution
%
\be
i(t) = i_{\rm max}\left(1- e^{-t/\tau}\right)
\ee
%
where $i_{\rm max} = {\cal E}/R$ and 
$\tau\equiv L/R$ is the time constant for the 
$LR$ circuit.

\i The current doesn't increase instantaneously
due to the inductance in the circuit.

\i After obtaining maximum current through the inductor,
one can remove the battery leaving just the resistor and
the inductor.
Then the current will decrease exponentially:
\be
i(t) = i_{\rm max}\,e^{-t/\tau}
\ee

\i NOTE: If you pull out the power chord of a running 
vacuum cleaner you will see a spark at the outlet, 
as the inductance of the motor is trying to keep the 
current following, in accordance with Lenz's law.

\ei


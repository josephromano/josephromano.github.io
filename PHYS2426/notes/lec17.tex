\section{Faraday's law of induction}

\subsection{Faraday's law of induction}
\bi
\i Three demonstrations:

(i) Move a permanent magnet toward or away from
a coil of wire.
A current is induced in the coil when the 
magnet is moving (CW or CCW current  depending
on whether the magnet is moving toward or
away from the wire, consistent with Lenz's law).

(ii) Move a coil of wire toward or away from 
a permanent magnet.
A current is induced in the moving coil 
(CW or CCW current  depending
on whether the coil is moving toward or
away from the magnet, consistent with Lenz's law).

(iii) Link two coils of wire by an iron core, 
with one of the coils connected to a battery and 
a switch.
Close the switch, and a current is induced in
the other coil.
Open the switch, and a current is induced in the
other coil in the opposite direction. 

\i Demos (i) and (ii) show that only {\em relative}
motion is important.
Demo (iii) shows that you can get an induced 
electric field in the absence of physical motion.

\i Bascially, the above demos illustrate that a
{\em changing magnetic field produces
(induces) an electric field}.

\i Mathematically, the induced emf in the coil
is given by
%
\be
{\cal E} = - N\frac{d\Phi_B}{dt},
\qquad
\Phi_B \equiv \int_S \vec B\cdot \hat n\, dA
\ee
%
where $N$ is the number of loops and $\Phi_B$ is
magnetic flux passing through any surface $S$
spanning a loop.
The minus sign comes from Lenz's law (see below).

\i Applications:
(i) GFCI (ground-fault circuit interruptor)
outlets;
(ii) pick-up of an electric guitar.

\ei

%%%%%%%%%%%%%%%%%%%%%%%%%%%%%%%%%%%%%%%%%%%%%%%%%
\subsection{Motional EMF}
\bi

\i Demonstration (ii) is an example of
{\em motional emf}; namely, the charge
carriers in a moving conductor feel a magnetic force.

\i The induced emf for this case is given by 
${\cal E} = \oint_C\vec f_{\rm mag}\cdot d\vec s$,
where $\vec f_{\rm mag}\equiv \vec v\cross\vec B$
is the magnetic force per unit charge on the
charge carriers in the moving conductor.

\i Connect the moving conductor to an external
circuit with a resistor, and current will flow
with the power dissipated in the resistor equal to the
power associated with the applied force, which 
moves the wire in the external magnetic field.
(The applied force will be balanced by an 
equal and opposite magnetic force once an 
induced current in the moving conductor starts
to flow.)

\ei

%%%%%%%%%%%%%%%%%%%%%%%%%%%%%%%%%%%%%%%%%%%%%%%%%
\subsection{Lenz's law}
\bi

\i Lenz's law: {\em The direction of the induced 
current {\em opposes} the change of magnetic flux 
through the circuit.}

\i Said another way, Nature tries to preserve the
status quo.
(Use the right-hand rule to determine whether
a CW or CCW induced current will oppose the change
of flux.)

\i Lenz's law is a consequence of conservation of energy.

\i Demos:
(i) Dropping a magnet down a conducting tube
(it takes a long time to fall);
(ii) a jumping hoop (a small conducting loop 
encircling an electromagnet flies off when the 
electromagnetic is switched on);
(iii) a pendulum bob made of a copper sheet is
damped when it swings through a permanent
magnetic field (due to eddy currents induced in 
the copper sheet).

\ei

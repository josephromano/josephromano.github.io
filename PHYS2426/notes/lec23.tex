\section{Nature of light, ray optics}

%%%%%%%%%%%%%%%%%%%%%%%%%%%%%%%%%%%%%%%%%%%%%%%%
\subsection{Nature of light}
\bi

\i As shown in the previous lecture, 
light is an electromagnetic wave.
But depending on the type of measurement you make,
light will sometimes behave as a wave, and 
other times behave as a stream of particles.

\i The wave nature of light is most evident in
effects involving {\em interference}---i.e., the
fact that waves can combine coherently or cancel
out with other waves.

\i The particle nature of light is most evident
when discussing the interaction of light with 
electrons---e.g., the Compton effect (scattering
of light by electrons) or the photoelectric
effect (the ejection of electrons from a piece of
metal when light is shined on it).

%\i Albert Einstein received a Nobel Prize in 
%physics in 1921 for his 1905 paper describing the
%photoelectric effect in terms of {\em photons};
%particle of light each having energy 
%$E=hf$, where $f$ is the frequency of the light
%and $h=6.63\times 10^{-34}~{\rm J}\cdot{\rm s}$
%is Planck's constant.

\ei
%%%%%%%%%%%%%%%%%%%%%%%%%%%%%%%%%%%%%%%%%%%%%%%%
\subsection{Measuring the speed of light}
\bi

\i Light travels with speed
$c = 2.998\times 10^8~{\rm m/s}$
in vacuum.
It travel slower in other materials, e.g., $v=3c/4$ in
water.

\i Galileo tried measuring the speed of light
having two people uncover lanterns $\sim 10~{\rm km}$
apart.
This attempt was unsucessful since human reaction
time is much longer than the time it takes light to 
travel that distance.

\i Romer estimated the speed of light to be
$2.3\times 10^8~{\rm m/s}$ by monitoring 
the time it took Jupiter's moon Io to pass behind it.
If the Earth were moving toward Jupiter, the time
for Io's passage was less than that when the Earth
was moving away from Jupiter, similar to the Doppler
effect for sound.

\i Fizeau estimated the speed of light to be
$3.1\times 10^8~{\rm m/s}$ by using 
a rotating wheel with teeth and notches to let
light pass from a source to a distant mirror and back.

\ei
%%%%%%%%%%%%%%%%%%%%%%%%%%%%%%%%%%%%%%%%%%%%%%%%
\subsection{Ray approximation}
\bi

\i The ray approximation for light assumes that 
light travels in a straight line in a uniform medium, 
in a direction perpendicular to the wavefronts.
 
\i The ray approximation is a good approximation
when the wavelength of the light is much smaller than
the size of any obstacle or inhomogenity in the
medium. (Otherwise, one must use wave optics.)

\i For example, in the ray approximation, light casts 
sharp shadows as it passes through an opening.

\ei
%%%%%%%%%%%%%%%%%%%%%%%%%%%%%%%%%%%%%%%%%%%%%%%%
\subsection{Reflection}
\bi

\i When light is incident on a smooth surface 
between two different media, 
part of the light is reflected back into the
first medium while the rest of it is 
transmitted into the second medium.

\i For reflection off of a smooth surface, 
the angle of incidence equals the angle
of reflection, $\theta_1=\theta_1'$, where the
angles are measured with respect to the normal
to the surface. 

\i For a rough surface, parallel rays of
light will be reflected in all different 
directions.
Most objects are rough surfaces with respect
to reflection.

\i A {\em corner reflector}, made by placing
three flat mirrors next to one another at
right angles, has the property that the
incident light is reflected back along the 
same direction from which it came.
This is useful for reflectors on bicycles
and cars.

\i The Apollo 11 astronauts put corner 
reflectors on the Moon.
These mirrors reflect back light sent 
from lasers here on Earth, used
to monitor the Earth-Moon separation 
to a precision of $\sim 15~{\rm cm}$.

\i The law of reflection can be proved
using {\em Fermat's principle of least time},
which says that travels between two points
so as to minimize the travel time between
the two points.

\ei 

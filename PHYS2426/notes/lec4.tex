\section{Gauss's law examples, conductors}

\subsection{Gauss's law examples}

\bi

\i Gauss's law can be used to calculate the electric
field created by highly-symmetric charge distributions.
These include charge distributions that are 
spherically symmetric, cylindrically symmetric,
or planar symmetric; or some combinations of these.

\i Spherical symmetry:

\ex Find the electric field $\vec E$ both inside and 
outside a uniformly-charged sphere of radius $a$ and 
charge $Q$.

\ans: The electric field is directed radially outward 
with 
magnitude
\be
E = 
\left\{
\begin{array}{ll}
{k_e Q}/{r^2}, & \quad r>a
\\
{k_e Qr}/{a^3}, & \quad 0<r<a
\end{array}
\right.
\ee
%
so the field is that of a point charge $Q$ at the 
center of the sphere if $r>a$.

\ex Find the electric field $\vec E$ both inside and outside
a uniformly-charged spherical shell of radius $a$ and charge $Q$.

\ans: The electric field is directly radially outward with 
magnitude
\be
E = 
\left\{
\begin{array}{ll}
{k_e Q}/{r^2}\,, & \quad r>a
\\
0\,, & \quad 0<r<a
\end{array}
\right.
\ee

\i Cylindrical symmetry: 

\ex Find the electric field $\vec E$ a perpendicular distance
$r$ away from an infinite line charge with uniform linear
charge density $\lambda$.

\ans The electric field is directed radially outward from the 
line charge with magnitude
$E = {\lambda}/{2\pi\epsilon_0 r}$.

\i Planar symmetry:

\ex Find the electric field $\vec E$ a perpendicular distance
$r$ away from an infinite planar sheet of charge with uniform 
surface charge density $\sigma$.

\ans The electric fields is directed perpendicular to the 
surface with magnitude
$E = {\sigma}/{2\epsilon_0}$.

\ei
%%%%%%%%%%%%%%%%%%%%%%%%%%%%%%%%
\subsection{Conductors in electrostatic fields}
\bi

\i Properties:

1) $\vec E=\vec 0$ everywhere inside a conductor, for both
solid conductors and conductors with {\em empty} cavities.

2) If an isolated conductor carries a charge, then that excess
charge resides on the outer surface of the conductor.

3) The electric field just outside the surface of a charged 
conductor is perpendicular to the surface with magnitude
$E= \sigma/\epsilon_0$, where $\sigma$ is the surface charge
density.

4) For irregularly shaped conductors, the surface charge 
density $\sigma$ is greatest where the radius of curvature of
the surface is smallest (i.e., where the surface is sharpest).

\i Property (1) arises since charges in the conductor 
rearrange themselves in response to an applied electric 
field such that the {\em induced} field associated with 
these charges exactly cancels the applied field in the 
conductor, $\vec E+\vec E_{\rm ind}=\vec 0$.

\i Property (1) implies that any external electric field
is shielded from a cavity inside a conductor.
That's why you're safe inside a car during a lightning
storm.

\i Fields within a cavity, produced e.g., by a charge,
do make it outside of the conductor.  So the outside of
a conductor is not shielded from the fields inside a 
cavity of a conductor.

\ei

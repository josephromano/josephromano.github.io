\section{Direct current circuits}

\subsection{Electromotive force}
\bi

\i {Electromotive force} (or EMF) is the potential 
difference associated with the force 
per unit charge that drives charges from the negative
terminal of a battery 
(solar cell, electric generator, etc.)\ 
to the positive terminal.

\i EMF is what's constant in a battery (e.g., a 9-volt 
transistor battery). 
EMF is typically denoted by ${\cal E}$.

\i The output voltage $\Delta V$ of a {\em real} 
battery depends on the amount of current $i$ 
flowing out of the battery, according to 
$\Delta V = {\cal E}-i r$, where $r$ is the {\em internal}
resistance of the battery.

\i Example 28.1 shows that as a battery ages, the 
internal resistance $r$ increases, which 
causes $\Delta V$ and the power delivered to a load 
resistor $R$ to become much smaller fractions of 
${\cal E}$ and the total power produced by the
battery, $P_r+P_R$.
The increased power delivered to the internal
resistance causes the battery to heat up.

\i Example 28.2 shows that the {\em maximum} power 
delivered to a load resistor $R$ occurs when its value 
agrees with the internal resistance $r$.

\ei
%%%%%%%%%%%%%%%%%%%%%%%%%%%%%%%%%%%%%%%%%%%%%%%%%%%%%%
\subsection{Series and parallel combinations of resistors}
\bi

\i For a series combination of resistors $R_1$ and $R_2$:
%
\be
R_{\rm eq} = R_1 + R_2
\ee
%
The proof of this relation uses the fact that the currents
are equal for both resistors, $I_1=I_2=I$, while the 
potential differences add, 
$\Delta V=\Delta V_1 + \Delta V_2$.

\i For a parallel combination of resistors $R_1$ and $R_2$:
%
\be
\frac{1}{R_{\rm eq}} = \frac{1}{R_1} + \frac{1}{R_2}
\ee
%
The proof of this relation uses the fact that the 
currents add, $I=1_1+I_2$, while the 
potential differences are equal 
$\Delta V=\Delta V_1 = \Delta V_2$.
Note that $R_{\rm eq} < R_{\rm smallest}$ for a 
parallel combination.

\i Application: Christmas tree lights

(i) Old-fashioned Christmas tree lights were connected in 
series, which meant that if one bulb burned out, all the
other bulbs would go out (no current to those bulbs).

(ii) Newer Christmas tree lights were connected in parallel
to solve that problem.
But a parallel connection draws a lot of current, which 
makes these lights a fire hazard.

(iii) ``Miniature" Christmas tree lights are wired again 
in series, but each bulb has an insulating {\em jumper} 
that becomes conductive when that bulb burn out. 
So the remaining bulbs stay lit.

\ei
%%%%%%%%%%%%%%%%%%%%%%%%%%%%%%%%%%%%%%%%%%%%%%%%%%%%%%
\subsection{Kirchhoff's rules}
\bi

\i Kirchhoff's rules are used to analyze circuits that 
cannot be reduced using simple 
parallel and series combinations of resistors.

\i Kirchoff's rules are a consequence of 
conservation of charge and 
conservation of energy:

(i) Junction rule: 
The sum of all the currents into a node is zero.

(ii) Loop rule: 
The sum of all the potential differences around a closed
loop is zero.

\i When applying Kirchhoff's rules, it is important to follow
certain sign conventions: 
(i) if the source of a potential difference 
(e.g., a battery) is traversed from the $-$ to $+$ terminal, 
$\Delta V$ is assigned a positive sign; 
(ii) if current $i$ flows through resistor $R$ in the same 
direction that the loop is traversed, the potential 
difference is assigned a negative sign, $-I R$, a so-called
``voltage drop".

\ei 

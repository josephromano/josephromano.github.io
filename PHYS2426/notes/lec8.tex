\section{Dielectrics}

\subsection{Electric dipoles}

\bi
\i An {\em electric dipole} $\vec p$ consists of two 
charges $\pm q$ separated by a distance $d=2a$.
The magnitude of $\vec p$ is $p=2a|q| = |q|d$.
The direction of $\vec p$ is given by the vector
that points from $-q$ to $+q$.

\i Electric dipoles can be induced in an insulator by 
an external electric field.  
The external field separates the centers of negative
(electrons) and positive (nuclei) charge in an atom
or molecule.
The induced dipole moment is often proportional to the 
magnitude of the applied field, $\vec p = \alpha \vec E$.

\i Electric dipoles can also exist in the absence of
an external field---e.g., for {\em polar} molecules like
H${}_2$O, whose centers of positive and 
negative charge are already separated due to the shape 
of the molecule.

\i An electric dipole in a uniform electric field $\vec E$,
feels no net force since $\vec F_+=-\vec F_-$,
but does experience a net torque
$\vec\tau = \vec p\times\vec E$.
The torque acts so as to align $\vec p$ with $\vec E$.

\i An electric dipole in an electrostatic field has an 
associated potential energy $U_E = -\vec p\cdot\vec E$,
which is the work required to rotate the dipole 
from some initial configuration to some final configuration,
$W= \int_{\rm i}^{\rm f} \tau\,d\theta$.

\i Note that $U_E=-\vec p\cdot \vec E$ increases from
$U_E=-pE$ for $\theta=0$ (aligned), 
to $U_E= 0$ for $\theta=\pi/2$ (perpendicular),
to $U_E=+pE$ for $\theta=\pi$ (anti-aligned).

\ei

%%%%%%%%%%%%%%%%%%%%%%%%%%%%%%%%%%%%%%%%%%%%%%%
\subsection{Dielectrics}

\bi

\i A {\em dielectric} is an insulator placed 
between the conductors of a capacitor.

\i Alignment of polarized molecules within the
dielectric creates an induced electric field that
{\em opposes} the applied electric field 
associated with the free charges $\pm Q$ on the two 
conductors.

\i This induced electric field reduces the total
electric field between the conductors, leading to 
a reduced value of the potential diffence $\Delta V$,
and a corresponding increase in $C=Q/\Delta V$.

\i The increase in capacitance is encoded in the 
(dimensionless) dielectric constant $\kappa\ge 1$, 
defined by 
$C=\kappa C_0$, where $C_0$ is the capacitance in
the absence of a dielectric.

\i Equivalently, the permitivity of free space
$\epsilon_0$ should be replaced by the permitivity
of the dielectric, $\epsilon= \kappa\epsilon_0$.

\i $\kappa$ is related to the free and induced
charge densities $\sigma$ and $\sigma_{\rm ind}$
via
\be
\frac{\sigma}{\kappa}=
\sigma-\sigma_{\rm ind}
\ee

\i For vacuum, $\kappa =1$;
for air, $\kappa = 1.00059$;
for paper, $\kappa = 3.7$;
for a conductor, $\kappa = \infty$
(see Table 26.1).

\ei

\section{Sources of a magnetic field (continued)}

\subsection{Examples of fields calculated using Amp\`ere's law}

\bi

\i Cylindrical wire (uniform current $I$, radius $R$):
$B= \mu_0 I r/2\pi R^2$ for $r<R$; 
$B= \mu_0 I /2\pi r$ for $r>R$;
For both cases $\vec B$ circles around the axis of the wire.
 
\i Torus (current $I$, $N$ turns, cross-sectional radius $a$, 
inner and outer radii $b$ and $c$): $B=\mu_0 NI/2\pi r$ for
$b<r<c$.  The direction of $\vec B$ is circumferential.
Outside the magnetic field is weak but non-zero, circling around the 
cross-section of the torus.

\i Solenoid (current $I$, $n$ turns per unit length, radius
$a$):
Inside $B=\mu_0 nI$.  The direction of $\vec B$ is along
the axis of the solenoid.
Outside the magnetic field is weak but non-zero, circling around 
the cross-section of the solenoid.

\ei
%%%%%%%%%%%%%%%%%%%%%%%%%%%%%%%%%%%%%%%%%%%%%%%%%%
\subsection{Gauss's law for magnetic fields}
\bi

\i One can define magnetic flux $\Phi_B$ through a 
surface $S$ analogous to electric flux $\Phi_E$:
%
\be
\Phi_B \equiv \int_S \vec B\cdot \hat n\, dA
\ee

\i But since magnetic field lines form closed loops
(because isolated magnetic poles do not exist),
the magnetic flux through a {\em closed} surface 
$S$ is zero:
%
\be
\oint_S \vec B\cdot \hat n\,dA = 0
\ee
%
Recall that for electric fields, Gauss's law is 
$\oint_S \vec E\cdot \hat n\,dA = q_{\rm enc}/\epsilon_0$.

\ei

%%%%%%%%%%%%%%%%%%%%%%%%%%%%%%%%%%%%%%%%%%%%%%%%
\subsection{Magnetic properties of materials}
\bi

\i The magnetic properties of materials are determined
by the magnetic dipole moments associated with atomic
current loops (orbital or spin angular momentum of 
electrons).
A proper treatment of magnetism at the atomic level 
requires quantum mechanics.

\i A classical treatment of orbital angular momentum of
an electron yields 
$\mu_{\rm orb} = eL/2 m_{\rm e}$, with $\vec\mu_{\rm orb}$
pointing opposite $\vec L$.
Quantum mechanically, $\mu_{\rm orb}$ is quantized, 
with values given by integer multiples of 
$\sqrt{2} e\hbar/2 m_{\rm e}$, where
$\hbar = 1.05\times 10^{-34}~{\rm J}\cdot{\rm s}$ is
Planck's constant.

\i A magnetic moment associated with the spin angular 
momentum of an electron doesn't make sense classically 
since an electron is a {\em point particle}.
Quantum mechanically an 
electron has an intrinsic spin angular momentum $\pm \hbar/2$
and corresponding spin magnetic moment
$\mu_{\rm spin} = \pm e\hbar/2m_{\rm e}$. 
Since electrons are paired with opposite spins
(Pauli exclusion principle),
the spin magnetic moment is non-zero only for atoms 
containing an {\em odd} number of electrons.

\i Ferromagnetic materials (iron, cobalt, nickel)
have strong coupling between neigboring magnetic
dipole moments associated with unpaired spins 
forming magnetic {\em domains}.
These domains increase in size and align 
themselves in the direction of an external magnetic 
field.
The induced magnetization remains even if the external 
field is removed.
For $T>T_{\rm curie}$ ($770~{}^\circ {\rm C}$ for iron),
the magnetization is lost due to thermal motion of
the atoms.

\i Paramagnetic materials have magnetic dipole moments
associated with unpaired electron spins that align 
themselves in the direction of an external magnetic field. 
The coupling between neighboring dipole moments is
weak, so thermal effects easily disrupt the alignment
of the spin magnetic moments.

\i Diamagnetic materials (water, copper, gold, silver bismuth)
have magnetic dipole moments associated with the orbital 
angular momentum of electrons that change in a direction
{\em opposite} to an external magnetic field. 
The induced magnetization thus repels the external field.
Magnetic levitation is possible with diamagnetic superconductors.

\ei

\section{Electric field, electric field lines}

%%%%%%%%%%%%%%%%%%%%%%%%%%%%%%%%%%%%%%%%%%%%%%%%%%%%
\subsection{Electric field}

\bi

\i The electric force between charges is a {\em non-contact}
force.
A set of charges creates a {\em field} to which another 
charge responds.

\i The electric field at a point $P$ in space is a vector 
$\vec E$, whose direction is given by the electric force 
$\vec F_e$ on
a small, positive (test) charge $q_0$ placed at $P$, and whose
magnitude is given by $F_e/q_0$:
%
\be
\vec E = \vec F_e/q_0
\ee

\i For a single point charge $q$:
%
\be
\vec E = \frac{k_e q}{r^2}\,\hat r
\ee

\i For a set of discrete 
point charges $q_i$, $i=1,2,\cdots, N$ or
a continuous charge distribution:
%
\be
\vec E = \sum_{i=1}^N\frac{k_e q_i}{r_i^2}\,\hat r_i
\quad({\rm discrete}),
\qquad\vec E = \int \frac{k_e dq}{r^2}\,\hat r
\quad({\rm continuous})
\ee
%
where
%
\be
dq = \rho\, dV\,,
\quad
dq = \sigma\, dA\,,
\quad
dq = \lambda\, dl\,,
\ee
%
for volume, surface, and line charge densities 
$\rho$, $\sigma$, and $\lambda$, respectively.

\ei

%%%%%%%%%%%%%%%%%%%%%%%%%%%%%%%%%%%%5
\subsection{Worked examples}
\bi

\i Electric dipole: Two charges, $+q$ and $-q$ separated by distance
$a$ along the $x$-axis (located at $x=-a/2$, $x=a/2$, respectively).
Find $\vec E$ at point $P=(0,y)$.
\be
E_y(P)=0\,,\qquad
E_x(P) = \frac{k_e q a}{\left[y^2 + (a/2)^2\right]^{3/2}}
\ee

For $y\gg a$, $E_x(P)\simeq k_e q a/y^3$ (note $1/y^3$ dependence).

\i Uniform line charge $\lambda$:
Length $l$, charge $Q$, located on the $x$-axis
with ends at $x=-l/2$, $x=l/2$.
Find $\vec E$ at point $P=(0,y)$.
\be
E_x(P)=0\,,\qquad
E_y(P) = \frac{k_e Q}{y}\frac{1}{\sqrt{y^2 + (l/2)^2}}
\ee

For $y\gg l$, $E_y(P)\simeq k_e Q/y^2$ (point charge).

For $y<<l$, $E_y(P)\simeq \lambda/2\pi \epsilon_0 y$ (infinite line charge).

\ei
%%%%%%%%%%%%%%%%%%%%%%%%%%%%%%%%%%%%5
\subsection{Electric field lines}

\bi

\i Pictorial representation of the electric field $\vec E$.  
Direction of field lines point in the direction of $\vec E$; 
density of field lines (number per area perpendicular to $\vec E$) 
is proportional to the magnitude of $\vec E$.

\i Rules for drawing electric field lines: 

(i) lines begin on $+$ charges, end on $-$ charges (or may 
extend to infinity);

(ii) the number of lines leaving or terminating on a charge
should be proportional to the magnitude of the charge;

(iii) electric field lines cannot cross.

\i For a point charge, the density of lines at a distance
$r$ from the charge is given by $N/4\pi r^2$, which is 
proportional to the magnitude of the field $E=k_e q/r^2$
as it should.
 
\ei

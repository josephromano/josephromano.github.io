\section{Magnetic fields (continued)}

\subsection{Magnetic force on a current-carrying wire}
\bi

\i The magnetic force on a current-carrying wire is
%
\be
\vec F_B = I \vec L\cross\vec B
\ee
%
which is a simple extension of the magnetic force on a 
single charged particle $q$ to $nAL$ charged particles 
moving with drift velocity $\vec v_{\rm d}$ in a wire
(recall $\vec I = qn\vec v_{\rm d}A$).

\i For a wire which changes its direction
$\vec F_B = \int d\vec F_B = I \int d\vec s\cross\vec B$.

\i There is no net force on a current-carrying loop in  
a uniform magnetic field (Example 29.4):
$\oint d\vec F_B=\vec 0$.

\ei
%%%%%%%%%%%%%%%%%%%%%%%%%%%%%%%%%%%%%%%%%%%%%%%%%%
\subsection{Torque on a current-carrying loop}
\bi

\i Consider a rectangular loop of wire (sides $a$
and $b$) in a uniform magnetic field $\vec B$, with
$\vec B$ parallel to the sides having length $b$ and
perpendicular to the area vector $\vec A$.

\i Then the total magnetic force on the wire is zero,
but there is a non-zero net torque about an axis perpendicular to 
$\vec B$ (passing through the center of the loop):
%
\be
\tau = IAB = \mu B
\ee
%
where $\mu \equiv I A$ is the {\em magnetic dipole
moment} of the current loop.

\i More generally, $\vec\tau = \vec \mu\cross\vec B$,
where $\vec\mu\equiv I\vec A$.
(Recall: $\vec\tau = \vec p\cross\vec E$ for an electric
dipole in a uniform electric field.)

\i The torque is in a direction to rotate the magnetic
dipole moment $\vec\mu$ in the direction of the 
magnetic field $\vec B$.

\i Similar to $U_E = -\vec p\cdot \vec E$, there is
a potential energy associated with a current
loop in a uniform magnetic field, $U_B = -\vec \mu\cdot\vec B$.

\i Application: motors 
(but need to alternate the direction of the current flow 
in order to keep the loop turning.)

\ei
%%%%%%%%%%%%%%%%%%%%%%%%%%%%%%%%%%%%%%%%%%%%%%%%%%
\subsection{Hall effect}

\bi
\i The Hall effect (Edwin Hall, 1879) is an experimental
way of determining the sign of the charge carriers.

\i Consider a rectangular slab of material with height
$d$, thickness $t$, carrying a current $I$ along its
length. 
Apply a uniform magnetic field $\vec B$ perpendicular 
to the current in the direction of the 
thickess of the slab.

\i Then the magnetic force on the current carriers
sets up a potential difference $\Delta V_H$, called
the Hall potential, given by 
$\Delta V_H = v_{\rm d} B d$ where 
$v_{\rm d}$ is the 
drift velocity of the charges.
(Recall: $I=|q|nv_{\rm d}A$, where $n$ is the number
density of charge carriers and $A=td$ is the cross-sectional
area of the current flow.)

\i The sign of the Hall potential indicates the sign
of the charge carriers.

\i NOTE: The quantum Hall effect refers to the fact that
$\Delta V_H$ is quantized.
(K.~von~Klitzking, Nobel Prize in 1985).

\ei

\section{Magnetic fields}

\subsection{Basic properties}
\bi

\i Two magnetic poles, $N$ and $S$, similar to $+$ and $-$ 
electrical charges.
Like poles repel, unlike poles attract; 
isolated magnetic poles do not exist in nature
(unlike isolated electric charges).

\i Iron filings are attracted to both poles of a magnet.
They become magnetized.

\i Iron filings align themselves with magnetic 
field lines.
Magnetic field lines form closed loops (as opposed 
to starting or terminating at $N$ and $S$ poles).

\i The Earth has an intrinsic magnetic field 
(due to motion of charged molten material in its
core) with its South magnetic pole located near
Earth's North geographic pole.
A compass needle is a small magnet which aligns
its North pole with Earth's South magnetic pole
(points toward geographic North).

\i All magnetic fields are ultimately due to the
motion of electric charges (e.g., electron orbital 
motion and instrinsic spin).

\ei

%%%%%%%%%%%%%%%%%%%%%%%%%%%%%%%%%%%%%%%%%%%%%%%%%%%%
\subsection{Magnetic force on a charged particle}
\bi
\i The magnetic force on a charged particle is 
%
\be
\vec F_B = q\vec v\cross\vec B
\ee
%
Magnitude: $F_B = |q|vB\sin\theta$, where $\theta$
is the angle between the velocity $\vec v$ of the 
charged particle and the magnetic field $\vec B$.
Direction: right-hand rule.

\i If the particle is at rest ($v=0$) or if
$\vec v$ is parallel to $\vec B$, then $\vec F_B=\vec 0$.
Measuring $\vec F_B$ for a given $\vec v$ determines
only the components of $\vec B$ perpendicular to $\vec v$.

\i Since $\vec F_B$ is perpendicular to $\vec v$,
the magnetic force does no work on a charged particle.
(It can only change the direction of $\vec v$, not
its magnitude).

\i Units of $\vec B$: 
$1~{\rm Tesla} \equiv 1~{\rm N}/{\rm A}\cdot{\rm m}$.
The Gauss defined by $1~{\rm Tesla} \equiv 10^4~{\rm Gauss}$ is 
also sometime used.

\i Earth's magnetic field $\sim 0.5~{\rm Gauss}$;
MRI magnet $\sim 1.5~{\rm T}$;
superconducting magnet $\sim 30~{\rm T}$;
magnetic field of a neutron star $\sim 10^8~{\rm T}$.

\ei

%%%%%%%%%%%%%%%%%%%%%%%%%%%%%%%%%%%%%%%%%%%%%%%%%%%%
\subsection{Motion of a charged particle in a uniform magnetic field}
\bi

\i The motion of a charged particle in a uniform
magnetic field with $\vec v$ perpendicular to $\vec B$ 
is {\em uniform circular motion} with $v={\rm const}$.
The speed, radius, magnetic field, and charge-to-mass
ratio of the particle are related by
%
\be
v = \frac{|q|rB}{m}
\quad\Leftrightarrow\quad
\omega = \frac{|q|B}{m}
\quad\Leftrightarrow\quad
T = \frac{2\pi m}{|q|B}
\ee
%
where $\omega$ and $T$ are the {\em cyclotron}
(angular) frequency and period, respectively.

\i If $\vec v$ has a component parallel to $\vec B$,
then the motion is a helix.

\i Applications of the motion of a charged particle in
a uniform magnetic field include:

(i) velocity selector: formed from perpendicular uniform
electric and magnetic fields. 
The velocity of a charged particle is constant provided
$v= E/B$.

(ii) mass spectrometer: a velocity selector followed by 
another uniform magnetic field $\vec B_0$.
It is used to select different particle masses by the 
radii of their circular motion $|q|/m = E/B B_0 r$.
J.J.~Thomson (1897) determined the charge-to-mass ratio of the 
electron this way.

(iii) a cyclotron: the first particle accelerator 
(E.O.~Lawrence, M.S.~Livingston, 1934) constructed
from a uniform magnetic field and two ``dees", having 
alternating potential difference.
A charged particle is accelerated to larger and larger 
velocities with correspondingly larger and larger radii.

\ei

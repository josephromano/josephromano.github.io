\section{Maxwell's equations and electromagnetic 
waves}

%%%%%%%%%%%%%%%%%%%%%%%%%%%%%%%%%%%%%%%%%%%%%%%%
\subsection{Displacement current}
\bi

\i As mentioned in previous lectures,
Faraday discovered that a changing magnetic field 
induces an electric field (Faraday's law of induction).

\i Maxwell realized that, likewise, 
a changing electric field induces a magnetic field.
This observation requires adding a {\em displacement current} 
%
\be
I_{\rm d}\equiv\epsilon_0\, d\Phi_E/dt
\ee
to the conduction current 
$I_{\rm enc}$ in Amp\`ere's law, where $\Phi_E$ is the
electric flux through a surface $S$ spanning a closed
curve $C$.

\i The displacement current is necessary for electric
charge conservation, as illustrated by charging a 
parallel-plate capacitor.  
(One can consider a surface $S$ spanning a closed 
curve $C$ that does not 
interesect the wire but extends into the region between
the plates of the capacitor.) 

\ei
%%%%%%%%%%%%%%%%%%%%%%%%%%%%%%%%%%%%%%%%%%%%%%%%
\subsection{Maxwell's equations}
\bi

\i Including the effects of changing magnetic 
and electric fields, the full set of 
equations for electricity and magnetism (called
Maxwell's equations) are:
\begin{align}
&\oint_S \vec E\cdot\hat n\,dA = \frac{q_{\rm enc}}{\epsilon_0}
\quad({\rm Gauss's\ law})
\\
&\oint_C \vec E\cdot d\vec s = -\frac{d\Phi_B}{dt}
\quad({\rm Faraday's\ law})
\\
&\oint_C \vec B\cdot d\vec s = \mu_0\left(
I_{\rm enc} + \epsilon_0\frac{d\Phi_E}{dt}\right)
\quad({\rm Ampere's\ law\ with\ Maxwell's\ displacement\ current})
\\
&\oint_S \vec B\cdot\hat n\,dA = 0
\quad({\rm no\ magnetic\ monopoles})
\end{align}

\i Maxwell's equations and the Lorentz force law:
%
\be
\vec F = q(\vec E +\vec v\cross\vec B)\,,
\qquad
\vec F = d\vec p/dt
\ee
%
describe {\em all} of electrodynamics.

\ei
%%%%%%%%%%%%%%%%%%%%%%%%%%%%%%%%%%%%%%%%%%%%%%%%
\subsection{Electromagnetic waves}
\bi

\i Maxwell's equations in vacuum admit {\em wave} 
solutions---disturbances in the electric 
and magnetic fields that propagate through space
with speed 
%
\be
v=\frac{1}{\sqrt{\mu_0\epsilon_0}} = 2.998\times 10^8~{\rm m/s}
\equiv c
\ee

\i The speed of electromagnetic waves in vacuum
is the {\em same} as the speed of light $c$, which led 
Maxwell to infer that light is an example of an 
electromagnetic wave.
(The relationship between light and electricity and
magnetism was not known before.)

\i Electromagnetic waves can propagate in vacuum
(in the absence of charges and currents), because
changes in the electric field induce magnetic fields,
whose changes induce electric fields, etc.

\i Visible light is just a small part of the 
electromagnetic {\em spectrum}, consisting of 
wavelengths between 400~nm (violet light) and 
700~nm (red light).
Infrared radiation, microwaves, and radio waves
have longer wavelengths.
Ultraviolet radiation, X-rays, and gamma rays have
shorter wavelengths.

\ei 

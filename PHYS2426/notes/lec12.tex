\section{Direct current circuits (continued)}

\subsection{Resistor-Capacitor ($RC$) circuits}

\bi
\i $RC$ circuits contain resistors, capacitors and possibly
sources of EMF.

\i Charging a capacitor:
When a capacitor $C$ is charged through a resistor $R$ 
by an EMF ${\cal E}$, the charge increases exponentially
while the current decreases exponentially:
%
\be
q(t) = Q_{\rm max}\left(1-e^{-t/RC}\right)\,,
\qquad
i(t) = I_{\rm max} \, e^{-t/RC}
\ee
%
where $Q_{\rm max} = {\cal E}C$ and 
$I_{\rm max} = {\cal E}/R$.
The combination $\tau\equiv RC$ in the 
exponential is called the $RC$ 
{\em time constant}, since it has units of time.

\i These results follow from applying the loop 
rule to the $RC$ circuit:
${\cal E} - i R- {q}/{C} =0$ or
${\cal E} - R\,{dq}/{dt}- {q}/{C} =0$,
and then solving the differential equation
for $q(t)$.

\i Discharging a capacitor:
When a capacitor $C$ is discharged through a resistor
$R$, both the charge on the capacitor and the current
through the circuit decrease exponentially:
%
\be
q(t) = Q_{\rm max} \, e^{-t/RC}\,,
\qquad
i(t) = I_{\rm max} \, e^{-t/RC}
\ee
%
where $I_{\rm max} = Q_{\rm max}/RC$.

\i These results follow from applying the loop 
rule to the $RC$ circuit:
${q}/{C} -i R=0$ or
${q}/{C} + R\,{dq}/{dt} =0$,
and then solving the differential equation
for $q(t)$.
(NOTE: $i = -dq/dt$ since the charge $q$ on the capacitor
is {\em decreasing}.)

\ei

\subsection{Household wiring and electrical safety}

\bi
\i Some important numbers:

- The EMF of a wall outlet is 120~volts (AC, but that really
doesn't matter for this analyis).

- 14-gauge household wire typically carries up to $\sim 20~{\rm A}$ 
of current.
It has a resistance of $2.5~\Omega/1000~{\rm ft}$, 
so 100~feet of that wire has a resistance 
$R_{\rm w} \sim 0.25~\Omega$.

- A household appliance (e.g., hair dryer) has 
$P\sim 1200~{\rm W}$ implying 
$I=10~{\rm A}$ and $R_{\rm load}=12~\Omega$.

- A human being has a resistance 
$R_h\sim 100,000~\Omega$ (dry skin),
but $R_h\sim 1000~\Omega$ (wet skin).

- Summary:
$R_{\rm w} \sim 0.1~\Omega \ll R_{\rm load} \sim 10~\Omega
\ll R_{\rm h}\sim 1000~\Omega$.

\i Current kills, not potential difference.
$I=0.01~{\rm A}$ will give a painful shock;
$I=0.1~{\rm A}$ is potentially lethal.
($120~{\rm V}$ and $R_{\rm h}\sim 1000~\Omega$
gives $I\sim 0.1~{\rm A}$, which can kill you.)

\i Modern household wire consists of 3 wires: 
a ``hot" wire encased in black plastic ($120~{\rm V}$), 
a ``neutral" wire encased in white
plastic (typically grounded, so $0~{\rm V})$, and a ``ground"
wire, usually bare or encased in green plastic, which should
be connected to the Earth (i.e., ``ground").

\i Circuit breakers typically consist of fuses (older houses)
or spring-loaded switches, which either ``burn out" or ``trip" 
if too much current (e.g., greater than 20~A) flows in the circuit,
thus preventing fires.

\i NEVER replace an old-fashioned fuse with a penny or
any other piece of metal that won't burn out if too much
current is flowing in the circuit.

\i Household appliances 
that have a three-prong plug have the ground wire connected 
to the chassis.
If the appliance is connected to a three-prong outlet and 
the hot wire accidently touches the chassis of the appliance, 
then the total
resistance is approximately that of the wire, which will cause
the circuit breaker to trip, which prevents you from being 
electrocuted even if you are touching the appliance.

\i If the appliance doesn't have a ground wire, and the hot
wire accidently touches the chassis, then the total resistance
of the circuit will be that of the load.
The circuit breaker won't trip since the current is normal
(i.e., less than 20~A).
But you will have 120~V across you, which will produce 0.1~A
of current through you if you have wet skin.
This can electrocute you.

\ei

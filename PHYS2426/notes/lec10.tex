\section{Resistance, superconductivity, electrical power}

%%%%%%%%%%%%%%%%%%%%%%%%%%%%%%%%%%%%%%%%%%%%%%%%%%
\subsection{Classical model of electrical conduction in metals}

\bi

%\i Goal: an expression for the resistivity $\rho$ or conductivity
%$\sigma$ in terms of fundamental properties of a material---i.e.,
%$q$, $n$, mass of charge carrier, etc.

\i Paul Drude ($\sim 1900$) came up with a {\em classical} model
of electrical conduction in metals, which treats the free electrons
as a classical {\em gas}.

\i In the absence of an applied electric field, the free electrons
move in random directions with thermal velocity $\vec v_{\rm th}$,
colliding with atomic nuclei, fixed in the metal.
In the presence of an applied electric field $\vec E$,
the free electrons accelerate in the direction opposite the field, 
with $\vec a = q\vec E/m_e$, so 
$\vec v_{\rm f} = \vec v_{\rm i} + \vec a \Delta t$.

\i Since $\vec v_{\rm i, avg}=\vec 0$, the average of $\vec v_{\rm f}$
over all free electrons gives the drift velocity
$\vec v_{\rm d}\equiv \vec v_{\rm f, avg} = (q\vec E/m_e)\tau$,
where $\tau\equiv l_{\rm avg}/v_{\rm th}$ is the average time 
between collisions. 

\i Thus, 
%
\be
\vec J = qn\vec v_{\rm d} = (q^2 n\tau/m_e)\vec E
\quad\Rightarrow\quad
\sigma = q^2 n\tau/m_e\,,
\quad
\rho = m_e/(q^2 n\tau)
\ee
 
\i Problems:
(i) the predicted resistivity values are a factor $10\times$ 
smaller than the observed values;
(ii) the predicted temperature dependence is 
$\rho\propto\sqrt{T}$ as opposed to the observed dependence 
$\rho\propto T$.

\i Reason: classical model (gas of electrons).
Proper treatment requires quantum mechanics.

\ei

%%%%%%%%%%%%%%%%%%%%%%%%%%%%%%%%%%%%%%%%%%%%%%%%%%
\subsection{Temperature dependence of resistivity}

\bi

\i Observed temperature dependence of resistivity:
$\rho = \rho_0\left[1+\alpha(T-T_0)\right]$,
where $\rho_0$ is the resistivity at temperature $T_0$,
and $\alpha$ is the temperature coefficient of
resistivity (units: $1/{}^\circ {\rm C}$).

\i NOTE: $\alpha\sim 10^{-3}/{}^\circ{\rm C}$
for most conductors.
It is possible for $\alpha<0$ for semiconductors
(increasing $T$ may increase density of charge carriers).

\ei

%%%%%%%%%%%%%%%%%%%%%%%%%%%%%%%%%%%%%%%%%%%%%%%%%%
\subsection{Superconductors}
\bi

\i The resistance of certain materials goes to zero
below a {\em critical temperature} $T_{\rm c}$.

\i In 1911, Kamerlingh-Onnes discovered that Mercury
was superconducting below 4.2~K (recall that 0~K, which is
{\em absolute zero}, corresponds to $-273.15~{}^\circ {\rm C}$).
In 1987, the Nobel Prize in Physics was awarded
to Bednorz and M\"{u}ller for discovering superconductivity
in copper-oxide materials at relatively high temperatures
($\sim 100~{\rm K}$).

\i Current continues to flow in a superconductor
even if the potential difference is removed.
No energy is dissipated as heat in a superconducting 
wire.

\i Application: Superconducting magnets made with coils 
of superconducting wire produce stronger magnetic fields
than conventional, non-superconducting, magnets.
Superconducting magnetis are often used in MRI and NMR 
machines in hospitals.

\ei
%%%%%%%%%%%%%%%%%%%%%%%%%%%%%%%%%%%%%%%%%%%%%%%%%%
\subsection{Electrical power}
\bi

\i Recall that power is the time rate of change of
energy production or consumption,
$P = dE/dt$.

\i In an electric circuit,
a charge $q$ loses energy $q\Delta V$ as it moves
through a circuit from the positive terminal of a
battery to the negative terminal.
It gains energy $q\Delta V$ as it moves from the 
negative terminal to the positive terminal 
{\em through} the battery.
(Chemical reactions provide the energy.)

\i Thus, the power delivered to a resistor $R$
with potential difference $\Delta V$ is
%
\be
P = q\Delta V/\Delta t = I\Delta V = I^2 R = (\Delta V)^2/R
\ee

\i The power delivered to a resistor is dissipated as heat,
so-called $I^2R$ {\em losses}.

\ei


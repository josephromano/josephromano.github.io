\section{Sources of a magnetic field}

\subsection{Biot-Savart law}
\bi
\i Oersted (in 1819) discovered that a current-carrying
wire produces a magnetic field that circles around the 
wire in accordance with the right-hand rule 
(thumb points in the direction of the current; fingers 
curl in the direction of the magnetic field).

\i Biot and Savart subsequently wrote down an expression for the 
(infinitesimal) magnetic field $d\vec B$ at point $P$
produced by an (infinitesimal) current element $I\,d\vec s$:
%
\be
d\vec B = \frac{\mu_0}{4\pi}\frac{I\,d\vec s\cross \hat r}{r^2}\,,
\quad{\rm where}\quad
\mu_0 = 4\pi \times 10^{-7}~\frac{{\rm T}\cdot{\rm m}}{\rm A}
\ee
%
The unit vector $\hat r$ points from the current element
$I\, d\vec s$ to $P$, with $r$ being the distance between them.
To obtain the total field produced by a full wire,
one must integrate: $\vec B = \int_{\rm wire}\, d\vec B$.

\i Examples:

- field at a distance $a$ above a finite segment of a 
straight wire, with endpoints making
angles $\theta_1$, $\theta_2$ with respect to vertical:
$B = (\mu_0/4\pi a)(\sin\theta_2-\sin\theta_1)$.

- field at a perpendicular distance $a$ away from 
infinitely-long, straight wire: $B= \mu_0 I/2\pi a$.

- field at the center of an arc $\Delta\theta$ of a circle 
of radius $a$: $B=\mu_0 I\Delta\theta/4\pi a$.

- field at the center of a circular loop: $B=\mu_o I/2a$.

The direction of $\vec B$ for these examples is given by 
the right-hand rule as discussed previously.

\ei

%%%%%%%%%%%%%%%%%%%%%%%%%%%%%%%%%%%%%%%%%%%%%%%%%%%%%%%
\subsection{Force between two current-carrying wires}
\bi

\i Consider two infinitely-long, straight current-carrying
wires (currents $I_1$, $I_2$) parallel to one another
and separated by perpendicular distance $a$.  Then the 
force per unit length between the wires is
%
\be
\frac{F}{l} = \frac{\mu_0\, I_1 I_2}{2\pi a}
\ee

\i The above result can be obtained by finding the 
magnetic field $\vec B_1$ 
produced by wire 1 at the location 
of wire 2, and then calculating the force that 
$\vec B_1$ exerts on the current $I_2$.

\i If the currents in the two wires flow in the same
direction, then the force is attractive; if the currents
flow in opposite directions, then the force is repulsive.

\ei
%%%%%%%%%%%%%%%%%%%%%%%%%%%%%%%%%%%%%%%%%%%%%%%%%%%%%%%
\subsection{Amp\`ere's law}
\bi

\i Amp\'ere's law is similar to Gauss's law for electric
fields, but it involves magnetic fields, enclosed currents,
and closed loops (instead of electric fields, enclosed
charges, and closed surfaces):
%
\be
\oint_C \vec B\cdot d\vec s = \mu_0\, I_{\rm enc}
\ee
%
where $I_{\rm enc}$ is the (net) current passing through
any surface spanning the closed curve $C$.
The current gets a positive sign if it passes in the same
direction as the normal vector to the surface, defined by the 
RHR.

\i One can prove this result for a single current-carryng
wire by using 
the expression for the magnetic field obtained using 
the Biot-Savart law, and then showing that the integral on 
the LHS is equal to $\mu_0$ times $I$, if the wire passes
through the closed loop, and 0 otherwise.

\i The result extends to multiple currents using the 
superposition principle.

\i Amp\`ere's law can be used to used to calculate the 
magnetic field for highly-symmetric current 
configurations (examples in next lecture).

\ei

\section{Electric charge, electric force}

\subsection{Electric charge} 

\bi

\i Properties of electric charge:

(i) Two types: {\em positive} and {\em negative}
($+$ and $-$), whose effects tend to cancel.
Opposite charges attract, like charges repel.

(ii) Total electric charge is {\em conserved}
(cannot be created or destroyed, only transferred from one 
object to another).

(iii) Electric charge is {\em quantized}
(integer multiples of the
charge of an electron $-e = -1.602\times 10^{-19}~{\rm C}$).

\i Materials:

(i) {\em Conductors}: one or more electrons per atom are able to 
move freely through the material.
Examples: aluminum, copper, gold, ... (metals).

(ii) {\em Insulators}: electrons are bound to atomic nuclei.
Examples: rubber, wood, paper, ...

\i Some demos: 

1) A plastic rod rubbed with fur is negatively charged;
a glass rod rubbed with silk is positively charged.
Show that opposite charges attract, like charges repel.

2) An uncharged aluminum can will be attracted to both
negative and positive rods.
Electrons in the aluminum can (a conductor) move away 
from the negative rod or toward the positive rod 
leading to an attractive force in bothe cases.

3) Bits of uncharged paper will be attracted to both
negative and positive rods due to charge separation 
within the atoms / molecules in the paper (an insulator).

4) A charged rubber ballon sticks to the white board
for the same reason as (3).

\ei

%%%%%%%%%%%%%%%%%%%%%%%%%%%%%%%%%%%%%%%%%%%%%%%%%%%%
\subsection{Electric force}
\bi

\i {\em Coulomb's law}:
The electric force exerted by point charge 
$q_1$ on point charge $q_2$, 
at rest with respect to one another and 
separated by a distance $r$:
%
\be
\vec F_{12} = \frac{k_e q_1 q_2}{r^2}\,\hat r\,,
\quad{\rm where}\quad
k_e= 8.98\times 10^9~
\frac{{\rm N}\cdot{\rm m}^2}{{\rm C}^2}
\ee
%
and $\hat r$ is a unit vector pointing from $q_1$ to $q_2$.

\i The force is attractive (in the $-\hat r$
direction) if the two charges have opposite signs;
the force is repulsive (in the $+\hat r$ direction) if the 
two charges have the same signs.

\i One sometimes writes
%
\be
k_e=\frac{1}{4\pi\epsilon_0}\,,
\quad{\rm where}\quad
\epsilon_0= 8.85\times 10^{-12}~\frac{{\rm C}^2}{{\rm N}\cdot{\rm m}^2}
\ee
where $\epsilon_0$ 
is the so-called {\em permitivity} of free space (vacuum).

\i The electric force on $q$ due to a set of point charges
$q_1$, $q_2$, ..., $q_N$ is given by the vector sum of the 
forces 
%
\be
\vec F_q = \sum_{i=1}^N \frac{k_e q q_i}{r_i^2}\,\hat r_i
\ee
%
where $r_i$ and $\hat r_i$ are the separation and unit vector
between $q$ and $q_i$.
This is called the {\em superposition principle}.

\ei



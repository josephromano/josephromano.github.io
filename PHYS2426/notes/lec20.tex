\section{Energy stored in an inductor, $LC$ circuits}

%%%%%%%%%%%%%%%%%%%%%%%%%%%%%%%%%%%%%%%%%%%%%%%%
\subsection{Energy stored in an inductor}
\bi

\i Energy is stored in the current flowing in
an inductor, similar to energy being stored in
the charge on a capacitor.

\i The power delivered to an inductor is given
by $P = i{\cal E}_L = i L\,di/dt$.
Integrating the power with respect to time to 
find the energy, we have
%
\be
U_B = \int_0^t P(t)\, dt = \int_0^i iL\,di
= \frac{1}{2}L i^2
\ee
%
The subscript $B$ indicates that the stored 
energy is associated with a magnetic field.
(NOTE: The above result is similar to 
$U_E= \frac{1}{2}C(\Delta V)^2= q^2/2C$
for the energy stored in a capacitor.)

\i Consider a solenoid 
($B=\mu_0 ni$, $L = \mu_0 n^2 lA$).  
Substituting these expressions into
$U_B = \frac{1}{2}Li^2$, we obtain
$u_B=\frac{1}{2\mu_0} B^2$
for the {\em energy density} (energy/volume).
(NOTE: The above result is similar to 
$u_E = \frac{1}{2}\epsilon_0 E^2$
for the energy density in the electric field, 
obtained from $U_E = \frac{1}{2}C(\Delta V)^2$ for 
a parallel-plate capacitor, $C=\epsilon_0 A/d$,
$E= \Delta V/d$.)

\i Although the above expressions for the energy
density $u_B$ and $u_E$ 
were obtained for very specific cases,
i.e., solenoid and parallel-plate capacitor,
these expressions hold for {\em general} electric and
magnetic fields.

\ei

%%%%%%%%%%%%%%%%%%%%%%%%%%%%%%%%%%%%%%%%%%%%%%%%
\subsection{$LC$ circuits}

\bi
\i Consider a fully-charged capacitor $C$ connected
to an inductor $L$.
When the switch is closed, charge will start to
flow in the circuit,
giving rise to a current $i=-dq/dt$ (minus
sign since charge leaving the capacitor
corresponds to a positive current).

\i Applying the loop rule leads to the differential 
equation ${q/C}-L\,di/dt=0$, or equivalently
%
\be
L\,\frac{d^2q}{dt^2}
+\frac{q}{C}=0
\ee

\i This equation is similar to the simple harmonic
oscillator equation for a mass $m$ attached to a spring
with spring constant $k$ in the absence of friction:
$m\, d^2 x/dt^2 = -kx$, which has oscillatory 
solutions $x(t) = x_{\rm max}\cos(\omega t + \phi)$, with 
angular frequency $\omega\equiv\sqrt{k/m}$.

\i Thus, for the $LC$ circuit, charge (and current) will 
oscillate with angular frequency $\omega\equiv 1/\sqrt{LC}$.

\ei
%%%%%%%%%%%%%%%%%%%%%%%%%%%%%%%%%%%%%%%%%%%%%%%%
\subsection{$RLC$ circuits}

\bi
\i Adding a resistor in series to the $LC$ circuit 
changes the differential equation to 
%
\be
L\,\frac{d^2 q}{dt^2} + R\,\frac{dq}{dt} + \frac{q}{C}=0
\ee
%
with solutions corresponding to damped oscillations:
%
\be
q(t) = q_{\rm max}\,e^{-t/2\tau}
\cos(\omega t +\phi)\,,
\qquad
\omega\equiv \frac{1}{\sqrt{LC}}\sqrt{1-\frac{R^2C}{4L}}
\ee

\i Depending on the sign of the quantity under
the square-root, 
the solutions correspond to either under-damped (+),
over-damped ($-$), or critically-damped (0) motion.

\i Energy is now dissipated in the resistor, analogous
to a frictional force $F_{\rm f}=-bv$ in a {\em damped} mass-spring system.

\i The correspondence between the mechanical and 
electrical oscillators are:
$L\leftrightarrow m$;
$1/C\leftrightarrow k$;
$R\leftrightarrow b$;
$q\leftrightarrow x$;
$i\leftrightarrow v$.

\ei

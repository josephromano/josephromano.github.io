\section{Image formation by refraction
and thin lenses}

%%%%%%%%%%%%%%%%%%%%%%%%%%%%%%%%%%%%%%%%%%%%%%%%
\subsection{Image formation by refraction}
\bi

\i Images can also be formed by refraction of 
rays of light from one medium into another.

\i We will consider a spherical surface 
between two media with indices of refraction
$n_1$ and $n_2$, respectively.  
$R$ will denote the radius of curvature of 
the spherical surface.

\i Tracing rays of light from the object point
$O$ to the image point $I$, using Snell's law
and trigonometry in the paraxial approximation, 
one finds
\be
\frac{n_1}{p}+\frac{n_2}{q} = \frac{n_2-n_1}{R}
\ee
where $p$ and $q$ denote the object and image
location, respectively.

\i The sign conventions for this equation are:

$p>0$ if the object is in medium 1, in front of
the surface; 

$q>0$ if the image is in medium 2, 
in back of the surface; 

$h'>0$ if the image
is upright; 

$R>0$ if the center of curvature is
in back of the surface.

\i Application: For a flat surface, 
$R\rightarrow 0$ implying $n_1/p=-n_2/q$.
Using this simplified formula, one can show that 
the apparent depth of an object in water is 
less than the actual depth by a factor of 
$n_2/n_1 = n_{\rm air}/n_{\rm water}=3/4$. 
(See Example 36.7.)

\ei
%%%%%%%%%%%%%%%%%%%%%%%%%%%%%%%%%%%%%%%%%%%%%%%%
\subsection{Thin lenses}
\bi

\i For the case of thin lenses (made of e.g., glass, 
with index of refraction $n$) in air, one needs to
consider the formation of images by refraction for 
both the front and back surfaces of the lens.

\i Taking the image formed by refraction from 
surface 1 as the object for the image formed by 
refraction from surface 2, one obtains (in the 
limit that the thickness of the lens goes to zero):
%
\be
\frac{1}{p}+ \frac{1}{q} = \frac{1}{f}\,,
\qquad
\frac{1}{f}\equiv
(n-1)\left(\frac{1}{R_1}-\frac{1}{R_2}\right)
\ee
%
where $R_1$ and $R_2$ are the radii of curvature
for the two surfaces of the lens.

\i The second equation is called the 
``lens maker's equation", where $f$ is the 
focal length of the thin lens.

\i The sign conventions for the thin lens equation
are the same as those for images formed by 
refraction from
a single surface, with the additional conventions 
that $R_{1,2}>0$ if the 
center of curvature of surface 1,2 is in back of 
the lens; and $f>0$ for a converging lens.

\ei

%%%%%%%%%%%%%%%%%%%%%%%%%%%%%%%%%%%%%%%%%%%%%%%%
\subsection{Ray tracing for converging and diverging
lenses}
\bi

\i A {\em converging} lens is thicker in the middle than
at the edges; parallel rays of light meet at the 
focal point in back of the converging lens, in the
paraxial approximation.

\i A {\em diverging} lens is thinner in the middle than
at the edges; parallel rays of light diverge from the 
focal point in front of the diverging lens, in the 
paraxial approximation.

\i By tracing rays of light through a lens,
one can show that:

(i) a converging lens produces both
real and virtual images (if $p>f$ and $p<f$, respectively).

(ii) a diverging lens {\em always} produces virtual
images that are upright and reduced in size.

\i Converging and diverging lenses behave similarly 
to concave and convex mirrors.

\ei

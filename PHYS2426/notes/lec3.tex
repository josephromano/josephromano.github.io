\section{Electric flux, Gauss's law}

\subsection{Electric flux}

\bi

\i Electric flux $\Phi_E$ is the ``flow" of the 
electric field through a surface. 
It is proportional to the (net) number of 
electric field lines passing through the surface.

\i For a uniform electric field $\vec E$ and 
flat surface $S$ with area $A$ and unit normal
$\hat n$,
%
\be
\Phi_E = EA\cos\theta = E_\perp A = E A_\perp
\ee
%
where $\theta$ is the angle between $\vec E$ and
$\hat n$.

\i $E_\perp$ is the component of $\vec E$ perpendicular
to the surface (so in the same direction as $\hat n$);
$A_\perp$ is the component of the area perpendicular to 
the electric field lines.

\i If the surface $S$ is curved or the electric field
$\vec E$ is non-uniform, we must sum up the 
contributions associated with infinitesimal area elements 
$d\vec A = \hat n\, dA$:
%
\be
\Phi_E = \int_S \vec E\cdot\hat n\, dA
\ee
 
\i For a closed surface $S$, $\hat n$ is taken to be the 
{\em outward-pointing} normal.

\ei
%%%%%%%%%%%%%%%%%%%%%%%%%%%%%%%%
\subsection{Gauss's law}
\bi

\i {\em Gauss's law}:

\be
\oint_S \vec E\cdot \hat n\,dA = \frac{q_{\rm enc}}{\epsilon_0} 
\ee
%
where $q_{\rm enc}$ is the (net) electric charge enclosed by
the surface $S$.

\i If $q_{\rm enc}=0$, the flux of the electric field through
$S$ is zero.

\i It is easy to ``prove" Gauss's law  for a single point 
charge $q$ enclosed by a spherical surface centered on $q$.

(i) The result is independent of the radius of the sphere,
since the surface area of the sphere is proportional to 
$r^2$, while the magnitude of the electric field falls-off
like $1/r^2$.

(ii) The result is also independent of the shape of the 
surface enclosing $q$, since 
%
\be
\hat r\cdot \hat n\,dA
=r^2 \sin\theta\, d\theta\, d\phi
\ee
for any area element $\hat n\, dA$ (i.e., the surface need
not be a sphere centered on the charge).

\i For something more complicated than a single point charge, 
use the superposition principle to calculate $\vec E$; the
RHS is the (net) sum of the charges making up the 
distribution that are enclosed by the surface $S$.

\ei

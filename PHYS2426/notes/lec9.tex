\section{Current and resistance}

\subsection{Current}
\bi

\i Current is the rate of flow of charge through
some surface:
\be
I_{\rm avg} = \Delta Q/\Delta t
\quad{\rm or}\quad
I = dQ/dt\quad({\rm instantaneous})
\ee
where $\Delta Q$ is the net charge that crosses 
the surface (area $A$) in time interval $\Delta t$.

\i The unit of current is an amp, 
$1~{\rm A} \equiv 1~{\rm C}/{\rm s}$.
The direction of current is the direction
of flow of the positive charges.
Current density $J$ is defined as
$J\equiv I/A$.

\i For charges to flow in a conductor, we need a 
non-zero electric field inside the conductor.  
(We are no longer dealing with electrostatics.)
The non-zero field creates a potential difference
$\Delta V$ between two ends of the conductor.

\ei

\subsection{Microscopic model of current flow}
\bi

\i Consider a cylindrical conductor with 
cross-sectional area $A$, charge carriers with 
number density $n$, each with charge $q$.

\i In the absence of an applied electric field,
there is no net flow of the charge carriers
(e.g., free electrons in the conductor move in 
random directions with velocity $\vec v_{\rm th}$
due to thermal effects).

\i In the presence of an applied electric field,
the charge carriers accelerate, acquiring a 
drift velocity $v_{\rm d}$, which is several orders 
of magnitude smaller than the thermal velocity
$v_{\rm th}$. (NOTE: $v_{\rm th} \sim 10^6~{\rm m}/{s}$
for electrons at room temperature.)

\i Take the length of the cylinder to be 
$\Delta x = v_{\rm d}\Delta t$,
so that in time interval $\Delta t$ all of the charges 
in the cylinder have passed through one end of the cylinder.
Then
%
\be
I_{\rm avg} = \frac{\Delta Q}{\Delta t}
= \frac{q n A \Delta x}{\Delta t}= qnA v_{\rm d}
\quad\Rightarrow\quad
J= qnv_{\rm d}
\ee
 
\i Example 27.4:
Calculate the drift velocity of free
electrons in a copper wire with cross-sectional area
$A=3.3\times 10^{-6}~{\rm m}^2$, $I=10~{\rm A}$,
mass density $\rho=8.92~{\rm g}/{\rm cm}^3$.
Assume that there is one free electron per 
copper atom.
Also, recall that the molar mass of copper is
$M=63.5~{\rm g}/{\rm mole}$, and $N_A=6.02\times 10^{23}$
is the number of of atoms/mole.
The result is $v_{\rm d}=2.23\times 10^{-4}~{\rm m}/{\rm s}$,
which means that it take more than 1~hr for a free
electron to travel 1~meter!

\i So it's not the speed of the free electrons that 
make a light turn on almost instantaneously when you 
flip a switch.
Rather it's the electrostatic force between neighboring
electrons which ``smooths" out their distribution, so
that there isn't an excess 
or lack of charge carriers in any part of the circuit.

\ei

\subsection{Resistance}
\bi

\i The drift velocity of charges carriers is produced by
a force.   
If that force is proportional to the applied electric
field, then the current density $J =qnv_{\rm d}$ is 
proportional to the field, 
$\vec J =\sigma \vec E$, where
$\sigma$ is the {\em conductivity} of the conductor
(property of the material).

\i Materials for which $\vec J = \sigma\vec E$ are said to 
obey {\em Ohm's law}.
(NOTE: Ohm's law really isn't a law; it's an empirical 
property of certain materials.)

\i For a cylindrical piece of conductor with conductivity
$\sigma$, cross-sectional area $A$, length $l$, and applied
field $\vec E$, we have
%
\be
J=\sigma E
\quad\Leftrightarrow\quad
I/A=\sigma\Delta V/l
\quad\Leftrightarrow\quad
R= \rho l/A
\ee
%
where $R\equiv \Delta V/I$ is the {\em resistance} of 
object and $\rho\equiv 1/\sigma$ is the {\em resisitivity}
of the material.

\i $R$ has units of ohms, 
$1~\Omega \equiv 1~{\rm V}/{\rm A}$,
and $\rho$ has units of $\Omega\cdot{\rm m}$.
Typical values of $\rho$ are
$\sim 10^{-8}~\Omega\cdot{\rm m}$ for metals, 
$\sim 10^{13}~\Omega\cdot{\rm m}$ for insulators,
and $\sim 1$ to $10^3$ for semiconductors.

\ei


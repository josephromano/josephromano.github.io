\section{Electrostatic potential}

\subsection{Work and potential energy}

\bi

\i Recall: The work done by a force in moving a particle
from point $A$ to point $B$ along some path is given
by 
%
\be
W_{A\rightarrow B} = \int_A^B \vec F\cdot d\vec s
\ee
%where $d\vec s$ is the infinitesimal displacement vector
%along the path.

\i For a general force, the value of the integral depends
on the path connecting $A$ and $B$.
If the force is {\em conservative}, then the integral
is path independent and depends only on the endpoints.

\i Examples of conservative forces are gravity, the 
spring force, and the electrostatic force.
An example of a non-conservative force is friction.

\i For conservative forces, one can a associate a 
{\em potential energy} $U$ with the work done by 
the force in moving the particle from $A$ to $B$:
%
\be
\Delta U\equiv U_B- U_A = -W_{A\rightarrow B} = 
-\int_A^B \vec F\cdot d\vec s
\ee

\i Note the minus sign in the above expression.
The potential energy of the system decreases if the 
force does positive work.
For example, as a mass $m$ falls in a gravitational
field $\vec g$, it loses potential energy, while gaining
kinetic energy.

\ei

%%%%%%%%%%%%%%%%%%%%%%%%%%%%%%%%
\subsection{Electrostatic potential}

\bi

\i The above results apply to the electrostatic force 
acting on a charge $q$.  
We define
%
\be
\Delta U = -\int_A^B \vec F_e\cdot d\vec s
= -q\int_A^B \vec E\cdot d\vec s
\ee

\i The electrostatic potential $V$ is then defined as 
the electrostatic potential energy per unit charge:
\be
\Delta V \equiv \frac{\Delta U}{q}
= -\int_A^B \vec E\cdot d\vec s
\ee
%
which has units of Joules per Coulomb, or volts
($1~{\rm volt} \equiv 1~{\rm J}/{\rm C}$).


\i Note that changing the electrostatic potential
$V$ by an additive constant ($V'= V+{\rm const}$) 
does not change the potential difference, since 
$\Delta V = \Delta V'$.
So the zero (or reference point) of the 
electrostatic potential can be set in whatever way
is most convenient for the problem at hand.
%(We will see that for point charges, the reference
%point is taken to be infinity.)

\i If $\vec E$ is everywhere orthogonal to $d\vec s$,
then $\Delta V=0$.
This means that the electric field $\vec E$ is 
{\em perpendicular} to an
{\em equipotential} surface.

\ei
%%%%%%%%%%%%%%%%
\subsection{Obtaining $\vec E$ from $V$}
\bi

\i The electrostatic potential $V$ is given by integrating
the electric field $\vec E$.
Conversely, by considering two nearby points $A$ and $B$, 
one can 
show that the electric field $\vec E$ is given by 
differentiating $V$:
%
\be
E_x = -\frac{\partial V}{\partial x}\,,\quad
E_y = -\frac{\partial V}{\partial y}\,,\quad
E_z = -\frac{\partial V}{\partial z}
\quad\Leftrightarrow\quad
\vec E=-\grad V
\ee
%

\i This relationship is very useful since it is often much
easier to first calculate the electrostatic potential $V$
for a distribution of charges, and then differentiate $V$
to obtain the electric field $\vec E$.

\i Recall that it is typically hard to calculate $\vec E$
for a distribution of charges since one has to sum up
{\em vectors} for each charge in the distribution.
Summing up scalars (for the potential $V$) is much easier.

\i Note that the units of $\vec E$ can also be expressed
in terms of volt per meter:
$1~{\rm N}/{\rm C} = 1~{\rm volt}/{\rm m}$.

\ei

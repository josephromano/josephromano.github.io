\section{Ray optics (continued)}

%%%%%%%%%%%%%%%%%%%%%%%%%%%%%%%%%%%%%%%%%%%%%%%%
\subsection{Refraction}
\bi

\i In addition to light rays reflecting off
of a surface between two media, they can also
be transmitted from one medium into the other.

\i If $v_{1,2}$ denote the speeds of light in
the two media, then the law of refraction
(called Snell's law) is
%
\be
\frac{\sin\theta_1}{v_1} =
\frac{\sin\theta_2}{v_2} 
\qquad
{\rm or,\ equivalently}
\qquad
n_1\sin\theta_1 = n_2\sin\theta_2
\ee
%
where $\theta_{1,2}$ are the angles that
the rays of light make with the 
normal to the surface between media 1,2.

\i The second equation is expressed in terms 
of {\em indices of refraction} 
\be
n\equiv c/v
\ee
for the two media.
Since light travels fastest in vacuum, $n\ge 1$.
For air, $n\approx 1.000293$;
for water $n\approx 1.33$;
for (crown) glass, $n\approx 1.5$.

\i One can prove Snell's law using Fermat's principle
of least time, similar to the proof for the law of
reflection.
(Lifeguard-and-drowning-swimmer analogy.)

\i As light travels from a medium with a lower index
of refraction to one with a higher index of refraction
($n_1<n_2$), the rays are bent {\em toward} the normal
(opposite for $n_1>n_2$).

\i Demo: A pencil appears to be bent if observed obliquely
in a glass of water.

\ei

%%%%%%%%%%%%%%%%%%%%%%%%%%%%%%%%%%%%%%%%%%%%%%%%
\subsection{Dispersion}
\bi

\i Dispersion refers to the fact that the index of
refraction of a material depends, in general, on 
the wavelength of the light propagating through it.

\i Demo: ``White" light passing through a prism is
split into its spectrum of colors (violet light 
is refracted through a 
greater angle than red light, since violet light
has a larger index of refraction).

\i Dispersion is responsible for the production 
of a {\em rainbow}  as ``white" light from the Sun
passes through a rain drop.
(The light is actually refracted, then reflected, 
and then refracted again by the rain drop, causing the 
red band of the rainbow to be above the violet band).

\i A {\em secondary} rainbow is formed when the light
is reflected twice by the back surface of the rain
drop.

\ei
%%%%%%%%%%%%%%%%%%%%%%%%%%%%%%%%%%%%%%%%%%%%%%%%
\subsection{Total internal reflection}
\bi

\i Total internal reflection occurs when light
propagates from a medium with higher index of
refraction to one with lower index of refraction 
($n_1>n_2$), provided the incident angle is 
greater than the {\em critical angle} $\theta_{\rm c}$
defined by
%
\be
\sin\theta_{\rm c} \equiv n_2/n_1
\ee

\i Note that for $\theta_1=\theta_c$, the 
refracted angle $\theta_2 = 90^\circ$, moving along the
surface between the two media.

\i Light can be made to follow the curved trajectory
of a plastic fiber, provided the curvature of the fiber
is not too sharp.
This is the basic principle underlying light transfer
using {\em fiber optics}.

\i Fiber optics has applications for both the 
telecommunications industry and medical devices.

\ei

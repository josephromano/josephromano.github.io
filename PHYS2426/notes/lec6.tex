\section{Electrostatic potential (continued)}

\subsection{Examples of electrostatic potentials}

\bi

\i For a uniform electric field $\vec E$, we have
$\Delta V = - Ed<0$ if we integrate in the same 
direction as $\vec E$.
Thus, for $q>0$, the electrostatic potential energy
{\em decreases} as you move in the direction of the 
field.

\i For a point charge $q$ at the origin,
$V(r) = k_e q/r$, where we have chosen zero potential 
at $r\rightarrow\infty$.

\i For a set of discrete point charges or a continous
charge distribution:
%
\be
V(P) = \sum_i \frac{k_e q_i}{r_i}
\quad({\rm discrete})
\,,\qquad
V(P) = \int \frac{k_e dq}{r}
\quad({\rm continuous})
 \ee

\i There are many worked examples in the text for 
finding $V(P)$ for different charge distributions.

\ei

%%%%%%%%%%%%%%%%%%%%%%%%%%%%%%%%%%%%%%%%%%%
\subsection{Electrostatic potential energy}
\bi

\i Consider three charges $q_1$, $q_2$, $q_3$ separated by 
distances $r_{12}$, $r_{13}$, $r_{23}$.
The electrostatic potential energy associated with this
system of charges can be found by calculating how much
work an external agent has to do to ``build" the configuration
by bringing in charges one-by-one from infinity.

\i You first bring $q_1$ to its position against no field;
then $q_2$ to its position against the field created by
$q_1$; and finally $q_3$ to its position against the 
field created by $q_1$ and $q_2$:
%
\be
W =\frac{k_e q_1 q_2}{r_{12}} + \frac{k_e q_1 q_3}{r_{13}}
+ \frac{k_e q_2 q_3}{r_{23}}
\ee

\ei
%%%%%%%%%%%%%%%%%%%%%%%%%%%%%%%%%%%%%%%%%%%
\subsection{Electrostatic potential associated with conductors}
\bi

\i Since $\vec E=\vec 0$ everywhere inside a conductor,
$\Delta V=-\int_A^B \vec E\cdot d\vec s=0$ for any path lying
within the conductor.
Thus, a conductor is an {\em equipotential}.

\i If a conductor has an empty cavity, then $\vec E=\vec 0$
inside the cavity (as well as inside the ``meat" of the 
conductor).  {\em Proof}:
If that wasn't the case, then one could integrate
$\vec E$ along one of its field lines inside the cavity,
from one side of the cavity to the other,
giving $\Delta V\ne 0$ or $V_B\ne V_A$, contradicting that
a conductor is an equipotential.

\i The surface charge density on the surface of a conductor
has its greatest magnitude where the radius of curvature is
smallest.  
{\em Proof}: Model your conductor as two conducting spheres 
with radii $r_1$, $r_2$ (with $r_1>r_2$) 
and charge $q_1$, $q_2$ connected by a metal wire.
Since this configuration is an equipotential, $k_e q_1/r_1 = k_e q_2/r_2$,
which implies $q_2/q_1 = r_2/r_1$.
It then follows that 
%
\be
\frac{\sigma_2}{\sigma_1} 
= \frac{q_2}{4\pi r_2^2}\bigg/\frac{q_1}{4\pi r_1^2}
= \frac{q_2}{q_1}\frac{r_1^2}{r_2^2}
= \frac{r_2}{r_1}\frac{r_1^2}{r_2^2}
= \frac{r_1}{r_2} > 1
\ee

\ei
%%%%%%%%%%%%%%%%%%%%%%%%%%%%%%%%%%%%%%%%%%%%%%
\subsection{Demonstrations}
\bi

\i An uncharged (metallic-coated) pith ball is 
attracted to the charged surface of a van de Graaff generator.
When it hits the surface of the generator, it becomes
charged by conduction and is then repelled.

\i We see the same behavior outside the outer surface of a 
charged metallic Faraday cage.

\i Inside the charged Faraday cage, the electric field is 
zero, so an uncharged pith ball feels no force inside.

\i If instead we charge the inner cylindrical surface of the 
Faraday cage, and place an uncharged pith ball 
between the inner and outer cylindrical surfaces, then we 
find that the pith ball is first attracted and then 
repelled by the inner cylinder, eventually hitting the outer 
cylinder and bouncing back, oscillating wildly.

\ei

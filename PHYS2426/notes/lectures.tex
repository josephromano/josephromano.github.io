\documentclass[10pt,twoside]{article}
\usepackage{amssymb,amsmath,amsthm,longtable}
\usepackage{latexsym}
\usepackage{graphicx}
\usepackage{hyperref}
\usepackage[top=1in, bottom=1in, left=1in, right=1in]{geometry}

% some definitions

\def\cross{\times}
\def\del{\nabla}
\def\grad{\vec\nabla}
\def\div{\grad\cdot}
\def\curl{\grad\cross}
\def\I{I\!\!\!-}
 
% begin equation, itemize, etc.

\def\be{\begin{equation}}
\def\ee{\end{equation}}
\def\bi{\begin{itemize}}
\def\ei{\end{itemize}}
\def\ben{\begin{enumerate}}
\def\een{\end{enumerate}}
\def\i{\item{}}

\def\defn{\underline{Definition}:\ }
\def\ex{\underline{Example}:\ }
\def\exer{\underline{Exercise}:\ }
\def\soln{\underline{Solution}:\ }
\def\theor{\underline{Theorem}:\ }
\def\pf{\underline{Proof}:\ }
\def\ques{\underline{Question}:\ }
\def\ans{\underline{Answer}:\ }
 
%%%%%%%%%%%%%%%%%%%%%%%%%%%%%%%%%%%%%%%%%%%%%%%%%%%%%%%%%%%%%%
            
\begin{document}

\title{Physics for Scientists and Engineers II \\
Lecture summaries}
%\author{J.D.\ Romano}
\author{DRAFT}
\date{Spring 2026}
\maketitle
 
\begin{abstract}
\noindent
{\bf Disclaimer:} 
These notes {\em summarize} the key concepts and 
formulae discussed during lecture.
They should be supplemented by any standard calculus-based
physics textbook on these topics (e.g., by Serway and Jewett) 
or internet resource to fill in any gaps.
Please send any comments, criticisms, suggestions to:
{\tt joseph.d.romano@gmail.com}.

\end{abstract}

\newpage
\tableofcontents
\cleardoublepage

\input lec1
\newpage

\input lec2
\newpage

\input lec3
\newpage

\input lec4
\newpage

\input lec5
\newpage

\input lec6
\newpage

\input lec7
\newpage

\input lec8
\newpage

\input lec9
\newpage

\input lec10
\newpage

\input lec11
\newpage

\input lec12
\newpage

\input lec13
\newpage

\input lec14
\newpage

\input lec15
\newpage

\input lec16
\newpage

\input lec17
\newpage

\input lec18
\newpage

\input lec19
\newpage

\input lec20
\newpage

\input lec21
\newpage

\input lec22
\newpage

\input lec23
\newpage

\input lec24
\newpage

\input lec25

\end{document}


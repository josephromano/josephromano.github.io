\documentclass[10pt]{article}
\usepackage{amssymb,amsmath,amsthm,longtable}
\usepackage{latexsym}
\usepackage{placeins}
\usepackage{graphicx}
\usepackage{caption}
\usepackage{textcomp}
\captionsetup{width=5.75in}
\usepackage[top=1in, bottom=1in, left=1in, right=1in]{geometry}

\numberwithin{equation}{section}
\usepackage{amsfonts}
\usepackage{upgreek}
\usepackage{bm}
\usepackage{dsfont}
\usepackage{dcolumn}
\usepackage{epsfig}
\usepackage{graphics}
\usepackage{subfigure}

% begin equation, itemize, etc.

\def\be{\begin{equation}}
\def\ee{\end{equation}}
\def\bi{\begin{itemize}}
\def\ei{\end{itemize}}
\def\ben{\begin{enumerate}}
\def\een{\end{enumerate}}
\def\i{\item{}}
\newcommand{\bs}[1]{\boldsymbol{#1}}
\newcommand{\mb}[1]{\mathbf{#1}}
\renewcommand{\vec}[1]{\mathbf{#1}}
\newcommand{\vecs}[1]{\bs{#1}}
\def\d{{\rm{d}}}

\begin{document}

%%%%%%%%%%%%%%%%%%%%%%%%%%%%%%%%%%%%%%%%%%%%%%%%%%%%%%%%%%%%%%%%%%%%%%
\section*{Info regarding prelim problems for ``Classical Dynamics"}

Prelim problems will be chosen from the following list of topics, 
at a level comparable to the problems in ``Classical Mechanics" by 
Goldstein, Poole, and Safko.  
An example problem (taken from a past prelim) is indicated in 
square brackets.

\ben

\i Lagrangian mechanics:
Write down the Lagrangian for a particle or system of particles,
possibly subject to constraints.
Obtain the equations of motion for the system and solve those 
equations for simple cases that allow for analytic solutions or
 integral expressions for the generalized coordinates.
Use the method of Lagrange multipliers to solve for the constraint 
force if desired.
Determine from the form of the Lagrangian whether energy, momentum, 
and/or angular momentum are conserved.
[D1P1 Jan 2005]
 
\i Hamiltonian mechanics:
Know how to obtain the Hamiltonian from a Lagrangian, as well as 
Hamilton's equations for the generalized coordinates and momenta.
[D2P3 Summer 2019]

\i Central force motion:
Know how to solve for motion of a system of two particles subject to
a central force, using conservation of angular momentum and energy,
and the concept of an effective potential to simplify the analysis.
Understand in detail the solution to the central force problem for an 
inverse square-law force (i.e., Kepler's problem), knowing e.g.,
the relationship between the conserved energy and angular momentum 
of the system with parameters describing an ellipse for a bound orbit.
[D2P1 Summer 2020; D2P2 Summer 2019]

\i Collisions and scattering:
Use conservation of energy and momentum to solve for the motion of
two particles undergoing an elastic collision, given the scattering
angle with resect to the center-of-mass frame.
Be able to transform back and forth between scattering angles calculated 
with respect to the lab and center-of-mass frames.
Know how to calculate the differential cross-section for central-force
scattering of one particle off of another, given the form of the 
central potential and the energy and angular momentum of the system.
[D1P2 Jan 2010]

\i Small oscillations:
Calculate the frequency for small oscillations for a system about a 
stable equilibrium configuration.
Calculate the normal mode frequencies and normal mode 
oscillations for systems having more than one degree of freedom.
[D2P3 Aug 2018]

\i Rigid body motion:
Calculate the kinetic energy and angular momentum of a rigid body
in terms of its principle moments of inertia.
Calculate the components of the angular velocity vector in terms
of the Euler angles and their time derivatives.
Obtain and solve the equations rigid body for some simple examples.
[D1P1 Spring 2011]

\i Non-inertial reference frames:
Obtain the equations of motion for a particle moving in a non-inertial reference
frame, and solve for these equations for simple scenarios.
[D2P2 Summer 2020]

\een

\end{document}

\documentclass[11pt]{NSF}
\sloppy

\usepackage{latexsym}
\usepackage{graphicx}
\usepackage{draftcopy}
\usepackage{longtable}
\usepackage{hyperref}

% some definitions
 
% begin equation, itemize, etc.

\def\be{\begin{equation}}
\def\ee{\end{equation}}
\def\bi{\begin{itemize}}
\def\ei{\end{itemize}}
\def\ben{\begin{enumerate}}
\def\een{\end{enumerate}}
\def\i{\item{}}

%%%%%%%%%%%%%%%%%%%%%%%%%%%%%%%%%%%%%%%%%%%%%%%%%%%%%%%%%%%%%%
            
\begin{document}

\begin{center}
COURSE SYLLABUS\\
PHYS 5306: ``Classical Dynamics"\\
Fall 2022\\
\end{center}

{\bf Instructor:}
Joseph D. Romano\\
Office: Science 14, Phone: 806-834-6522, E-mail: joseph.d.romano@ttu.edu

{\bf Lectures:}
PHYS 5306: TTh 12:30~PM - 1:50~PM, (SCI 204)

{\bf Office hours:}
Virtual (preferred) and in-person office hours by appointment.

{\bf Recommended textbooks:}\\
``Mechanics" (3rd edition) by L.D.~Landau and E.M.~Lifshitz\\
``Classical Mechanics" by M.J.~Benacquista and J.D.~Romano

{\bf Course Description:}
Lagrangian dynamics and variational principles. Kinematics and dynamics of
two-body scattering. Rigid body dynamics. Hamiltonian dynamics, canonical
transformations, and Hamilton-Jacobi theory of discrete and continuous systems.
M.S.\ and Ph.D.\ core course.  

{\bf Learning Outcomes:}
Students should be able to thoroughly understand the concepts and
methods of classical mechanics. Students are expected to be
able to apply key physics principles to explain and solve problems,
at the graduate level, in this area of physics.

{\bf Outcome assessment:} 
The expected course outcomes will be assessed through problem solving 
in class and on exams.

{\bf Course website:}
\url{https://josephromano.github.io/PHYS5306/} with announcements
circulated via Blackboard \url{https://ttu.blackboard.com/}.

{\bf Problem notebook}:
Solving problems is probably the most important part of this course.
Although you won't have graded homework assignments, you will 
work on suggested problems throughout the semester, very often
during class time with your fellow students.
You should purchase a 3-ring binder to keep the solutions to 
all the problems you solve for this course.

{\bf Quizzes}:
There will be weekly quizzes at the start of class to make sure 
that you are up-to-date with the material.
No make-up quizzes will be given.

{\bf Exams}:
There will be two written midterm exams and a comprehensive 
oral final (see `Course Calendar' for dates).
No make-up examinations will be given. In the case of a
serious emergency or illness, please contact me to discuss how 
the final grade will be determined.

{\bf Grading:}\\
Problem notebook: 20\%\\
Quizzes (best 10 grades): 20\%\\ 
Midterm x 2: 40\%\\
Final (oral): 20\%\\ 
Total: 100\%\\
Number grades will be converted to letter grades as follows:\\
A: 100-80; B: 79-60; XX: 59-50; C: 49-30; F: 29-0\\ 
with the XX borderline case determined primarily by class participation.

{\bf REQUIRED AND RECOMMENDED SYLLABUS LANGUAGE:}

\url{https://www.depts.ttu.edu/tlpdc/RequiredSyllabusStatements.php}

\url{https://www.depts.ttu.edu/tlpdc/RecommendedSyllabusStatements.php}

\end{document}


\documentclass[11pt]{NSF}
\sloppy

\usepackage{latexsym}
\usepackage{graphicx}
\usepackage{draftcopy}
\usepackage{longtable}
\usepackage{hyperref}

% some definitions
 
% begin equation, itemize, etc.

\def\be{\begin{equation}}
\def\ee{\end{equation}}
\def\bi{\begin{itemize}}
\def\ei{\end{itemize}}
\def\ben{\begin{enumerate}}
\def\een{\end{enumerate}}
\def\i{\item{}}

%%%%%%%%%%%%%%%%%%%%%%%%%%%%%%%%%%%%%%%%%%%%%%%%%%%%%%%%%%%%%%
            
\begin{document}

\begin{center}
COURSE SYLLABUS\\
PHYS 5306: ``Classical Dynamics"\\
Fall 2021\\
\end{center}

{\bf Instructor:}
Joseph D. Romano\\
Office: Science 14, Phone: 806-834-6522, E-mail: joseph.d.romano@ttu.edu

{\bf Lectures:}
PHYS 5306: TTh 12:30~PM - 1:50~PM, (EDUC 370)

{\bf Office hours:}
Virtual (preferred) and in-person office hours by appointment.

{\bf Recommended textbooks:}\\
``Mechanics" (3rd edition) by L.D.~Landau and E.M.~Lifshitz\\
``Classical Mechanics" by M.J.~Benacquista and J.D.~Romano

{\bf Course Description:}
Lagrangian dynamics and variational principles. Kinematics and dynamics of
two-body scattering. Rigid body dynamics. Hamiltonian dynamics, canonical
transformations, and Hamilton-Jacobi theory of discrete and continuous systems.
M.S.\ and Ph.D.\ core course.  

{\bf Learning Outcomes:}
Students should be able to thoroughly understand the concepts and
methods of classical mechanics. Students are expected to be
able to apply key physics principles to explain and solve problems,
at the graduate level, in this area of physics.

{\bf Outcome assessment:} 
The expected course outcomes will be assessed through problem solving 
in class and on exams.

{\bf Course website:}
\url{https://josephromano.github.io/PHYS5306/} with announcements
circulated via Blackboard \url{https://ttu.blackboard.com/}.

{\bf Class participation}:
Solving problems is probably the most important part of 
this course.
Although you won't have graded homework assignments, you will 
work on suggested problems throughout the semester.
There will be unannounced ``pop" quizzes in class from time to time 
to make sure that you are up-to-date with the material.

{\bf Exams}:
There will be two written midterm exams and a comprehensive 
oral final (see `Course Calendar' for dates).
No make-up examinations will be given. In the case of a
serious emergency or illness, please contact me to discuss how 
the final grade will be determined.

{\bf Grading:}\\
Class participation: 20\%\\ 
Midterm x 2: 60\%\\
Final (oral): 20\%\\ 
Total: 100\%\\
Number grades will be converted to letter grades as follows:\\
A: 100-80; B: 79-60; XX: 59-50; C: 49-30; F: 29-0\\ 
with the XX borderline case determined primarily by class participation.

{\bf REQUIRED SYLLABUS LANGUAGE:}

{\bf Vaccinations:} Texas Tech University strongly recommends students
adhere to CDC guidelines on COVID-19, including obtaining COVID-19
vaccinations. If you were unable to obtain a vaccination prior to your arrival
on campus, the COVID-19 vaccine is available at Student Health Services by
appointment. You can find additional information about the vaccine here, and
about the recently announced incentive program here.

{\bf Face Covering Policy:} As of May 19, 2021, face coverings are optional
in TTU facilities and classrooms but, based on CDC guidelines, are recommended
and welcome, especially for those who have not been vaccinated for COVID-19 or
who may have susceptibilities to the virus. Face coverings are required in
public transportation (e.g., Citibus) and in the Student Health Clinic.

{\bf Seating Charts and Social Distancing:} There is no longer a mandated
social distancing protocol for classroom seating, but using a seating chart and
taking attendance are recommended in support of campus contact tracers if
needed. Social distancing is recommended in rooms that will enable it.

{\bf Illness-related Absences:} In general student absences due to illness are
to be considered as they were prior to the pandemic, with consideration given
to the fact that students who are isolating with COVID-19 and students who are
quarantining for symptoms or direct exposure may have extended days of absence.
Thus, it is to everybody's advantage to stay healthy and avoid getting sick or
getting other people sick.  Guidance for students is provided at {\url
https://ttucovid19.ttu.edu/User/Consent.}

{\bf In-Person Office Hours:} In-person office hours will be provided upon
request if a virtual (online) meeting is not feasible, for whatever reason.
For in-person office visits, I strongly recommend that students wear masks and
practice social distancing to help reduce the spread of COVID-19.  Outdoor
meeting venues (weather permitting) is also an option.

{\bf Personal Hygiene:} We all should continue to practice frequent hand
washing, use hand sanitizers after touching high-touch points (e.g., door
handles, shared keyboards, etc.), and cover faces when coughing or sneezing.

{\bf Potential Changes:} The University will continue to monitor CDC, State,
and TTU System guidelines in continuing to manage the campus implications of
COVID-19. Any changes affecting class policies or delivery modality will be in
accordance with those guidelines and announced as soon as possible. If Texas
Tech University campus operations are required to change because of health
concerns related to the COVID-19 pandemic, it is possible that this course will
move to a fully online delivery format. Should that be necessary, students will
be advised of technical and equipment requirements, such as web cam,
microphone, and remote proctoring software.

{\bf Academic Honesty (OP 34.12)}:
It is the aim of the faculty of Texas Tech University to foster a
spirit of complete honesty and high standard of integrity. The attempt
of students to present as their own any work not honestly performed is
regarded by the faculty and administration as a most serious offense
and renders the offenders liable to serious consequences, possibly
suspension. 

``Scholastic dishonesty” includes, but it not limited to,
cheating, plagiarism, collusion, falsifying academic records,
misrepresenting facts, and any act designed to give unfair academic
advantage to the student (such as, but not limited to, submission of
essentially the same written assignment for two courses without the
prior permission of the instructor) or the attempt to commit such an
act.  

The full policy is available at
\url{http://www.depts.ttu.edu/opmanual/OP34.12.pdf} 

{\bf Special Accommodation for Students with Disabilities 
(OP 34.22):} 
Any student who, because of
a disability, may require special arrangements in order to meet the
course requirements should contact the instructor as soon as possible
to make any necessary arrangements. Students should present
appropriate verification from Student Disability Services during the
instructor's office hours. Please note: instructors are not allowed to
provide classroom accommodations to a student until appropriate
verification from Student Disability Services has been provided. For
additional information, please contact Student Disability Services in
West Hall or call 806-742-2405.  

The full policy is available at
\url{http://www.depts.ttu.edu/opmanual/OP34.22.pdf}

{\bf Student Absence for
Observance of a Religious Holy Day (OP 34.19):}

\ben
\i “Religious holy day”
means a holy day observed by a religion whose places of worship are
exempt from property taxation under Texas Tax Code §11.20.  

\i A student who intends to observe a religious holy day should make that
intention known in writing to the instructor prior to the absence. A
student who is absent from classes for the observance of a religious
holy day shall be allowed to take an examination or complete an
assignment scheduled for that day within a reasonable time after the
absence.  

\i A student who is excused under section 2 may not be
penalized for the absence; however, the instructor may respond
appropriately if the student fails to complete the assignment
satisfactorily.  
\een

The full policy is available at
\url{http://www.depts.ttu.edu/opmanual/OP34.19.pdf}

{\bf TTU Resources for Discrimination, Harassment, and Sexual
Violence:}
Texas Tech University is committed to providing and strengthening an
educational, working, and living environment where students, faculty,
staff, and visitors are free from gender and/or sex discrimination of
any kind. Sexual assault, discrimination, harassment, and other Title
IX violations are not tolerated by the University. 

Report any
incidents to the Office of Student Rights \& Resolution, 806-742-SAFE
(7233) or file a report online at 
\url{http://www.depts.ttu.edu/titleix/}. 

Faculty and staff
members at TTU are committed to connecting you to resources on campus.
Some of these available resources are: 

\bi
\i {\bf TTU Student Counseling Center}\\
Phone: 806-742-3674\\
Website: \url{https://www.depts.ttu.edu/scc/}\\
(Provides confidential support on campus.) 

\i {\bf TTU 24-hour Crisis Helpline}\\
Phone: 806-742-5555\\
(Assists students who are experiencing a mental health or
interpersonal violence crisis. If you call the helpline, you will
speak with a mental health counselor.) 

\i {\bf Voice of Hope Lubbock Rape Crisis Center}\\
Phone: 806-742-7273\\
Website: \url{http:voiceofhopelubbock.org}\\
(24-hour hotline
that provides support for survivors of sexual violence.) 

\i {\bf The Risk, Intervention, Safety and Education (RISE) Office}\\
Phone: 806-742-2110\\
Website: \url{http://www.depts.ttu.edu/rise/}\\
(Provides a range of resources and support options
focused on prevention, education, and student wellness.) 

\i {\bf Texas Tech Police Department}\\
Phone: 806-742-3931\\ 
Website: \url{http://www.depts.ttu.edu/ttpd/}\\ 
(To report criminal activity that occurs on or near Texas Tech campus.)

\i {\bf LGBTQIA:}
Within the Center for Campus Life, the Office of LGBTQIA
serves the Texas Tech community through facilitation and leadership of
programming and advocacy efforts. This work is aimed at strengthening
the lesbian, gay, bisexual, transgender, queer, intersex, and asexual
(LGBTQIA) community and sustaining an inclusive campus that welcomes
people of all sexual orientations, gender identities, and gender
expressions.

\ei
\end{document}


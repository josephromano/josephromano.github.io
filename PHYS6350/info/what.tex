\documentclass[10pt]{article}
\usepackage{amssymb,amsmath,amsthm,longtable}
\usepackage{latexsym}
\usepackage{placeins}
\usepackage{graphicx}
\usepackage{caption}
\usepackage{textcomp}
\captionsetup{width=5.75in}
\usepackage[top=1in, bottom=1in, left=1in, right=1in]{geometry}

\numberwithin{equation}{section}
\usepackage{amsfonts}
\usepackage{upgreek}
\usepackage{bm}
\usepackage{dsfont}
\usepackage{dcolumn}
\usepackage{epsfig}
\usepackage{graphics}
\usepackage{subfigure}

% begin equation, itemize, etc.

\def\be{\begin{equation}}
\def\ee{\end{equation}}
\def\bi{\begin{itemize}}
\def\ei{\end{itemize}}
\def\ben{\begin{enumerate}}
\def\een{\end{enumerate}}
\def\i{\item{}}
\newcommand{\bs}[1]{\boldsymbol{#1}}
\newcommand{\mb}[1]{\mathbf{#1}}
\renewcommand{\vec}[1]{\mathbf{#1}}
\newcommand{\vecs}[1]{\bs{#1}}
\def\d{{\rm{d}}}

\begin{document}

%%%%%%%%%%%%%%%%%%%%%%%%%%%%%%%%%%%%%%%%%%%%%%%%%%%%%%%%%%%%%%%%%%%%%%
\title{What should you know how to do by the end of this course?}

\maketitle
\date{}

%\tableofcontents

%%%%%%%%%%%%%%%%%%%%%%%%%%%%%%%%%%%%%%%%%%%%%%%%%%%%%%%%%%%%%%%%%%%%%%
\section{Lagrangian mechanics (\S1-5)}

\ben

\i Write down the Lagrangian for a simple system in terms of generalized coordinates.

\i Distinguish generalized coordinates from Cartesian coordinates.

\i Write down Lagrange's equations.

\i Define the action in terms of the Lagrangian, and derive Lagrange's equations starting from the action.

\i Show that Lagrange's equations are unchanged if one adds a total time derivative $d f(q,t)/d t$ to $L$.

\i Include holonomic and non-holonomic constraint forces in the Lagrangian formalism by introducing Lagrange multipliers.

\i Define and give examples of a {\em closed system}, {\em constant
external field}, and {\em uniform field}.

\een

%%%%%%%%%%%%%%%%%%%%%%%%%%%%%%%%%%%%%%%%%%%%%%%%%%%%%%%%%%%%%%%%%%%%%%
\section{Conservation laws (\S6-10)}

\ben

\i Show how conservation of energy, momentum, and angular momentum are connected to time translation, space translation, and rotational symmetry.

\i Derive the transformation equations for energy, momentum, and angular momentum from one inertial frame $K$ to another $K'$.

\i Write down the general expression for the energy function $E$.

\i Explain what it means for a function to be homogeneous of degree $k$.

\i Write down the expression for the generalized momentum $p_i$.

\i Write down the expression for the center of mass (COM) of a system of particles.

\i Write down the virial theorem for a system whose motion takes place in a finite region of space and whose potential energy is a homogoneous function of degree $k$.

\een

%%%%%%%%%%%%%%%%%%%%%%%%%%%%%%%%%%%%%%%%%%%%%%%%%%%%%%
\section{Hamiltonian mechanics (\S40)}

\ben

\i Write down the Hamiltonian $H(p,q,t)$ for a simple system starting from a Lagrangian $L(q,\dot q,t)$.

\i Write down Hamilton's equations for $p_i$ and $q_i$.

\i Explain the fundamental difference between Hamilton's equations and Lagrange's equations.

\i Show the equivalence of Hamilton's equations and Lagrange's equation for simple systems.

\een

%%%%%%%%%%%%%%%%%%%%%%%%%%%%%%%%%%%%%%%%%%%%%%%%%%%%%%%%%%%%%%%%%%%%%%
\section{Central force motion (\S11, 13-15)}

\ben

\i Write down an integral expression for $t$ in terms of $x$ for 1-d motion in a constant external field $U(x)$.

\i Determine the allowed values of the energy and turning points for 1-d motion in a constant external field.

\i Transform the problem of two interacting particles into an effective one-body problem by working in the COM frame.

\i Show that both energy and angular momentum are conserved for a central potential.

\i Write down an expression for the effective potential $U_{\rm eff}(r)$ in terms of $U(r)$ and $\ell$.

\i Plot the effective potential for some simple central force potentials.

\i From the graph of the effective potential, determine the different types of allowed motion.

\i Write down integral expressions for $t$ and $\phi$ in terms of $r$ for a general central potential.

\i Evaluate these two integrals for Kepler's problem for bound orbits, using appropriate trig substitutions.

\i Derive the relationship between $E$, $\ell$, $a$, $b$, $e$, and $p$ for an ellipse.

\i State the only two central potentials that have closed bound orbits.

\i State and derive Kepler's three laws of planetary motion.

\i Explain the difference in $E$ and $e$ for elliptical, parabolic, and hyperbolic motion.

\een

%%%%%%%%%%%%%%%%%%%%%%%%%%%%%%%%%%%%%%%%%%%%%
\section{Collisions and scattering (\S16-20)}

\ben

\i Draw diagrams relating velocities in the lab and COM frames for the disintegration of a single particle.

\i Draw diagrams relating the momenta in the lab and COM frames for an elastic collision of two particles ($m_2$ initially at rest in the lab frame).

\i Explain what information can and cannot be obtained for an elastic collison of two particles, using just conservation of momentum and kinetic energy.

\i Derive formulas relating the scattering angles $\chi$, $\theta_1$, $\theta_2$ in the COM and lab frames.

\i Draw diagrams showing how the scattering angle $\chi$ is related to the angle of closest approach $\phi_0$.

\i Relate the impact parameter $\rho$ and initial velocity $v_\infty$ to the energy $E$ and angular momentum $\ell$.

\i Derive an integral expression for $\phi_0$ and solve it for simple potentials---e.g., $U(r) = \alpha/r$ for Rutherford scattering.

\i Write down expressions for ${\rm d}\sigma$ in terms of ${\rm d}\rho$, ${\rm d}\chi$, ${\rm d}\theta_1$, ${\rm d}\theta_2$, or ${\rm d}\Omega$, ${\rm d}\Omega_1$, ${\rm d}\Omega_2$.

\i Explain how one can obtain an expression for small-angle scattering starting from the integral equation for $\phi_0$.

\een

%%%%%%%%%%%%%%%%%%%%%%%%%%%%%%%%%%%%%%%%%%%%%
\section{Small oscillations (\S21-23)}

\ben

\i Explain what stable equilibrium means in terms of the potential energy $U(q)$.

\i Calculate the frequency for small oscillations about a position of stable equilibrium.

\i Solve the equations of motion for both free and forced oscillation in one dimension, noting the difference between the general solution of the homogeneous equation and a particular integral of the inhomogeneous equation.
 
\i Calculate the normal mode frequencies and normal mode solutions for small oscillations of systems with more than one DOF.

\een

%%%%%%%%%%%%%%%%%%%%%%%%%%%%%%%%%%%%%%%%%%%%%
\section{Rigid body motion (\S31-36, 38)}

\ben

\i Draw a diagram showing the body frame and fixed inertial reference frame.

\i Show that the angular velocity vector is unchanged under a shift of the origin of the body frame.

\i Write down an expression for the components  $I_{ik}$ of the inertia tensor as a sum over discrete mass points or as an integral over the volume of the body.

\i Indicate how the components of the inertia tensor change if you shift the origin of the body frame.

\i Obtain or identify the principal axes of inertia for various rigid bodies.

\i Calculate the principal moments of inertia for various rigid bodies.

\i Calculate the kinetic energy of a rigid body in terms of its COM motion and rotational kinetic energy.

\i Write down an expression for the angular momentum vector ${\bf M}$ in terms $I_{ik}$ and ${\Omega}_i$.

\i Write down the equations of motion for a rigid body with respect to an inertial frame.

\i Derive Euler's equations for rigid body motion (equations of motion in the body frame).

\i Draw a diagram showing the definition of the Euler angles $(\phi,\theta,\psi)$.

\i Calculate the components of $\boldsymbol{\Omega}$ wrt the body frame in terms of the Euler angles and their time derivatives.

\i Solve for the reaction forces for rigid bodies in static equilibrium.

\een

%%%%%%%%%%%%%%%%%%%%%%%%%%%%%%%%%%%%%%%%%%%%%%%%%%%%%%
\section{Non-inertial reference frames (\S39)}

\ben

\i Draw a diagram relating an inertial and non-inertial reference frame.

\i Write down the relationship between velocity vectors in inertial and non-inertial reference frames.

\i Distinguish non-inertial reference frames associated with translational and rotational motion.

\i Derive the Coriolis, centrifugal, translational acceleration, and rotational acceleration fictitious force terms.

\i Explain the physical significance of Foucault's pendulum.


\een

\end{document}
